\documentclass[leqno]{article}  % equation numbers on left
\usepackage[utf8]{inputenc}
\usepackage[english,russian]{babel}
\usepackage{indentfirst}
\usepackage{enumerate}
\usepackage{amsmath,amsfonts,amssymb}
\usepackage{multicol}  % for 'multicols' env.
%\usepackage{hyperref}
\usepackage{graphicx,color}
\usepackage{tikz}
\usepackage{pgfplots}
\usepackage{pgfplotstable}
\usepackage{varioref}
% recommended:
%\usepackage{booktabs}
%\usepackage{array}
%\usepackage{colortbl}
\pgfplotsset{compat=1.7}

\title{\renewcommand*{\thefootnote}{\arabic{footnote}}
Теория производства{}\footnotemark}
\author{\renewcommand*{\thefootnote}{\fnsymbol{footnote}}
Чарльз\,У.~Кобб\thanks{Амхерстский колледж} \and Пол\,Х.~Дуглас\thanks{Чикагский университет}}
\begin{document}
\maketitle
\setcounter{footnote}{1}
\section{Введение}
\footnotetext{г-н Дуглас ответственен за §§ 1--5 и §§ 8--10 данной статьи, и г-н Кобб ответственен за §§ 6--7.}
Последовательное совершенствование методов исчисления физического объёма промышленного производства, происходившее в последние годы, делает насущным необходимость решения следующих задач: (1) определить изменения в объёмах труда и капитала, которые были затрачены для производства соответствующих товаров, и (2) определить характер взаимосвязи между тремя агрегатами: трудом, капиталом и производством.
\par
Даже при том, что относительные предложения труда и капитала от года к году можно установить хотя бы приблизительно, неизбежно возникает ряд дальнейших вопросов, требующих своих ответов, следующие из которых являются типовыми. 
\begin{enumerate}[{(1)}]
\item Возможно ли в рамках разумного оценить, является ли такой прирост производства совершенно случайным, происходил ли он в основном за счёт технологий, а также в какой мере он реагирует (если вообще реагирует) на изменения в количествах труда и капитала?
\item Возможно ли определить, опять же в рамках разумного, \emph{относительное} влияние объёмов труда и капитала на объёмы производства?
\item Возможно ли, с учётом того, что соотношение объёмов труда и капитала меняется от года к году, определить вклад каждой единицы труда и капитала в совокупном количестве физического производства и, что еще более важно, \emph{конечных} единиц труда и капитала по каждому рассмотренному году?
\par
Возможно ли по меньшей мере выдержать ретроспективный подход к теориям убывающей вменённой производительности (уменьшающегося прироста совокупного продукта) и открыть путь к дальнейшим попыткам привязать количественные приближения к предполагаемым закономерностям, в случае если действительно будет обнаружено их статистическое обоснование?
\item Возможно ли измерить реальные углы наклона кривых, связывающих приращения объёмов производства с приращением объёмов труда и с приращением объёмов капитала соответственно, и тем самым придать большую определённость тому, что в настоящее время является исключительно гипотезой без количественных обоснований?
\item И наконец, по результатам такого исследования (a) объёмов физического производства от года к году, приписываемых условным единицам труда и капитала соответственно, в логической связке с (б) исследованием относительной меновой стоимости натуральной единицы абстрактных промышленных товаров в пределах рассматриваемого периода и в сопоставлении с (в) фактическим изменением <<реальной>> заработной платы в обрабатывающей промышленности и реальной ставки процента (если изменение последней может быть установлено), возможно ли пролить свет на вопрос о том, возможно ли в принципе моделировать процессы распределения достаточно точно посредством процессов создания стоимостей?
\end{enumerate}
\par
В данной статье предпринимается попытка рассмотреть поставленные выше вопросы, а также привнести в них чуть больше ясности. Однако прежде, чем сделать это, необходимо, конечно же, построить ряды относительных объёмов труда и капитала, которые были использованы в работе над данной статьёй, чему и посвящаются два следующих параграфа; при этом дальнейшие параграфы посвящаются рассмотрению потенциальных взаимосвязей между ними.
\section{Рост объёмов основного капитала в обрабатывающей промышленности в САСШ в 1899--1922\,гг.}
Промышленные переписи периодически содержали в своих листах вопрос о капиталовложениях в различные предприятия обрабатывающей промышленности, а также содержали в своих результирующих таблицах ответы респондентов на этот вопрос. Такие ответы, тем не менее, включали в себя, помимо данных по основному капиталу в форме механизмов и зданий, данные по оборотному капиталу, включая сырьё и материалы, незавершённое производство и товары на складах. Также такие ответы включали данные по земельным участкам. Поскольку мы предпринимаем попытки оценить размеры такого капитала, который способствует производству товаров, нам следует исключить из рассмотрения оборотный капитал, так как последний является результатом, а не предпосылкой производственного процесса\footnote{Конечно, оборотный капитал обычно <<производит>> стоимость для своего обладателя, однако мы рассматриваем здесь не стоимость, но физическое производство.}. Нам также следует исключить стоимости земельных участков, поскольку они в значительной мере состоят из приращённой стоимости. Следовательно, нам следует попытаться оценить изменения физических объёмов (1) механизмов, орудий труда и оборудования и (2) производственных зданий и сооружений.
\par
К сожалению, несмотря на то, что статистические данные по совокупному капиталу представлены практически по всем переписным годам, они дополнялись разбивкой на указанные группы только в 1889, 1899 и 1904\,гг.\footnote{См. \emph{13-я Перепись населения} 1900\,г., VI, с. xcvii, и \emph{Промышленная перепись} 1904\,г., часть I, сс. lxiv--lxv.}  Бюро переписи населения в своём отчёте о \emph{<<Национальном богатстве, государственном долге и налогообложении>>} за 1922\,г. даёт оценку доли производственных механизмов, орудий труда и оборудования в совокупном объёме производственного капитала на уровне 30\,\%\footnote{Бюро переписи населения, Оценка национального богатства (1925\,г.), сс. 9--10.}. Поскольку Бюро определило стоимость производственного капитала в размере 52 610 млн долл., это означает, что стоимость производственных механизмов, орудий труда и оборудования составляет 15 783 млн долл.
\par



\begin{table}[!h]
\centering
\footnotesize{
\begin{tabular}{p{0.15\textwidth}|p{0.15\textwidth}|p{0.15\textwidth}|p{0.15\textwidth}|p{0.15\textwidth}}
\hline
\label{t:0}
Год & Стоимость производственных зданий и сооружений, млн\,долл. & Доля в совокупном объёме производственного капитала,\,\% & Стоимость механизмов, орудий труда и оборудования, млн\,долл. & Доля в совокупном объёме производственного капитала,\,\% \\ \hline
1889 & \hfill 879 \hspace*{1mm} & \hfill 13,4 \hspace*{1mm} & \hfill 1 584 \hspace*{1mm}  & \hfill 24,3 \hspace*{1mm} \\
1899 & \hfill 1 450 \hspace*{1mm} & \hfill 14,8 \hspace*{1mm} & \hfill 2 543 \hspace*{1mm} & \hfill 25,9 \hspace*{1mm} \\
1904 & \hfill 1 996 \hspace*{1mm} & \hfill 15,8 \hspace*{1mm} & \hfill 3 490 \hspace*{1mm} & \hfill 27,5 \hspace*{1mm} \\
1922 & & & \hfill 15 783 \hspace*{1mm} & \hfill 30,0\footnotemark\ \hspace*{0.005mm} \\ \hline
\end{tabular}
}
\end{table}

\footnotetext{Оценка Бюро переписи населения.}


\par
Стоимости, которые таким образом были соотнесены с каждой из указанных групп капитала, а также доли, которые они составляли от размера совокупного капитала в каждом из указанных годов, приведены на странице \vref{t:0}.
\par
Приведённые статистические данные служат базой оценки вероятных стоимостей указанных групп производственного капитала в тех годах, когда разбивка на такие группы не производилась. Увеличивался не только совокупный объём капитала, но и основной капитал постепенно составлял большую долю от указанной увеличивающейся суммы.
\par
Очевидно, что количество зданий и механизмов не увеличивалось столь же стремительно по сравнению с размером оборотного капитала в течение 1880-х\,гг., как это происходило в течение пятнадцати лет, последовавших за 1889\,г., когда доля зданий возросла с 13,4\,\% до 15,8\,\%, прирост на 2,4\,\%, и доля механизмов, орудий труда и оборудования возросла с 24,3\,\% до 27,5\,\%, прирост на 3,2\,\%. Таким образом, ежегодный прирост данных показателей составил 0,16\,\% и 0,21\,\% соответственно. Мы предположили, что темпы роста долей от размера совокупного капитала, которые приходятся на здания и механизмы, в 1890-х\,гг. составляли не более чем одной четвёртой и одной пятой (одной седьмой -- \emph{Прим. перев.}) долей процента соответственно. Таким образом, возможные доли от совокупного капитала, которые приходятся на здания и механизмы в 1879\,г., составляют 13,0\,\% и 24,0\,\% соответственно.
\par
Если мы примем оценку Бюро переписи населения на уровне 30\,\% в качестве оценки доли от размера совокупного капитала, которая приходилась на механизмы в 1922\,г. -- в последнем году рассматриваемого периода, -- мы сможем достаточно равномерно распределить прирост в 2,5\,\% с уровня 27,5\,\% в 1904\,г. Тем не менее, начиная с 1914\,г. ежегодные темпы прироста безусловно были несколько более высокими, чем в течение предыдущего десятилетия, и это обстоятельство следует принимать в расчёт.
\par
Построение временного ряда прироста относительного значения зданий в совокупном капитале, начиная с 1904\,г., представляется более затруднительным, т. к. нам неизвестна конечная величина, от которой необходимо отталкиваться. Несмотря на то, что абсолютные приросты были весьма внушительными, мы отвергли предположение о том, что относительное значение зданий по сравнению с другими группами капитала росло такими же темпами, что и в течение 1889--1904\,гг. На основе предыдущего рассуждения, а также исследования по штату Миссури, мы оценили соответствующее значение на уровне 16,5\,\% для 1922\,г. и распределили приросты по предыдущим годам также с учётом их более высоких темпов, начиная с 1914\,г. В Таблице I представлены оценки долей каждой из указанных групп производственного капитала по различным годам, а также соответствующие значения, выраженные в долларах.
\par
Некоторые данные указывают на то, что оценка суммы долей зданий и механизмов в совокупном производственном капитале на уровне 46,5\,\% не далека от истины. Так, согласно Федеральному бюро статистики труда штата Миссури, в 1923\,г. в указанном штате вложения в производственные здания, механизмы и т. д. составили 334,7 млн долл., и 58,7 млн долл. -- в <<территории>>\footnote{\emph{Сорок четвёртый ежегодный отчёт Федерального бюро статистики труда штата Миссури}. (1923\,г.), с. 155.}. Точное абсолютное значение оборотного капитала для указанного штата не приводится, однако относительное значение этого показателя приводится Федеральной торговой комиссией для страны в целом на уровне 45,7\,\% от размера совокупного капитала\footnote{Федеральная торговая комиссия, \emph{<<Национальное богатство и доходы>>}, с. 135. (Документ Сената 126, 69-й Конгресс САСШ, 1-я сессия).}. Поскольку указанная цифра основана на данных по 54 862 предприятиям с совокупным капиталом в размере 33,65 млрд долл., её можно принять в качестве наилучшей оценки соответствующего показателя в масштабах всей страны, которая нам доступна. Если мы применим указанное значение к данным штата Миссури, мы получим 331,1 млн долл. или в сумме с остальными формами капитала -- 724,6 млн долл. Таким образом, стоимость зданий, механизмов и оборудования, как указано выше, независимо оценивается в исследовании по штату Миссури на уровне 334,7 млн долл., что составляет 46,2\,\% от стоимости совокупного производственного капитала. Последнее значение находится в практически точном соответствии с оценкой на уровне 46,5\,\%, которую мы получили для указанных групп капитала для 1922\,г. Поскольку структуру промышленности штата Миссури нельзя считать нерепрезентативной\footnote{Так, например, при том, что в штате Миссури нет текстильной промышленности, за исключением небольшой швейной промышленности, значительный объём капитала в этом штате инвестирован в полиграфическую промышленность, литьё металлов, производство автомобилей, мясохладобойные, плавильные и кирпичные заводы, а также заводы по производству извести. Присутствует также быстрорастущая обувная промышленность.} по отношению к стране в целом, нашу оценку можно считать обоснованной и до тех пор, пока более качественные статистические данные не получены, вероятно наилучшей из всех, что могут быть сделаны.

\begin{table}[!h]
\centering
\footnotesize{
\caption{Оценка стоимостей производственных зданий и механизмов, орудий труда и оборудования, а также доли от размера совокупного производственного капитала, которые приходились на данные группы, 1879--1922\,гг.}%
\label{tab1}%
\begin{tabular}{p{0.12\textwidth}|p{0.12\textwidth}|p{0.12\textwidth}|p{0.12\textwidth}|p{0.12\textwidth}|p{0.12\textwidth}}
\hline
&  \multicolumn{2}{p{0.24\textwidth}|}{\centering Доля от размера совокупного производственного капитала,\,\%}  &  \multicolumn{3}{p{0.36\textwidth}}{\centering Стоимость, млн долл.}  \\

\cline{2-6}
\multicolumn{1}{c|}{Год} & Здания & Механиз-мы и оборудование & Здания & Механиз-мы и оборудование & Всего \\
\hline
1879 & \centering 13,0 & \centering 24,0 & \hfill 363 \hspace*{1mm} & \hfill 670 \hspace*{1mm} & \hfill 1 033  \hspace*{1mm} \\
1889 & \centering 13,4 & \centering 24,3 & \hfill 879 \hspace*{1mm} & \hfill 1 584 \hspace*{1mm} & \hfill 2 463  \hspace*{1mm} \\
1899 & \centering 14,8 & \centering 25,9 & \hfill 1 450 \hspace*{1mm} & \hfill 2 543 \hspace*{1mm} & \hfill 3 993  \hspace*{1mm} \\
1904 & \centering 15,8 & \centering 27,5 & \hfill 1 996 \hspace*{1mm} & \hfill 3 490 \hspace*{1mm} & \hfill 5 486  \hspace*{1mm} \\
1909 & \centering 16,0 & \centering 28,1 & \hfill 2 948 \hspace*{1mm} & \hfill 5 178 \hspace*{1mm} & \hfill 8 126  \hspace*{1mm} \\
1914 & \centering 16,2 & \centering 28,7 & \hfill 3 692 \hspace*{1mm} & \hfill 6 541 \hspace*{1mm} & \hfill 10 233  \hspace*{1mm} \\
1919 & \centering 16,4 & \centering 29,5 & \hfill 7 293 \hspace*{1mm} & \hfill 13 118 \hspace*{1mm} & \hfill 20 411  \hspace*{1mm} \\
1922 & \centering 16,5 & \centering 30,0 & \hfill 8 681 \hspace*{1mm} & \hfill 15 783 \hspace*{1mm} & \hfill 24 464  \hspace*{1mm} \\
\hline
\end{tabular}
}
\end{table}

%\begin{tabular}{0.8\textwidth}{|c|c|c|r@{\,}l|r@{\,}l|r@{\,}l}
%\multicolumn{1}{|c|}{Год} & \multicolumn{2}{c|}{Доля от общей стоимости капитала,\,\%} & \multicolumn{3}{c|}{Стоимость, млн\,долл.} \\
%\multicolumn{1}{|c|}{} & \multicolumn{1}{c|}{Здания и сооружения} & \multicolumn{1}{c|}{Машины и оборудование} & \multicolumn{1}{c|}{Здания и сооружения} & \multicolumn{1}{c|}{Машины и оборудование} & \multicolumn{1}{c|}{Всего} \\ \hline
%1879 & 13,0 & 24,0 & & 363 & & 670 & 1 & 033 \\
%1889 & 13,4 & 24,3 & & 879 & 1 & 584 & 2 & 463 \\
%1899 & 14,8 & 25,9 & 1 & 450 & 2 & 543 & 3 & 993 \\
%1904 & 15,8 & 27,5 & 1 & 996 & 3 & 490 & 5 & 486 \\
%1909 & 16,0 & 28,1 & 2 & 948 & 5 & 178 & 8 & 126 \\
%1914 & 16,2 & 28,7 & 3 & 692 & 6 & 541 & 10 & 233 \\
%1919 & 16,4 & 29,5 & 7 & 293 & 13 & 118 & 20 & 411 \\
%1922 & 16,5 & 30,0 & 8 & 681 & 15 & 783 & 24 & 464 \\ \hline
%\end{tabular}

\par
Вместе с тем остаётся резонный вопрос относительного того, как следует трактовать имеющиеся данные переписи и насколько качественными являются исходные данные. В течение последних лет Бюро переписи населения инструктировало своих агентов следить за тем, чтобы эти статистические данные принимались <<по балансовой стоимости>>. Однако остаётся неясным, какая стоимость зданий, механизмов и т.\,д. принимается в качестве балансовой -- первоначальная или восстановительная? Г-н Ла-Верн-Билз, главный статистик по обрабатывающей промышленности, который возможно является наиболее компетентным специалистом в этой области, прокомментировал\footnote{Письмо автору, 23 октября 1925\,г.}, что <<производители, как правило, предоставляли данные по капиталу скорее на базе первоначальной, а не восстановительной стоимости>>.
\par
Действительно, Бюро переписи населения часто официально предостерегали от безоговорочного принятия их данных по размерам совокупного производственного капитала и по факту не включали соответствующий вопрос в свои листы для переписей 1921, 1923 и 1925\,гг. Впрочем, при том, что трудности, возникающие в силу того, что капиталовложения рассчитаны в ценах тех различных периодов, в которых они были фактически произведены, являются разрешимыми, а также при том, что ряд объёмов капитала можно таким образом привести к постоянным ценам, представляется, что после проделанного не останется веской причины не принимать полученные данные в качестве достаточно надёжной оценки ряда не абсолютного роста основного капитала, но \emph{относительного роста} последнего. Более того, должная коррекция искажений, создаваемых изменяющимися уровнями цен, устранит большинство возражений, которые могут быть предъявлены против такого рода количественных показателей, используемых в качестве оценки совокупного объёма капитала. Прежде чем мы сможем построить безразрывный и сопоставимый ряд, останутся ещё две задачи: (1) найти вероятные приросты для каждого из промежуточных годов и (2) привести полученные разные приросты основного капитала к сопоставимым ценам.
\par
Поскольку статистические данные предоставляются на базе первоначальной стоимости, первая задача заключается в нахождении ежегодных приростов капитала, выраженных в ценах того года, в котором они имели место быть, и в сложении полученных значений со значениями предшествующего года. В сжатой форме использованный подход заключался в следующем: (1) установить объёмы производства по каждому году из интервала 1899--1922\,гг. по следующим материалам: чугун, катаная и кованая сталь, древесина, кокс, цемент, кирпичи и медь\footnote{Исходные данные были получены из Статистических ежегодников САСШ для различных годов. Также \emph{<<Полезные ископаемые САСШ>>}, 1921\,г. Часть I, сс. 235--82; 565--98; Часть II, сс. 371--440.}. Необходимо отметить, что указанные материалы являются наиболее важными из тех, которые используются для производства механизмов и строительства зданий. В тех немногих случаях, когда невозможно было получить реальное значение производства по конкретному году, такое значение оценивалось по данным других годов и на основании относительного изменения индекса физического объёма производства профессора Дэя для той товарной группы, к которой относился рассматриваемый материал\footnote{Э. Э. Дэй, <<Индекс физического объёма производства>>, \emph{Review of Economic Statistics}, Том II (1920\,г.), сс. 328--29.
\par
Э. Э. Дэй, <<Физический объём производства САСШ за 1923\,г.>>
\par
Там же. Том VI (1924\,г.), с. 201.}. Для периода 1880--1889\,гг. были использованы данные по объёмам производства чугуна, стали, кокса, цемента и меди. (2) Объёмы производства по каждому товару по каждому году были умножены на соответствующие текущие цены за единицу товара\footnote{В качестве цены на древесину было использовано среднее из цен на ель и клён. }. Для периода 1890--1922\,гг. были использованы цены, полученные и опубликованные Федеральным бюро статистики труда САСШ\footnote{Статистический бюллетень 335 Федерального бюро статистики труда САСШ, \emph{<<Оптовые цены>>}, 1890--1922\,гг., сс. 126--56.}, и для десятилетия 1880--1890\,гг. были использованы цены, опубликованные в отчётах
комитета Сената САСШ по финансам под председательством Олдрича\footnote{\emph{<<Отчёт комитета Сената САСШ по Оптовым ценам, по заработной плате, а также по перевозкам>>}, Приложение A. Критические замечания в адрес индекса цен здесь неприменимы, поскольку были использованы абсолютные значения цен, приведённые в соответствующих источниках.}. В отдельных случаях возможно было непосредственно установить значение совокупного продукта так, чтобы не было необходимости умножать объём физического производства на цену за единицу товара, и во всех таких случаях было использовано приведённое в соответствующих источниках значение совокупного продукта. (3) Стоимостные объёмы производства каждого товара за конкретный год затем были просуммированы с тем, чтобы получить совокупные стоимостные объёмы выпуска данных товаров производственного назначения по каждому году. (4) Стоимости данных товаров производственного назначения, которые были произведены в период между двумя переписными годами затем были просуммированы (например, с 1880\,г. по 1889\,г. включительно), и стоимости по каждому году разделили на совокупную стоимость за такой период с тем, чтобы получить их удельные веса в такой совокупной стоимости за такой период. Затем полученные удельные веса были умножены на совокупный прирост стоимости зданий и механизмов за тот же период, и тем самым были получены оценки годовых приростов стоимостей данных товаров производственного назначения.
\par
Описанная процедура может быть проиллюстрирована следующим примером. Прирост стоимости зданий и механизмов в период между 1879\,г. и 1889\,г. составил 1430 млн долл. Совокупные номинальные стоимости указанных товаров производственного назначения по каждому году из указанного периода, а также удельные веса совокупных номинальных стоимостей по каждому году в совокупной номинальной стоимости за весь указанный период были следующими:
\par
\begin{table}[htbp]
\centering
\footnotesize{
\begin{tabular}{p{0.25\textwidth}|p{0.25\textwidth}|p{0.25\textwidth}}
\hline
\centering Год & Стоимость указанных средств производства, млн\,долл. & Доля года в десятилетии,\,\% \\
\hline
1880 & \hfill 200 \hspace*{1cm} & \hfill 9,6 \hspace*{1cm} \\
1881 & \hfill 210 \hspace*{1cm} & \hfill 10,0 \hspace*{1cm} \\
1882 & \hfill 216 \hspace*{1cm} & \hfill 10,3 \hspace*{1cm} \\
1883 & \hfill 184 \hspace*{1cm} & \hfill 8,8 \hspace*{1cm} \\
1884 & \hfill 148 \hspace*{1cm} & \hfill 7,1 \hspace*{1cm} \\
1885 & \hfill 141 \hspace*{1cm} & \hfill 6,7 \hspace*{1cm} \\
1886 & \hfill 211 \hspace*{1cm} & \hfill 10,0 \hspace*{1cm} \\
1887 & \hfill 282 \hspace*{1cm} & \hfill 13,5 \hspace*{1cm} \\
1888 & \hfill 241 \hspace*{1cm} & \hfill 11,5 \hspace*{1cm} \\
1889 & \hfill 263 \hspace*{1cm} & \hfill 12,5 \hspace*{1cm} \\
& \hfill  \hrulefill\ \hspace*{1cm} & \hfill  \hrulefill\ \hspace*{1cm} \\
\centering Итого & \hfill 2 096 \hspace*{1cm} & \hfill 100,0 \hspace*{1cm} \\
\hline
\end{tabular}
}
\end{table}

%\pgfplotstabletypeset
%[col sep=&,row sep=\\,sci zerofill]
%{
%period & val & per\\
%1880 & 200.0 & 9.6\\
%1881 & 210.0 & 10.0\\
%1882 & 216.0 & 10.3\\
%1883 & 184.0 & 8.8\\
%1884 & 148.0 & 7.1\\
%1885 & 141.0 & 6.7\\
%1886 & 211.0 & 10.0\\
%1887 & 282.0 & 13.5\\
%1888 & 241.0 & 11.5\\
%1889 & 263.0 & 12.5\\
%Total & 2096.0 & 100.0\\
%}

\par
Полученные удельные веса затем были умножены на абсолютный прирост стоимости зданий и механизмов в течение десятилетия, 1430 млн долл., и тем самым были получены оценки годовых приростов стоимости зданий и механизмов. Совокупная стоимость зданий и механизмов за 1879\,г., взятая вместе с суммой полученных годовых приростов, с необходимостью составит их совокупную стоимость за 1889\,г. Разумеется, основное предположение состоит в том, что объём основного капитала на базе первоначальной стоимости увеличивался от года к году соразмерно номинальной стоимости производства инвестиционных товаров.
\par
Как бы то ни было, поскольку полученные оценки прироста основного капитала исчислены в ценах рассматриваемых периодов, необходимо устранить эффект изменяющихся уровней цен, если мы желаем получить ряд относительных объёмов реального капитала.
\par
В связи с этим был рассчитан дефлятор капитала на основе трёх рядов относительных цен: (1) оптовых цен на металлы и металлические изделия, (2) оптовых цен на строительные материалы и (3) номинальной заработной платы. Отчёт комитета Олдрича был использован для получения цен по первым двум указанным группам товаров за период с 1880\,г. по 1889\,г.\footnote{\emph{<<Отчёт комитета Сената САСШ по Оптовым ценам, по заработной плате, а также по перевозкам>>}, сс. 92--99. Данные по прославленным двадцати пяти разновидностям складных ножей были изъяты из ряда оптовых цен на металлы и металлические изделия перед его дальнейшим использованием.}, тогда как соответствующие ряды данных Федерального бюро статистики труда были использованы для периода с 1890\,г. по 1922\,г.\footnote{Статистический бюллетень 335 Федерального бюро статистики труда САСШ, \emph{<<Оптовые цены>>}, 1890--1922\,гг., сс. 8--9.} Ряд по заработной плате, в свою очередь, составлен из ряда, ранее построенного одним из авторов данной работы, для периода с 1890\,г. и далее\footnote{Пол Х. Дуглас, <<Изменение реальной заработной платы в течение последних лет и его экономическая значимость>>. Приложение, \emph{American Economic Review}, март 1926\,г., с. 30.}, а также из рядов средней заработной платы, которые были построены д-ром Р. П. Фолкнером в рамках отчёта комитета Олдрича и отражающих изменения в течение 1880-х\,гг. Три рассмотренных ряда затем были сведены к базисным индексам, для которых уровень 1880\,г. был принят за 100, после чего на их основе
был исчислен средневзвешенный индекс. При этом перечисленным группам были присвоены следующие веса: для металлов и металлических изделий -- 4; для строительных материалов -- 2; и для заработной платы -- 3.
\par

\begin{table}
\centering
\footnotesize{
\caption{Оценки ежегодных приростов основного капитала в обрабатывающей промышленности и кумулятивного совокупного капитала в соответствии с первоначальной стоимостью, а также в ценах 1880\,г., 1899--1922\,гг., млн\,долл.}%
\label{tab2}%
\begin{tabular}{p{0.12\textwidth}|p{0.12\textwidth}|p{0.12\textwidth}|p{0.12\textwidth}|p{0.12\textwidth}|p{0.12\textwidth}}
\hline
Год & Ежегодный прирост в соответствии с первоначальной стоимостью & Дефлятор капитала (1880\,г. = 100) & Ежегодный прирост в ценах 1880\,г. & Совокупный основной капитал в ценах 1880\,г. & Относи-тельный объём совокупного капитала (1899\,г. = 100) \\
& (1) & (2) & (3) & (4) & (5) \\
\hline
1899 & \hfill 339 \hspace*{2.5mm} & \hfill 88 \hspace*{2.5mm} & \hfill 387 \hspace*{2.5mm} & \hfill 4 449 \hspace*{2.5mm} & \hfill 100 \hspace*{2.5mm} \\
1900 & \hfill 264 \hspace*{2.5mm} & \hfill 89 \hspace*{2.5mm} & \hfill 297 \hspace*{2.5mm} & \hfill 4 746 \hspace*{2.5mm} & \hfill 107 \hspace*{2.5mm} \\
1901 & \hfill 277 \hspace*{2.5mm} & \hfill 88 \hspace*{2.5mm} & \hfill 315 \hspace*{2.5mm} & \hfill 5 061 \hspace*{2.5mm} & \hfill 114 \hspace*{2.5mm} \\
1902 & \hfill 342 \hspace*{2.5mm} & \hfill 89 \hspace*{2.5mm} & \hfill 383 \hspace*{2.5mm} & \hfill 5 444 \hspace*{2.5mm} & \hfill 122 \hspace*{2.5mm} \\
1903 & \hfill 328 \hspace*{2.5mm} & \hfill 91 \hspace*{2.5mm} & \hfill 362 \hspace*{2.5mm} & \hfill 5 806 \hspace*{2.5mm} & \hfill 131 \hspace*{2.5mm} \\
1904 & \hfill 282 \hspace*{2.5mm} & \hfill 87 \hspace*{2.5mm} & \hfill 326 \hspace*{2.5mm} & \hfill 6 132 \hspace*{2.5mm} & \hfill 138 \hspace*{2.5mm} \\
1905 & \hfill 457 \hspace*{2.5mm} & \hfill 92 \hspace*{2.5mm} & \hfill 494 \hspace*{2.5mm} & \hfill 6 626 \hspace*{2.5mm} & \hfill 149 \hspace*{2.5mm} \\
1906 & \hfill 612 \hspace*{2.5mm} & \hfill 100 \hspace*{2.5mm} & \hfill 611 \hspace*{2.5mm} & \hfill 7 237 \hspace*{2.5mm} & \hfill 163 \hspace*{2.5mm} \\
1907 & \hfill 629 \hspace*{2.5mm} & \hfill 106 \hspace*{2.5mm} & \hfill 595 \hspace*{2.5mm} & \hfill 7 832 \hspace*{2.5mm} & \hfill 176 \hspace*{2.5mm} \\
1908 & \hfill 373 \hspace*{2.5mm} & \hfill 94 \hspace*{2.5mm} & \hfill 397 \hspace*{2.5mm} & \hfill 8 229 \hspace*{2.5mm} & \hfill 185 \hspace*{2.5mm} \\
1909 & \hfill 569 \hspace*{2.5mm} & \hfill 96 \hspace*{2.5mm} & \hfill 591 \hspace*{2.5mm} & \hfill 8 820 \hspace*{2.5mm} & \hfill 198 \hspace*{2.5mm} \\
1910 & \hfill 422 \hspace*{2.5mm} & \hfill 100 \hspace*{2.5mm} & \hfill 420 \hspace*{2.5mm} & \hfill 9 240 \hspace*{2.5mm} & \hfill 208 \hspace*{2.5mm} \\
1911 & \hfill 379 \hspace*{2.5mm} & \hfill 99 \hspace*{2.5mm} & \hfill 384 \hspace*{2.5mm} & \hfill 9 624 \hspace*{2.5mm} & \hfill 216 \hspace*{2.5mm} \\
1912 & \hfill 457 \hspace*{2.5mm} & \hfill 103 \hspace*{2.5mm} & \hfill 443 \hspace*{2.5mm} & \hfill 10 067 \hspace*{2.5mm} & \hfill 226 \hspace*{2.5mm} \\
1913 & \hfill 497 \hspace*{2.5mm} & \hfill 110 \hspace*{2.5mm} & \hfill 453 \hspace*{2.5mm} & \hfill 10 520 \hspace*{2.5mm} & \hfill 236 \hspace*{2.5mm} \\
1914 & \hfill 356 \hspace*{2.5mm} & \hfill 101 \hspace*{2.5mm} & \hfill 353 \hspace*{2.5mm} & \hfill 10 873 \hspace*{2.5mm} & \hfill 244 \hspace*{2.5mm} \\
1915 & \hfill 1 017 \hspace*{2.5mm} & \hfill 105 \hspace*{2.5mm} & \hfill 967 \hspace*{2.5mm} & \hfill 11 840 \hspace*{2.5mm} & \hfill 266 \hspace*{2.5mm} \\
1916 & \hfill 1 899 \hspace*{2.5mm} & \hfill 135 \hspace*{2.5mm} & \hfill 1 402 \hspace*{2.5mm} & \hfill 13 242 \hspace*{2.5mm} & \hfill 298 \hspace*{2.5mm} \\
1917 & \hfill 2 891 \hspace*{2.5mm} & \hfill 173 \hspace*{2.5mm} & \hfill 1 673 \hspace*{2.5mm} & \hfill 14 915 \hspace*{2.5mm} & \hfill 335 \hspace*{2.5mm} \\
1918 & \hfill 2 473 \hspace*{2.5mm} & \hfill 183 \hspace*{2.5mm} & \hfill 1 350 \hspace*{2.5mm} & \hfill 16 265 \hspace*{2.5mm} & \hfill 366 \hspace*{2.5mm} \\
1919 & \hfill 1 898 \hspace*{2.5mm} & \hfill 196 \hspace*{2.5mm} & \hfill 969 \hspace*{2.5mm} & \hfill 17 234 \hspace*{2.5mm} & \hfill 387 \hspace*{2.5mm} \\
1920 & \hfill 2 096 \hspace*{2.5mm} & \hfill 237 \hspace*{2.5mm} & \hfill 884 \hspace*{2.5mm} & \hfill 18 118 \hspace*{2.5mm} & \hfill 407 \hspace*{2.5mm} \\
1921 & \hfill 780 \hspace*{2.5mm} & \hfill 184 \hspace*{2.5mm} & \hfill 424 \hspace*{2.5mm} & \hfill 18 542 \hspace*{2.5mm} & \hfill 417 \hspace*{2.5mm} \\
1922 & \hfill 1 177 \hspace*{2.5mm} & \hfill 181 \hspace*{2.5mm} & \hfill 650 \hspace*{2.5mm} & \hfill 19 192 \hspace*{2.5mm} & \hfill 431 \hspace*{2.5mm} \\
\hline
\end{tabular}
}
\end{table}

%\pgfplotstabletypeset
%[col sep=&,row sep=\\,sci zerofill]
%{
%period & col1 & col2 & col3 & col4 & col5\\
%1899 & 339 & 88 & 387 & 4449 & 100\\
%1900 & 264 & 89 & 297 & 4746 & 107\\
%1901 & 277 & 88 & 315 & 5061 & 114\\
%1902 & 342 & 89 & 383 & 5444 & 122\\
%1903 & 328 & 91 & 362 & 5806 & 131\\
%1904 & 282 & 87 & 326 & 6132 & 138\\
%1905 & 457 & 92 & 494 & 6626 & 149\\
%1906 & 612 & 100 & 611 & 7237 & 163\\
%1907 & 629 & 106 & 595 & 7832 & 176\\
%1908 & 373 & 94 & 397 & 8229 & 185\\
%1909 & 569 & 96 & 591 & 8820 & 198\\
%1910 & 422 & 100 & 420 & 9240 & 208\\
%1911 & 379 & 99 & 384 & 9624 & 216\\
%1912 & 457 & 103 & 443 & 10067 & 226\\
%1913 & 497 & 110 & 453 & 10520 & 236\\
%1914 & 356 & 101 & 353 & 10873 & 244\\
%1915 & 1017 & 105 & 967 & 11840 & 266\\
%1916 & 1899 & 135 & 1402 & 13242 & 298\\
%1917 & 2891 & 173 & 1673 & 14915 & 335\\
%1918 & 2473 & 183 & 1350 & 16265 & 366\\
%1919 & 1898 & 196 & 969 & 17234 & 387\\
%1920 & 2096 & 237 & 884 & 18118 & 407\\
%1921 & 780 & 184 & 424 & 18542 & 417\\
%1922 & 1177 & 181 & 650 & 19192 & 431\\
%}

Каждое значение годового прироста стоимости производственных зданий и механизмов затем было разделено на значение дефлятора капитала в соответствующем году, и тем самым был получен ряд <<дефлятированных>> приростов или -- более точно -- ряд приростов, выраженных в ценах 1880\,г. на товары производственного назначения. На следующем и завершающем этапе полученный ряд дефлятированных годовых приростов суммировался нарастающим итогом с оценкой совокупного размера зданий и механизмов для 1879\,г. Весь описанный материал представлен в Таблице 2. Поскольку по другим показателям мы имеем данные только за период 1899--1922\,гг., данные по капиталу по годам, предшествующим 1899\,г., не были включены в указанную таблицу. Стоимости приведены в млн долл.
\par
Полученный индекс несовершенен в том отношении, что он не учитывает замещение выбытия первоначального капитала в соответствии с изменяющимися уровнями цен. Переписная статистика по балансовой стоимости, бесспорно включает замещения выбытия первоначального капитала, совершённые при различных и преимущественно более высоких ценах по сравнению с теми, которые преобладали в периоде, когда производились первоначальные инвестиции в капитал. Следовательно, рост от года к году происходит не только в результате накопления дополнительных приростов капитала, но также частично включает замещение выбытия первоначального капитала при иных уровнях цен на первоначальный капитал по мере того, как последний изнашивается. В силу сказанного выше, значения нашего индекса по большей части являются несколько более высокими, чем следует. Мы надеемся опубликовать в недалёком будущем переработанный индекс, который не будет содержать данную систематическую ошибку. Между тем приведённый индекс предлагается в качестве первого приближения.
\par
Данный индекс, конечно, не отражает краткосрочные колебания объёмов использованного капитала. Таким образом, в расчёт не принимаются как недозагруженность капитала, свойственная периодам спада деловой активности, так и более высокая капиталоёмкость по сравнению с её нормальным уровнем, выражающаяся в большем количестве вечерних смен и т. д., которая свойственна периодам подъёма деловой активности.
\par
Между тем обоснованность полученного индекса относительного объёма капитала оказывается некоторым образом усиленной, если мы сопоставим оценки прироста балансовой стоимости капитала, которые мы получили для САСШ\footnote{Данная колонка не была включена в Таблицу II по причине недостаточности пространства.} за период 1910--1920\,гг. с
приростом совокупного капитала в штате Массачусетс, исчисленного по аналогичному методу\footnote{См. Ежегодные отчёты бюро статистики труда штата Массачусетс, \emph{Статистика обрабатывающей промышленности}, 1910--1920\,гг.}. На странице 146 представлены относительные приросты указанных показателей, при этом 1910\,г. использован в качестве базового периода.
\par
%Год	Массачусетс (Совокупный капитал)	Оценка для САСШ (Основной капитал)
%\pgfplotstabletypeset
%[col sep=&,row sep=\\,sci zerofill]
%{
%period & ma & us\\
%1911 & 105 & 104\\
%1912 & 110 & 110\\
%1913 & 113 & 116\\
%1914 & 130 & 120\\
%1915 & 130 & 132\\
%1916 & 150 & 154\\
%1917 & 188 & 188\\
%1918 & 210 & 217\\
%1919 & 248 & 239\\
%1920 & 250 & 263\\
%}

\begin{table}
\centering
\footnotesize{
\begin{tabular}{p{0.25\textwidth}|p{0.25\textwidth}|p{0.25\textwidth}}
\hline
Год & Массачусетс (Совокупный капитал) & Оценка для САСШ (Основной капитал) \\ \hline
1911 & \hfill 105 \hspace*{1cm} & \hfill 104 \hspace*{1cm} \\
1912 & \hfill 110 \hspace*{1cm} & \hfill 110 \hspace*{1cm} \\
1913 & \hfill 113 \hspace*{1cm} & \hfill 116 \hspace*{1cm} \\
1914 & \hfill 130 \hspace*{1cm} & \hfill 120 \hspace*{1cm} \\
1915 & \hfill 130 \hspace*{1cm} & \hfill 132 \hspace*{1cm} \\
1916 & \hfill 150 \hspace*{1cm} & \hfill 154 \hspace*{1cm} \\
1917 & \hfill 188 \hspace*{1cm} & \hfill 188 \hspace*{1cm} \\
1918 & \hfill 210 \hspace*{1cm} & \hfill 217 \hspace*{1cm} \\
1919 & \hfill 248 \hspace*{1cm} & \hfill 239 \hspace*{1cm} \\
1920 & \hfill 250 \hspace*{1cm} & \hfill 263 \hspace*{1cm} \\ \hline
\end{tabular}
}
\end{table}

\par
Согласованность этих двух индексов проявляется достаточно наглядно, и это становится ещё более очевидным, если мы вспомним о том, что более высокий прирост, который демонстрировали САСШ в целом, объясняется по преимуществу тем, что основной капитал увеличивался более высокими темпами, чем наличный совокупный капитал в обрабатывающей промышленности.
\par
Необходимо отметить, что полученный индекс демонстрирует поистине беспрецедентный рост объёмов основного капитала. Так, объёмы основного капитала практически удвоились в течение десятилетнего периода 1899--1909\,гг. Таким образом, общий среднегодовой темп прироста составил 7\,\%. Темп прироста в течение последующего десятилетия практически соответствовал указанному. Начиная с 1919\,г. темп прироста замедлился в течение трёх последующих годов, однако несмотря на то, что мы не исчисляли темпы роста после 1922\,г., с того момента они вне всякого сомнения преумножились. Если брать рассматриваемый период в совокупности, то можно увидеть, что объёмы промышленного капитала практически удваивались в течение каждого десятилетия, или, иными словами, соответствующий среднегодовой темп прироста составлял примерно 6\,\% с учётом поправки на дополнительные затраты на замещение выбытия первоначального капитала. Считается, что такое значение темпа прироста не наблюдалось ни в одной другой стране мира\footnote{Наш индекс демонстрирует более чем двукратный рост за период с 1879\,г. по 1899\,г., а также прирост около 90\,\% в течение 1890-х\,гг.}. В этой связи следует вспомнить, что Кассель оценил темп прироста объёмов капитала в Западной Европе на уровне 3\,\% в год. Если такая оценка верна, то темп прироста объёмов промышленного капитала в САСШ был вдвое больше указанного, хотя если исчислить темп прироста в расчёте на душу населения, то различие будет ещё более выраженным.
%\section{Рост объёмов предложения труда в \discretionary{1899--}{--1922\,гг.}{1899--1922\,гг.}}
\section{Рост объёмов предложения труда в \\ 1899--1922\,гг.}
Различные промышленные переписи содержат данные по средней численности занятых по каждому переписному году\footnote{А именно 1899, 1904, 1909, 1914, 1919 и 1921\,гг.}. Исходя из таких данных, мы сможем получить вероятные численности занятых по межпереписным годам посредством использования индекса относительной занятости. Данный индекс был построен для периода 1899--1904\,гг. путём объединения статистических данных по относительной численности занятых от года к году в штатах Массачусетс\footnote{См. Ежегодные отчёты по \emph{Статистике обрабатывающей промышленности, Массачусетс}, 1900--1905\,гг.} и Пенсильвания\footnote{См. Отчёты государственного департамента внутренних дел штата Пенсильвания.}. Для периода 1904--1914\,г. данные по штату Пенсильвания заменили данными по штату Нью-Джерси\footnote{Ежегодные тома Бюро труда и промышленности штата Нью-Джерси, \emph{Статистика обрабатывающей промышленности}, 1904--1914\,гг.}. В обоих периодах относительные индексы для каждого штата затем были взвешены по численности занятых согласно данным переписи на начало периода в соответствующем штате, и тем самым был получен сводный индекс.
\par
Затем было сделано предположение о том, что численность занятых по стране в целом изменялась аналогично численностям занятых в двух указанных штатах. В случае если темп изменений в двух указанных штатах для переписного года отличался от темпа по стране в целом, то предполагалось, что такая положительная или отрицательная разница была равномерно распределена по промежуточным годам, а относительное изменение по двум указанным штатам было уменьшено или увеличено соответственно в целях подгонки к данным по стране в целом\footnote{Данный подход идентичен тому, которого придерживался я (Пол Х. Дуглас -- \emph{Прим. перев.}) при интерполировании данных по среднегодовому заработку на межпереписные годы на основе статистики по заработкам в различных штатах.}. Например, прирост численности занятых в 1904\,г. по сравнению с 1899\,г. согласно данным переписей составил 1 066 000, или 21\,\%. Поскольку соответствующий прирост по штатам Массачусетс и Нью-Джерси составил 24\,\%, постольку предполагалось, что разности темпов прироста по стране и по двум указанным штатам увеличивались ежегодно с темпом в одну пятую от 3\,\%, или 0,6\,\%. И далее поскольку соответствующий прирост по штатам Массачусетс и Нью-Джерси в 1900\,г. по сравнению с 1899\,г. составил 4,6\,\%, он был уменьшен до 4,0\,\%. Аналогичный подход применялся к данным последующих годов.
\par
Значения индекса для периода с 1914\,г. по 1919\,г. были получены путём объединения соответствующего индекса Федерального бюро статистики труда\footnote{См. данные журнала \emph{<<Monthly Labor Review>>}.} по ряду отраслей и индекса по штату Нью-Йорк. При этом индексам бюро статистики труда и штата Нью-Йорк были присвоены веса, равные 3 и 1 соответственно\footnote{См. \emph{Статистический бюллетень по рынку труда} штата Нью-Йорк.}. Для периода, начиная с 1919\,г., был использован индекс Совета управляющих Федеральной резервной системой, который в свою очередь в значительной степени был построен на основе индекса Федерального бюро статистики труда. Аналогичный по существу подход применялся для определения вероятной численности занятых по каждому
из межпереписных годов вплоть до 1922\,г.\footnote{Поскольку ряды статистических данных по занятости из указанных источников начинаются только с июля 1914\,г., среднее значение за указанный год было получено путём экстраполяции имеющихся данных на предшествующие шесть месяцев исходя из помесячных колебаний занятости, представленных в промышленной переписи за 1914\,г.} В Таблице III представлены полученные оценки численности занятых, начиная с 1899\,г., выраженные в абсолютных и относительных значениях.

%\begin{tabular}{|l|c|c||l|c|c|}
%1899 & 4 713 & 100 & 1911 & 6 855 & 145\\
%1900 & 4 968 & 105 & 1912 & 7 167 & 152\\
%1901 & 5 184 & 110 & 1913 & 7 277 & 154\\
%1902 & 5 554 & 118 & 1914 & 7 026 & 149\\
%1903 & 5 784 & 123 & 1915 & 7 269 & 154\\
%1904 & 5 468 & 116 & 1916 & 8 601 & 182\\
%1905 & 5 906 & 125 & 1917 & 9 218 & 196\\
%1906 & 6 251 & 133 & 1918 & 9 446 & 200\\
%1907 & 6 483 & 138 & 1919 & 9 096 & 193\\
%1908 & 5 714 & 121 & 1920 & 9 110 & 193\\
%1909 & 6 615 & 140 & 1921 & 6 947 & 147\\
%1910 & 6 807 & 144 & 1922 & 7 602 & 161\\
%\end{tabular}
%\par
%\pgfplotstabletypeset
%[col sep=&,row sep=\\,sci zerofill]
%{
%period & lab & idx & period & lab & idx\\
%1899 & 4713 & 100 & 1911 & 6855 & 145\\
%1900 & 4968 & 105 & 1912 & 7167 & 152\\
%1901 & 5184 & 110 & 1913 & 7277 & 154\\
%1902 & 5554 & 118 & 1914 & 7026 & 149\\
%1903 & 5784 & 123 & 1915 & 7269 & 154\\
%1904 & 5468 & 116 & 1916 & 8601 & 182\\
%1905 & 5906 & 125 & 1917 & 9218 & 196\\
%1906 & 6251 & 133 & 1918 & 9446 & 200\\
%1907 & 6483 & 138 & 1919 & 9096 & 193\\
%1908 & 5714 & 121 & 1920 & 9110 & 193\\
%1909 & 6615 & 140 & 1921 & 6947 & 147\\
%1910 & 6807 & 144 & 1922 & 7602 & 161\\
%}

\begin{table}[!h]
\centering
\footnotesize{
\caption{Оценочная средняя численность работников, занятых на производстве, 1899--1922\,гг.}%
\label{tab3}%
\begin{tabular}{p{0.10\textwidth}|p{0.13\textwidth}|p{0.13\textwidth}||p{0.10\textwidth}|p{0.13\textwidth}|p{0.13\textwidth}}
\hline
Год & Средняя численность занятых, тыс. чел. & Относи\-тельное число (1899\,г. = 100) & Год & Средняя численность занятых, тыс. чел. & Относи\-тельное число (1899\,г. = 100) \\
\hline
1899 & \hfill 4 713 \hspace*{2.5mm} & \hfill 100 \hspace*{2.5mm} & 1911 & \hfill 6 855 \hspace*{2.5mm} & \hfill 145 \hspace*{2.5mm} \\
1900 & \hfill 4 968 \hspace*{2.5mm} & \hfill 105 \hspace*{2.5mm} & 1912 & \hfill 7 167 \hspace*{2.5mm} & \hfill 152 \hspace*{2.5mm} \\
1901 & \hfill 5 184 \hspace*{2.5mm} & \hfill 110 \hspace*{2.5mm} & 1913 & \hfill 7 277 \hspace*{2.5mm} & \hfill 154 \hspace*{2.5mm} \\
1902 & \hfill 5 554 \hspace*{2.5mm} & \hfill 118 \hspace*{2.5mm} & 1914 & \hfill 7 026 \hspace*{2.5mm} & \hfill 149 \hspace*{2.5mm} \\
1903 & \hfill 5 784 \hspace*{2.5mm} & \hfill 123 \hspace*{2.5mm} & 1915 & \hfill 7 269 \hspace*{2.5mm} & \hfill 154 \hspace*{2.5mm} \\
1904 & \hfill 5 468 \hspace*{2.5mm} & \hfill 116 \hspace*{2.5mm} & 1916 & \hfill 8 601 \hspace*{2.5mm} & \hfill 182 \hspace*{2.5mm} \\
1905 & \hfill 5 906 \hspace*{2.5mm} & \hfill 125 \hspace*{2.5mm} & 1917 & \hfill 9 218 \hspace*{2.5mm} & \hfill 196 \hspace*{2.5mm} \\
1906 & \hfill 6 251 \hspace*{2.5mm} & \hfill 133 \hspace*{2.5mm} & 1918 & \hfill 9 446 \hspace*{2.5mm} & \hfill 200 \hspace*{2.5mm} \\
1907 & \hfill 6 483 \hspace*{2.5mm} & \hfill 138 \hspace*{2.5mm} & 1919 & \hfill 9 096 \hspace*{2.5mm} & \hfill 193 \hspace*{2.5mm} \\
1908 & \hfill 5 714 \hspace*{2.5mm} & \hfill 121 \hspace*{2.5mm} & 1920 & \hfill 9 110 \hspace*{2.5mm} & \hfill 193 \hspace*{2.5mm} \\
1909 & \hfill 6 615 \hspace*{2.5mm} & \hfill 140 \hspace*{2.5mm} & 1921 & \hfill 6 947 \hspace*{2.5mm} & \hfill 147 \hspace*{2.5mm} \\
1910 & \hfill 6 807 \hspace*{2.5mm} & \hfill 144 \hspace*{2.5mm} & 1922 & \hfill 7 602 \hspace*{2.5mm} & \hfill 161 \hspace*{2.5mm} \\
\hline
\end{tabular}
}
\end{table}




%\begin{tabular}{|r||r@{--}l|p{1.5in}|} \hline
%\multicolumn{4}{|c|}{Животноводство} \\ \hline\hline
%& \multicolumn{2}{c|}{Цены}
%& \\ \cline{2-3}
%\multicolumn{1}{|c||}{Год}
%& \multicolumn{1}{r@{\,\vline\,}}{мин.}
%& макс. & \multicolumn{1}{c|}{Примечания} \\ \hline
%1971 & 97 & 245 & Неудачный год для фермеров на Западе \\ \hline
%72 & 245 & 245 & {\raggedright Уменьшение продаж\\
%из-за суровой зимы} \\ \hline
%\end{tabular}


\par
Полученный индекс, рассматриваемый в качестве точной метрики рабочей силы, несовершенен в некоторых отношениях. (1) Он не включает в себя информацию по служащим, численность которых увеличивалась примерно в два раза более высокими темпами, чем численность работников. (2) Он основан на данных по человеко-годам, а не на данных по <<стандартным>> человеко-часам. Средняя продолжительность стандартной рабочей недели в течение исследуемого периода снизилась таким образом, что прироста численности персонала было всего лишь достаточно, чтобы компенсировать такое снижение. Один из авторов данной работы построил предварительный ряд стандартных человеко-часов посредством умножения численности работников по каждому году на соответствующие средние количества часов в <<нормальной>> рабочей неделе. Впрочем, имеются основания полагать, что построенный таким образом ряд не был ещё в достаточной мере проработан, поэтому предпочтение было отдано ряду человеко-часов. Мы надеемся включить данные по <<стандартным>> человеко-часам в дальнейшие исследования. (3) Он не показывает отклонения от продолжительностей стандартных рабочих недель, которые возникают вследствие переходов на неполное или сверхурочное рабочее время в периоды спада или подъёма соответственно.
\par
Полученный индекс, разумеется, не учитывает потенциальные изменения квалификации работников или интенсивности их труда. Хотя влияние данных факторов может быть весьма значительным, в настоящий момент не представляется возможным оценить такое влияние количественно с достаточной точностью, и до тех пор, пока такая возможность не представится, чтобы не делать нереалистичные оценки их значимости в рамках какого-либо статистического исследования, их не следует принимать в расчёт. Когда их можно будет определить, тогда и следует их включить.
\section{Рост физического объёма промышленного \\ производства, 1899--1922\,гг.}
В целях нашего исследования мы использовали известный индекс физического объёма производства Э. Э. Дэя
\par

%Таблица IV
\begin{table}
\centering
\footnotesize{
\caption{Индекс физического объёма промышленного производства в САСШ}%
\label{tab4}%
\begin{tabular}{p{0.2\textwidth}|p{0.2\textwidth}||p{0.2\textwidth}|p{0.2\textwidth}}
\hline
Год & Показатель производства & Год & Показатель производства \\
\hline
1899 & \hfill 100 & \hspace*{2.5mm} 1911 & \hfill 153 \hspace*{2.5mm} \\
1900 & \hfill 101 & \hspace*{2.5mm} 1912 & \hfill 177 \hspace*{2.5mm} \\
1901 & \hfill 112 & \hspace*{2.5mm} 1913 & \hfill 184 \hspace*{2.5mm} \\
1902 & \hfill 122 & \hspace*{2.5mm} 1914 & \hfill 169 \hspace*{2.5mm} \\
1903 & \hfill 124 & \hspace*{2.5mm} 1915 & \hfill 189 \hspace*{2.5mm} \\
1904 & \hfill 122 & \hspace*{2.5mm} 1916 & \hfill 225 \hspace*{2.5mm} \\
1905 & \hfill 143 & \hspace*{2.5mm} 1917 & \hfill 227 \hspace*{2.5mm} \\
1906 & \hfill 152 & \hspace*{2.5mm} 1918 & \hfill 223 \hspace*{2.5mm} \\
1907 & \hfill 151 & \hspace*{2.5mm} 1919 & \hfill 218 \hspace*{2.5mm} \\
1908 & \hfill 126 & \hspace*{2.5mm} 1920 & \hfill 231 \hspace*{2.5mm} \\
1909 & \hfill 155 & \hspace*{2.5mm} 1921 & \hfill 179 \hspace*{2.5mm} \\
1910 & \hfill 159 & \hspace*{2.5mm} 1922 & \hfill 240 \hspace*{2.5mm} \\
\hline
\end{tabular}
}
\end{table}

%\pgfplotstabletypeset
%[col sep=&,row sep=\\,sci zerofill]
%{
%period & pro & period & pro\\
%1899 & 100 & 1911 & 153\\
%1900 & 101 & 1912 & 177\\
%1901 & 112 & 1913 & 184\\
%1902 & 122 & 1914 & 169\\
%1903 & 124 & 1915 & 189\\
%1904 & 122 & 1916 & 225\\
%1905 & 143 & 1917 & 227\\
%1906 & 152 & 1918 & 223\\
%1907 & 151 & 1919 & 218\\
%1908 & 126 & 1920 & 231\\
%1909 & 155 & 1921 & 179\\
%1910 & 159 & 1922 & 240\\
%}
\begin{figure}
\centering
\begin{tikzpicture}
\begin{axis}[
    width=0.8\textwidth,
    height=0.6\textwidth,
    legend pos = north west,
    x tick label style={/pgf/number format/.cd,
        scaled x ticks = false,
        set thousands separator={},
        fixed},
%    yticklabel={\pgfmathparse{50*(\tick)}\pgfmathprintnumber[fixed]{\pgfmathresult}},
    xlabel = Год,
    ylabel = Индекс,
    ytick={100,150,...,500},
    ymode=log,
    grid=major,
    legend entries={Основные средства,Объём производства,Численность занятых},
]

\addplot table {cobb_douglas_usa_cap.dat};
\addplot table {cobb_douglas_usa_pro.dat};
\addplot table {cobb_douglas_usa_lab.dat};

\end{axis}
\end{tikzpicture}
\caption{Развитие обрабатывающей промышленности, 1899--1922 гг., 1899\,г.\,=\,100}
\end{figure}

за 1899--1922\,гг., поскольку на момент, когда мы выполняли наше исследование, более поздний индекс, представленный д-ром Томасом, не был ещё опубликован\footnote{Подробное описание методологии и источников, использованных при построении индекса производства в обрабатывающей промышленности, см. в Э. Э. Дэй и У. М. Пёрсонс, <<Индекс физического объёма производства>>. \emph{Review of Economic Statistics}, II (1920\,г.), сс.\,309--37; 361--67.\parСм. также Эйда М. Мэтьюс, <<Физический объём производства САСШ за 1924\,г.>>,\parТам же., VII. (1925\,г.), с. 215.}.
\par
На рисунке I представлен график роста объёмов основного капитала, рабочей силы и производства в обрабатывающей промышленности в течение рассматриваемого периода, ось ординат которого имеет логарифмический масштаб.
\begin{table}
\centering
\footnotesize{
\caption{Соотношение количеств труда и капитала, затраченных на производство продукции обрабатывающей промышленности, 1899--1922\,гг. (1899\,г. = 100)}%
\label{tab5}%
\begin{tabular}{p{0.2\textwidth}|p{0.2\textwidth}||p{0.2\textwidth}|p{0.2\textwidth}}
\hline
Год & Отношение количества труда к объёмам капитала & Год & Отношение количества труда к объёмам капитала \\
\hline
1899 & \hfill 100 & \hspace*{2.5mm} 1911 & \hfill 67 \hspace*{2.5mm} \\
1900 & \hfill 98 & \hspace*{2.5mm} 1912 & \hfill 67 \hspace*{2.5mm} \\
1901 & \hfill 96 & \hspace*{2.5mm} 1913 & \hfill 65 \hspace*{2.5mm} \\
1902 & \hfill 97 & \hspace*{2.5mm} 1914 & \hfill 61 \hspace*{2.5mm} \\
1903 & \hfill 94 & \hspace*{2.5mm} 1915 & \hfill 58 \hspace*{2.5mm} \\
1904 & \hfill 84 & \hspace*{2.5mm} 1916 & \hfill 61 \hspace*{2.5mm} \\
1905 & \hfill 84 & \hspace*{2.5mm} 1917 & \hfill 59 \hspace*{2.5mm} \\
1906 & \hfill 82 & \hspace*{2.5mm} 1918 & \hfill 55 \hspace*{2.5mm} \\
1907 & \hfill 78 & \hspace*{2.5mm} 1919 & \hfill 50 \hspace*{2.5mm} \\
1908 & \hfill 65 & \hspace*{2.5mm} 1920 & \hfill 47 \hspace*{2.5mm} \\
1909 & \hfill 71 & \hspace*{2.5mm} 1921 & \hfill 35 \hspace*{2.5mm} \\
1910 & \hfill 69 & \hspace*{2.5mm} 1922 & \hfill 37 \hspace*{2.5mm} \\
\hline
\end{tabular}
}
\end{table}

\par
Необходимо отметить, что к 1922\,г. предложение капитала увеличилось более чем в четыре раза по сравнению с 1899\,г. при том, что предложение рабочей силы за тот же период увеличилось всего на 61\,\%. Отношение количества капитала к количеству рабочей силы действительно в 2,67 раз больше в 1922\,г. по сравнению с таким же показателем за 1899\,г. Прирост промышленного производства в течение рассматриваемого периода составил 140\,\%, или приблизительно 50\,\% на одного работника\footnote{Как показано д-р Томасом, наиболее существенный прирост производительности, который почти не отражён в приведённых выше данных, пришёлся на период, который берёт своё начало в 1921\,г.}.
\section{Отношение количества труда к объёму капитала}
Изменяющееся соотношение между количеством труда и объёмом капитала по сравнению с взятым в качестве базового 1899\,г. может быть получено посредством деления индекса относительного предложения труда на индекс относительного объёма основного капитала (\( L / C \)). Полученные в результате такой операции данные представлены в Таблице V. Таким образом мы получаем оценку изменяющихся соотношений двух указанных факторов по каждому году рассматриваемого периода.
\par
Необходимо отметить, что так как наш индекс труда показывает снижение численности занятых в периоды спада деловой активности в то время, как наш индекс капитала не учитывает недозагруженность капитала, полученное отношение труда к капиталу резко снижается в течение таких периодов с тенденцией к росту в течение последующих годов. Общий тренд, тем не менее, разумеется, направлен в сторону снижения в силу более высоких темпов прироста объёмов капитала.
\section{Теория производства}
Построенная на основе полученных индексов Производства, Труда и Капитала для периода 1899--1922\,гг. функция от Труда и Капитала
\[P'=1,01L^{\frac34}C^{\frac14}\]
обладает следующими свойствами:
\begin{enumerate}[{1)}]
\item Утверждение о том, что \(P'\) отражает фактическое Производство \(P\) равносильно конкретному выражению известной теории.
\item \(P'\) стремится к нулю, если либо \(L\) либо \(C\) стремятся к нулю.
\item \(P'\) является приближением \(P\) в рамках рассматриваемого периода.
\item Отклонения \(P'\) от \(P\) являются статистически значимыми по отдельности.
\item \(P'\) тесно коррелирует с \(P\), если данные включают долгосрочный тренд.
\item \(P'\) тесно коррелирует с \(P\), если данные не включают долгосрочный тренд.
\end{enumerate}
В смысле вышесказанного назовём \(P'\) \emph{нормой} \(P\) в рамках рассматриваемого периода, и перейдём к более подробному рассмотрению её свойств.
\begin{enumerate}[{(1)}]
\item Согласно теории, на которую мы ссылаемся (связанной с именами \linebreak Дж.\,Б.\,Кларка, Уикстида и др.), Производство, Труд и Капитал связаны таким образом, что если мы умножим как Труд, так и Капитал на параметр \(m\), то Производство увеличится в \(m\) раз, иначе говоря Производство является однородной функцией первой степени от Труда и Капитала. Аналогично \(P'\) принимается в качестве такой функции.
\item На функцию \(P'\) такого класса накладывается дополнительное теоретическое ограничение, заключающееся в том, что она должна стремиться к нулю, если либо \(L\) либо \(C\) стремятся к нулю.
\par
Возьмём конкретную форму функции\footnote{Что равносильно предположению о том, что предельная производительность труда пропорциональна объёму производства на единицу труда, а предельная производительность капитала пропорциональна объёму производства на единицу капитала. Названные свойства выводятся из <<избранной>> функции в следующем параграфе.} из множества всех функций с указанными выше свойствами (1) и (2), и рассмотрим следствия, связанные с таким выбором, оставляя за собой право по желанию сделать альтернативный выбор. Возьмём функцию
\[P'=bL^kC^{1-k}\]
и определим такие численные значения параметров \(b\) и \(k\), при которых \(P'\) является <<наилучшей>> аппроксимацией \(P\) согласно Методу наименьших квадратов. Таким образом на основе полученных индексов для рассматриваемого периода мы получаем норму
\[P'=1,01L^{\frac34}C^{\frac14}.\]
\par
\item При заданных индексах \(L\) и \(C\), функция \(P'\) может быть вычислена, а также её значения могут быть сопоставлены со значениями \(P\), отображёнными в Таблице VI и Рисунке II ниже по тексту:
\par
\begin{table}
%longtable
\centering
\footnotesize{
\caption{Соотношение (1) Физического объёма производства, исчисленного по формуле \(P'=1,01L^{\frac34}C^{\frac14}\) исходя из исторических данных \(L\) и \(C\) (Таблицы II и III), и (2) исторических данных Физического объёма производства (Таблица IV)}%
\label{tab6}%
\begin{tabular}{p{0.1\textwidth}|p{0.15\textwidth}|p{0.15\textwidth}|p{0.15\textwidth}|p{0.35\textwidth}}
\hline
Год & \(P'\) \par Физический объём производства исчисленный (1) & \(P\) \par Физический объём производства исторический (2) & Отклонение в процентах \par \(\frac{(2)-(1)}{(2)}\) & Экономические хроники\footnote{У.\,Л. Торп \emph{<<Экономические хроники>>}, с. 138 и далее.} \\
\hline
1899 & \hfill 101 \hspace*{2.5mm} & \hfill 100 \hspace*{2.5mm} & \hfill \(-\)1,0 \hspace*{2.5mm} & Подъём\\
1900 & \hfill 107 \hspace*{2.5mm} & \hfill 101 \hspace*{2.5mm} & \hfill \(-\)6,0 \hspace*{2.5mm} & Подъём; Кратковременная Рецессия\\
1901 & \hfill 112 \hspace*{2.5mm} & \hfill 112 \hspace*{2.5mm} & \hfill 0,0 \hspace*{2.5mm} & Подъём\\
1902 & \hfill 121 \hspace*{2.5mm} & \hfill 122 \hspace*{2.5mm} & \hfill 0,8 \hspace*{2.5mm} & Подъём\\
1903 & \hfill 126 \hspace*{2.5mm} & \hfill 124 \hspace*{2.5mm} & \hfill \(-\)1,6 \hspace*{2.5mm} & Подъём; рецессия\\
1904 & \hfill 123 \hspace*{2.5mm} & \hfill 122 \hspace*{2.5mm} & \hfill \(-\)0,8 \hspace*{2.5mm} & Умеренный спад; оживление\\
1905 & \hfill 133 \hspace*{2.5mm} & \hfill 143 \hspace*{2.5mm} & \hfill 7,0 \hspace*{2.5mm} & Подъём\\
1906 & \hfill 141 \hspace*{2.5mm} & \hfill 152 \hspace*{2.5mm} & \hfill 7,0 \hspace*{2.5mm} & Подъём\\
1907 & \hfill 148 \hspace*{2.5mm} & \hfill 151 \hspace*{2.5mm} & \hfill 2,0 \hspace*{2.5mm} & Подъём, паника, рецессия, Спад\\
1908 & \hfill 137 \hspace*{2.5mm} & \hfill 126 \hspace*{2.5mm} & \hfill \(-\)9,0 \hspace*{2.5mm} & Спад\\
1909 & \hfill 155 \hspace*{2.5mm} & \hfill 155 \hspace*{2.5mm} & \hfill 0,0 \hspace*{2.5mm} & Оживление, умеренный подъём\\
1910 & \hfill 160 \hspace*{2.5mm} & \hfill 159 \hspace*{2.5mm} & \hfill \(-\)0,6 \hspace*{2.5mm} & Рецессия\\
1911 & \hfill 163 \hspace*{2.5mm} & \hfill 153 \hspace*{2.5mm} & \hfill \(-\)6,5 \hspace*{2.5mm} & Умеренный спад\\
1912 & \hfill 170 \hspace*{2.5mm} & \hfill 177 \hspace*{2.5mm} & \hfill 4,0 \hspace*{2.5mm} & Оживление; подъём\\
1913 & \hfill 174 \hspace*{2.5mm} & \hfill 184 \hspace*{2.5mm} & \hfill 5,5 \hspace*{2.5mm} & Подъём; рецессия\\
1914 & \hfill 171 \hspace*{2.5mm} & \hfill 169 \hspace*{2.5mm} & \hfill \(-\)1,2 \hspace*{2.5mm} & Спад\\
1915 & \hfill 179 \hspace*{2.5mm} & \hfill 189 \hspace*{2.5mm} & \hfill 5,0 \hspace*{2.5mm} & Оживление; подъём\\
1916 & \hfill 209 \hspace*{2.5mm} & \hfill 225 \hspace*{2.5mm} & \hfill 7,2 \hspace*{2.5mm} & Подъём\\
1917 & \hfill 227 \hspace*{2.5mm} & \hfill 227 \hspace*{2.5mm} & \hfill 0,0 \hspace*{2.5mm} & Подъём; военные действия\\
1918 & \hfill 236 \hspace*{2.5mm} & \hfill 223 \hspace*{2.5mm} & \hfill \(-\)6,0 \hspace*{2.5mm} & Военные действия; рецессия\\
1919 & \hfill 233 \hspace*{2.5mm} & \hfill 218 \hspace*{2.5mm} & \hfill \(-\)7,0 \hspace*{2.5mm} & Оживление; подъём\\
1920 & \hfill 236 \hspace*{2.5mm} & \hfill 231 \hspace*{2.5mm} & \hfill \(-\)2,2 \hspace*{2.5mm} & Подъём; рецессия, спад\\
1921 & \hfill 194 \hspace*{2.5mm} & \hfill 179 \hspace*{2.5mm} & \hfill \(-\)8,4 \hspace*{2.5mm} & Спад\\
1922 & \hfill 209 \hspace*{2.5mm} & \hfill 240 \hspace*{2.5mm} & \hfill 13,0 \hspace*{2.5mm} & Оживление; подъём\\
\hline
\end{tabular}
}
\end{table}

\par
\begin{figure}
\centering
\begin{tikzpicture}
\begin{axis}[
    width=0.8\textwidth,
    height=0.6\textwidth,
    legend pos = south east
,
    x tick label style={/pgf/number format/.cd,
        scaled x ticks = false,
        set thousands separator={},
        fixed},
    xlabel = Год,
    ylabel = Индекс,
    ymode=log,
    grid=major,
    legend entries={Фактический объём производства,Расчётный объём производства},
]

\addplot table {cobb_douglas_usa_pro.dat};
\addplot table {cobb_douglas_usa_pro_com.dat};

\end{axis}
\end{tikzpicture}
\caption{График теоретического и фактического производства, 1899--1922\,гг., 1899\,г. = 100
}
\end{figure}
\par
Среднее процентное отклонение \(P'\) от \(P\) без учёта знака составляет 4,2\,\%. Более того, линия \(P\) лежит ближе к \(P'\), чем к линии своей трёхлетней скользящей средней, соответствующие среднеквадратические отклонения составляют 8,7\,\% и 11,7\,\% (6,6\,\% и 11,8\,\% -- \emph{Прим. перев.}) соответственно.
\par
\begin{table}
\centering
\footnotesize{
\caption{Отклонения от Трендов \(P\) и \(P'\) (Тренды представлены трёхлетними скользящими средними)}%
\label{tab7}%
\begin{tabular}{p{0.10\textwidth}|p{0.13\textwidth}|p{0.13\textwidth}||p{0.10\textwidth}|p{0.13\textwidth}|p{0.13\textwidth}}
\hline
Год & Отклонение \(P\) от Тренда \(P\) & Отклонение \(P'\) от Тренда \(P'\) & Год & Отклонение \(P\) от Тренда \(P\) & Отклонение \(P'\) от Тренда \(P'\) \\
\hline
1900 & \hfill \(-\)3 \hspace*{2.5mm} & \hfill 0 \hspace*{2.5mm} & 1911 & \hfill \(-\)10 \hspace*{2.5mm} & \hfill \(-\)1 \hspace*{2.5mm} \\
1901 & \hfill 0 \hspace*{2.5mm} & \hfill \(-\)1 \hspace*{2.5mm} & 1912 & \hfill 6 \hspace*{2.5mm} & \hfill 1 \hspace*{2.5mm} \\
1902 & \hfill 3 \hspace*{2.5mm} & \hfill 1 \hspace*{2.5mm} & 1913 & \hfill 7 \hspace*{2.5mm} & \hfill 2 \hspace*{2.5mm} \\
1903 & \hfill 1 \hspace*{2.5mm} & \hfill 3 \hspace*{2.5mm} & 1914 & \hfill \(-\)12 \hspace*{2.5mm} & \hfill \(-\)4 \hspace*{2.5mm} \\
1904 & \hfill \(-\)8 \hspace*{2.5mm} & \hfill \(-\)4 \hspace*{2.5mm} & 1915 & \hfill \(-\)5 \hspace*{2.5mm} & \hfill \(-\)7 \hspace*{2.5mm} \\
1905 & \hfill 4 \hspace*{2.5mm} & \hfill 1 \hspace*{2.5mm} & 1916 & \hfill 11 \hspace*{2.5mm} & \hfill 4 \hspace*{2.5mm} \\
1906 & \hfill 3 \hspace*{2.5mm} & \hfill 0 \hspace*{2.5mm} & 1917 & \hfill 2 \hspace*{2.5mm} & \hfill 3 \hspace*{2.5mm} \\
1907 & \hfill 8 \hspace*{2.5mm} & \hfill 6 \hspace*{2.5mm} & 1918 & \hfill 0 \hspace*{2.5mm} & \hfill 4 \hspace*{2.5mm} \\
1908 & \hfill \(-\)18 \hspace*{2.5mm} & \hfill \(-\)10 \hspace*{2.5mm} & 1919 & \hfill \(-\)6 \hspace*{2.5mm} & \hfill \(-\)2 \hspace*{2.5mm} \\
1909 & \hfill 7 \hspace*{2.5mm} & \hfill 4 \hspace*{2.5mm} & 1920 & \hfill 22 \hspace*{2.5mm} & \hfill 15 \hspace*{2.5mm} \\
1910 & \hfill 3 \hspace*{2.5mm} & \hfill 1 \hspace*{2.5mm} & 1921 & \hfill \(-\)38 \hspace*{2.5mm} & \hfill \(-\)19 \hspace*{2.5mm} \\
\hline
\end{tabular}
}
\end{table}

\par
\item Из представленных выше Таблицы VI и Рисунка II видно, что тренды \(P\) и \(P'\) (в нашем случае представленные трёхлетними скользящими средними) являются практически идентичными; фактически \(P'\) построен таким образом, что они должны быть таковыми. По рассмотрении Таблицы VII и Рисунка III можно также отметить, что вообще \(P'\) и \(P\) изменяются
\begin{figure}
\centering
\begin{tikzpicture}
\begin{axis}[
    width=0.8\textwidth,
    height=0.6\textwidth,
    legend pos = south west,
    x tick label style={/pgf/number format/.cd,
        scaled x ticks = false,
        set thousands separator={},
        fixed},
    xlabel = Год,
    ylabel = Процентное отклонение,
    grid=major,
    legend entries={Отклонения \(P\),Отклонения \(P'\)},
]

\addplot table {cobb_douglas_usa_sub_pro.dat};
\addplot table {cobb_douglas_usa_sub_pro_com.dat};

\end{axis}
\end{tikzpicture}
\caption{Процентное отклонение линий \(P\) и \(P'\) от линий своих трендов
Линия тренда = трёхлетняя скользящая средняя}
\end{figure}
соответственно фазам цикла деловой активности, с той лишь разницей, что амплитуда колебаний \(P'\) (относительно тренда) меньше амплитуды колебаний \(P\) в силу стабилизирующего влияния монотонно возрастающего \(C\).
\par
Если мы также принимаем в расчёт данные Экономических хроник У.\,Л. Торпа, становится очевидным (Таблица VI и Рисунок IV)\footnote{Следует отметить, что алгебраические знаки процентных отклонений в приведённом рисунке противоположны тем, что содержатся в соответствующей таблице.}, что в большинстве случаев результаты наших расчётов имеют свойство быть ниже в периоды подъёма деловой активности и быть выше -- в периоды спада деловой активности. Таким образом не только \(P'\) следует фазам экономического цикла, но и отклонения \(P'\) от \(P\) следуют таким фазам.
\par
\item и (6) Если данные включают или не включают долгосрочный тренд, то коэффициент корреляции \(P\) и \(P'\) составляет 0,97 и 0,94 соответственно.
\end{enumerate}
Вплоть до этого момента мы считали само собой разумеющимся, что <<нормальный>> объём производства \(P'\) был бы произведён при заданных объёмах труда и капитала при <<нормальных>> условиях. Такие нормальные условия являются фиктивными. Например, производительная сила <<среднего>> работника или доллара постоянной покупательной способности предполагается постоянной в течение исследуемого периода. При нормальных условиях управление не может быть более или менее эффективным в различные периоды времени. При нормальных условиях не было бы ни подъёмов, ни спадов деловой активности, ни военных действий и
\begin{figure}
\centering
\begin{tikzpicture}
\begin{axis}[
    width=0.8\textwidth,
    height=0.6\textwidth,
    legend pos = south west,
    x tick label style={/pgf/number format/.cd,
        scaled x ticks = false,
        set thousands separator={},
        fixed},
    xlabel = Год,
    ylabel = Процентное отклонение,
    grid=major,
]

\addplot table {cobb_douglas_usa_sub.dat};

\end{axis}
\end{tikzpicture}
\caption{Процентные отклонения Расчётного Производства от Фактического Производства, 1899--1922\,гг.}
\end{figure}
т. д. Различия между производством в нормальных условиях и производством в реальных условиях могут быть соотнесены, как указано в пункте (4) выше, с \emph{<<Экономическими хрониками>>} за период, год к году.
\par
На данном этапе существует возможность применить математический анализ к фиктивному производству \(P'\), однако не существует возможности применить такой анализ к фактическому производству \(P\) до тех пор, пока мы не сделаем в явном или неявном виде некоторые дальнейшие предположения. Сделаем следующие предположения, а также положим обоснования оных базироваться на том, что мы из них выводим.
\begin{enumerate}
\renewcommand{\theenumi}{\Asbuk{enumi}}
\renewcommand{\labelenumi}{(\theenumi)}
\item Физический объём производства пропорционален Объёму производства, полученному только в рамках обрабатывающей промышленности.
\item Любое отклонение \(P\) от \(P'\) можно представить через изменение значения множителя \(L^\frac{3}{4}C^\frac{1}{4}\) так, что всегда
\[P=bL^\frac{3}{4}C^\frac{1}{4},\]
причём значение \(b\) не зависит от \(L\) или \(C\).
\end{enumerate}

Два указанных выше предположения сделаны в соответствии с общим принципом не принимать в расчёт количественные эффекты любой природы, по которым у нас отсутствуют количественные данные. Множитель \(b\) таким образом становится обобщающим показателем для эффектов такой природы.
\par
По указанным выше предположениям на основании математического анализа отсюда непосредственно следует, что:
\par
\begin{enumerate}[{I.}]
\item Предельная производительность труда составляет 3/4 \(P/L\).
\item Предельная производительность капитала составляет 1/4 \(P/C\).
\item Производительность совокупного труда составляет 3/4 \(P\).
\item Производительность совокупного капитала составляет 1/4 \(P\).
Это означает, что три четверти достигнутого производства в течение рассматриваемого периода приходится на труд и одна четверть -- на капитал.
\item Эластичность производства относительно малого изменения труда составляет 3/4.
\item Эластичность производства относительно малого изменения капитала составляет 1/4.
\end{enumerate}
Это означает, что малое процентное изменение труда имеет в три раза более выраженный эффект, чем от такого же малого процентного изменения капитала.
\par
Указанные выше шесть теорем доказываются в следующем параграфе. Тем не менее следует помнить о том, что наши результаты сопровождаются точными численными значениями только для определённости, в то время как сами полученные численные значения жёстко привязаны к конкретному периоду и к конкретным индексам. При уточнении индексов или изменении рассматриваемого периода вполне возможно, что полученный постоянный параметр 3/4 сменится некими постоянными значениями 0,7 или 0,6, или вообще некоторой переменной. Возможно потребуется изменить даже \emph{форму} функции \(P'\).
\par
К тому же целью данной статьи является не изложение результатов, но демонстрация подхода к решению задачи. Выбор какой-либо конкретной Нормы Производства в качестве первого приближения отнюдь не гарантирует выбора наилучшей из возможных альтернатив. Вообще представляется, что преимущество выбора нормы заключается в том, что оно влечёт за собой логические следствия, которые могут быть сопоставлены с фактами по мере того, как мы такие факты получаем. Всё это позволяет нам рассуждать с большей точностью и формулировать выводы, которые обретают статус гипотез.
\section{Математический анализ}
Пусть задана функция
\[P=bL^kC^{1-k},\]
где значение параметра \(b\) не зависит от \(L\) или \(C\) и (для определённости) параметр \(k\) полагается постоянным и равным 3/4. Тогда шесть теорем предыдущего параграфа могут быть обоснованы следующими шестью уравнениями:
\begin{multicols}{2}
\begin{align}
\frac{\partial P}{\partial L} &= \frac34 \frac{P}{L}\\
\frac{\partial P}{\partial C} &= \frac14 \frac{P}{C}\\
\frac{L\,\partial P}{\partial L} &= \frac34 P
\end{align}

\begin{align}
\frac{C\,\partial P}{\partial C} &= \frac14 P\\
\frac{\partial \ln P}{\partial \ln L} &= \frac34\\
\frac{\partial \ln P}{\partial \ln C} &= \frac14
\end{align}
\end{multicols}

Если \(b\) принимается равным, например, 1,01, тогда
\begin{equation}
\label{eq:eq7}
\frac{\partial P}{\partial L}=1,\!01 \times \frac34 \times \left(\frac{L}{C}\right)^{-\frac14}\text{,} \quad b=1,\!01;
\end{equation}

\begin{equation}
\label{eq:eq8}
\frac{\partial P}{\partial C}=1,\!01 \times \frac14 \times \left(\frac{L}{C}\right)^\frac34\text{,} \quad b=1,\!01.
\end{equation}

Из уравнений (\ref{eq:eq7}) и (\ref{eq:eq8}) следует, что подобно тому, как для Производства существует аппроксимирующая его норма, существуют, в свою очередь, аппроксимирующие нормы для предельных производительностей труда и капитала, а именно: кривые \(y=1,01(L/C)^{-\frac14}\) и \(y=1,01(L/C)^{\frac34}\) соответственно.
\par
Указанные выше три нормы и соответствующие количественные измерения связаны таким образом, что если одно количественное измерение, например, производство, превысит собственную норму на 5\,\%, то каждое из двух других количественных измерений также превысит собственную норму на 5\,\%. Такое поведение обусловлено алгебраическим тождеством
\par
{Рисунок V}\footnote{На приведённом рисунке <<нормальные>> кривые построены без учёта множителя 1,01, а также индексы предельной производительности пропорционально уменьшены.}
\begin{tikzpicture}
\begin{axis}[
    axis lines = left,
    xlabel = $\frac{L}{C}$,
    ylabel = Индекс,
    grid=major,
]

\addplot [
    domain=10:100, 
    samples=100, 
    color=red,
]
{101*10^0.5*x^(-0.25)};
\addlegendentry{$\frac{3}{4}\frac{P}{L}$}

\addplot [
    domain=10:100, 
    samples=100, 
    color=blue,
    ]
    {1.01*10^0.5*x^0.75};
\addlegendentry{$\frac{1}{4}\frac{P}{C}$}

%    \addplot[
%        scatter/classes={labpro={blue}, cappro={red}},
%        scatter, mark=*, only marks, 
%        scatter src=explicit symbolic,
%%    ] table [meta=labcap] {cobb_douglas_usa_pro_ty.dat};
%    ] table {cobb_douglas_usa_pro_ty.dat};

\addplot+ [scatter,
only marks,]
table [x=labcap,y=labpro]
{cobb_douglas_usa_pro_ty.dat};

\addplot+ [scatter,
only marks,]
table [x=labcap,y=cappro]
{cobb_douglas_usa_pro_ty.dat};

\end{axis}
\end{tikzpicture}
\par
\[\frac{P}{L}:\left(\frac{L}{C}\right)^{-\frac14}=\frac{P}{C}:\left(\frac{L}{C}\right)^\frac34=P:L^{\frac34}C^{\frac14}=b:1.\]
\par
Теперь мы можем определить производные предельных производительностей и общих производительностей путём дифференцирования уравнений (1)--(4) с заменой постоянного параметра 3/4 на неопределённый параметр \(k\), при этом не теряя из виду, что \(k\) является постоянным, положительным и не превосходит 1.
%(9)
\begin{equation}
\frac{\partial}{\partial C}\!\left[\frac{\partial P}{\partial L}\right]=k(1-k)\frac{P}{LC},
\end{equation}
%(10)
\begin{equation}
\frac{\partial}{\partial L}\!\left[\frac{\partial P}{\partial C}\right]=k(1-k)\frac{P}{LC},
\end{equation}
\par
а, следовательно, и:
\par
В результате использования дополнительной условной единицы капитала происходит \emph{увеличение} производительности условной единицы труда. В результате использования дополнительной условной единицы труда происходит \emph{увеличение} производительности условной единицы капитала. Полученные темпы прироста (которые равны при заданных значениях \(L\) и \(C\)) выражены в правых частях уравнений (\ref{eq:eq7}) и (\ref{eq:eq8}).
%(11)
\begin{equation}
\label{eq:eq11}
\frac{\partial}{\partial L}\!\left[\frac{\partial P}{\partial L}\right]=k(1-k)\frac{P}{L^2},
\end{equation}
а, следовательно, и (в силу убывающей доходности):
\par
В результате использования дополнительной условной единицы труда происходит \emph{снижение} производительности условной единицы труда (поскольку \(k-1\) отрицательно) с темпом, заданным выражением в правой части уравнения (\ref{eq:eq11}).
\par
Аналогично:
%(12)
\begin{equation}
\label{eq:eq12}
\frac{\partial}{\partial C}\!\left[\frac{\partial P}{\partial C}\right]=k(1-k)\frac{P}{C^2},
\end{equation}
а, следовательно, и (в силу убывающей доходности):
\par
В результате использования дополнительной условной единицы капитала происходит \emph{снижение} производительности условной единицы капитала с темпом, заданным выражением в правой части уравнения (\ref{eq:eq12}).
%(13)
\par
\begin{equation}
\label{eq:eq13}
\frac{\partial}{\partial L}\!\left[L \frac{\partial P}{\partial L}\right]=k^2\frac{P}{L},
\end{equation}
а, следовательно, и:
\par
В результате использования дополнительной условной единицы труда происходит \emph{увеличение} производительности совокупного труда с темпом, заданным выражением в правой части уравнения (\ref{eq:eq13}).
%(14)
\begin{equation}
\label{eq:eq14}
\frac{\partial}{\partial C}\!\left[C \frac{\partial P}{\partial C}\right]=(1-k)^2\frac{P}{C},
\end{equation}
а, следовательно, и:
\par
В результате использования дополнительной условной единицы капитала происходит \emph{увеличение} производительности совокупного капитала с темпом, заданным выражением в правой части уравнения (\ref{eq:eq14}).
%(15)
\begin{equation}
\label{eq:eq15}
\frac{\partial}{\partial L}\!\left[C \frac{\partial P}{\partial C}\right]=k(1-k)\frac{P}{L},
\end{equation}
\par
а, следовательно, и:
\par
В результате использования дополнительной условной единицы труда происходит \emph{увеличение} производительности совокупного капитала с темпом, заданным выражением в правой части уравнения (\ref{eq:eq15}).
%(16)
\begin{equation}
\label{eq:eq16}
\frac{\partial}{\partial C}\!\left[L \frac{\partial P}{\partial L}\right]=k(1-k)\frac{P}{C},
\end{equation}
а, следовательно, и:
\par
В результате использования дополнительной условной единицы капитала происходит \emph{увеличение} производительности совокупного труда с темпом, заданным выражением в правой части уравнения (\ref{eq:eq16}).
\par
Наконец, при предположении о том, что \(k\) может принимать различные значения, \(P'\) становится функцией трёх переменных, и мы имеем новую группу теорем, как то: <<Если \(k\) увеличивается при при заданных \(L\) и \(C\), то \(P'\) увеличивается, если \(L / C\) больше 1, и \(P'\) снижается, если \(L / C\) меньше\,1>>.
\par
Так, например, если мы возьмём \(k\) меньше 3/4 (например, 2/3 в течение всего рассматриваемого периода), то рассчитанная таким образом кривая \(P'\) будет располагаться выше кривой \(P'\), рассчитанной при \(k=3/4\), всегда, когда \(L / C\) меньше 1, т. е. в течение практически всего рассматриваемого периода. Соотношение между \(P\) и новым \(P'=1,01L^{\frac23}C^{\frac13}\) приводится в следующей таблице.
\par
\begin{table}
\centering
\footnotesize{
\caption{Соотношение между \(P\) и \(P'=1,01L^{\frac23}C^{\frac13}\)}%
\label{tab8}%
\begin{tabular}{p{0.06\textwidth}|p{0.06\textwidth}|p{0.06\textwidth}|p{0.14\textwidth}||p{0.06\textwidth}|p{0.06\textwidth}|p{0.06\textwidth}|p{0.14\textwidth}}
\hline
Год & \(P\) & \(P'\) & \(\frac{P-P'}{P} \times 100\) & Год & \(P\) & \(P'\) & \(\frac{P-P'}{P} \times 100\) \\
\hline
1899 & \hfill 100 & \hfill 101 & \hfill \(-\)1 \hspace*{4mm} & 1911 & \hfill 153 & \hfill 166 & \hfill \(-\)8 \hspace*{4mm} \\
1900 & \hfill 101 & \hfill 106 & \hfill \(-\)5 \hspace*{4mm} & 1912 & \hfill 177 & \hfill 173 & \hfill 2 \hspace*{4mm} \\
1901 & \hfill 112 & \hfill 111 & \hfill 1 \hspace*{4mm} & 1913 & \hfill 184 & \hfill 178 & \hfill 3 \hspace*{4mm} \\
1902 & \hfill 122 & \hfill 119 & \hfill 2 \hspace*{4mm} & 1914 & \hfill 169 & \hfill 176 & \hfill \(-\)4 \hspace*{4mm} \\
1903 & \hfill 124 & \hfill 125 & \hfill \(-\)1 \hspace*{4mm} & 1915 & \hfill 189 & \hfill 185 & \hfill 2 \hspace*{4mm} \\
1904 & \hfill 122 & \hfill 123 & \hfill \(-\)1 \hspace*{4mm} & 1916 & \hfill 225 & \hfill 214 & \hfill 5 \hspace*{4mm} \\
1905 & \hfill 143 & \hfill 133 & \hfill 7 \hspace*{4mm} & 1917 & \hfill 227 & \hfill 234 & \hfill \(-\)3 \hspace*{4mm} \\
1906 & \hfill 152 & \hfill 142 & \hfill 7 \hspace*{4mm} & 1918 & \hfill 223 & \hfill 244 & \hfill \(-\)9 \hspace*{4mm} \\
1907 & \hfill 151 & \hfill 149 & \hfill 1 \hspace*{4mm} & 1919 & \hfill 218 & \hfill 243 & \hfill \(-\)11 \hspace*{4mm} \\
1908 & \hfill 126 & \hfill 139 & \hfill \(-\)10 \hspace*{4mm} & 1920 & \hfill 231 & \hfill 247 & \hfill \(-\)7 \hspace*{4mm} \\
1909 & \hfill 155 & \hfill 157 & \hfill \(-\)1 \hspace*{4mm} & 1921 & \hfill 179 & \hfill 208 & \hfill \(-\)16 \hspace*{4mm} \\
1910 & \hfill 159 & \hfill 163 & \hfill \(-\)3 \hspace*{4mm} & 1922 & \hfill 240 & \hfill 223 & \hfill 7 \hspace*{4mm} \\
\hline
\end{tabular}
}
\end{table}

\section{Какие существуют свидетельства в пользу справедливости приведённой выше теории?}
В пользу того, что уравнение \(P'=1,01L^{\frac34}C^{\frac14}\) достаточно точно описывает реальные процессы производства в обрабатывающей промышленности в течение рассматриваемого периода, свидетельствуют следующие аргументы:
\begin{enumerate}[{(1)}]
\item Близкое совпадение \(P\) и \(P'\), как показано в Таблице VI и на Рисунке II, при коэффициенте корреляции равном 0,97. Если принимать в рассмотрение трёхлетние скользящие средние \(P\) и \(P'\), то соответствие окажется ещё более точным: среднее процентное отклонение \(P'\) от \(P\) (без учёта знака) составит всего 2,6\,\% в год по сравнению c 4,2\,\%, полученными на основе данных по году к году. Суммарное отклонение трёхлетней скользящей средней \(P'\) от трёхлетней скользящей средней \(P\), в свою очередь, составляет всего \(-\)0,1\,\%.
\par
\item Степень близости, с которой теоретические кривые графиков вменённых производительностей условной единицы труда и условной единицы капитала, т. е. \(y=(L/C)^{-\frac14}\) и \(y=(L/C)^{\frac34}\) соответственно, образуют кривые наилучшего приближения для <<зарегистрированных>> значений производительностей условных единиц труда и капитала.
\par
\item Периодически заявлялось, что обнаруженная взаимосвязь между капиталом, трудом и промышленным производством является совершенно случайной и что одинаково справедливые результаты могли бы быть получены при сопоставлении относительного изменения поголовья свиней и крупного рогатого скота в штате Висконсин с физическим производством в обрабатывающей промышленности. Однако существуют логическая и экономическая связи между трудом, капиталом и производством, которые отсутствуют в приведённой попытке \emph{reductio ad absurdum}. Более того, факт того, что полученные отклонения \(P'\) и \(P\) от соответствующих собственных трёхлетних скользящих средних изменяются в достаточно точном соответствии друг с другом, как показано на Рисунке III, и что значение коэффициента корреляции для данных показателей составляет 0,94, свидетельствует о том, что исследуемая взаимосвязь объясняется не только тем, что указанные факторы имеют долгосрочную тенденцию к возрастанию своих значений.
\par
\item Отклонения \(P'\) от \(P\) практически в каждом случае находятся в полном соответствии с ожидаемыми значениями. Так, в периоды спада деловой активности, значительные объёмы капитала с необходимостью остаются незадействованными, однако наш индекс роста объёмов капитала этого не учитывает. Так же в силу применения режима неполного рабочего времени количество отработанных человеко-часов снижается более быстрыми темпами, чем численность занятых. Таким образом, следует ожидать, что исчисленный нами индекс \(P'\) будет больше фактического индекса \(P\). Далее отметим, что в 1908, 1911, 1914, 1920 и 1921\,гг. -- годы спада деловой активности -- \(P'\) был выше \(P\) на 9, 7, 1, 2 и 8\,\% соответственно, а также что в 1900, 1903, 1904 и 1910\,гг. -- годы замедления или небольшого спада деловой активности -- \(P'\) также был выше \(P\) на 6, 2, 1 и 1\,\% соответственно.
\end{enumerate}
С другой стороны, поскольку наши индексы количества труда и объёма капитала не учитывают соответственно сверхурочного рабочего времени и более высокой капиталоёмкости, свойственных периодам подъёма деловой активности, следует ожидать, что \(P'\) будет ниже \(P\) на протяжении такой фазы цикла деловой активности. Что и воспроизводится на практике. В 1905 и 1906\,гг. -- годы подъёма деловой активности -- \(P'\) был ниже \(P\) на 7\,\%, и в 1907\,г., первые три квартала которого были охарактеризованы повышенной деловой активностью, \(P'\) был ниже \(P\) на 2\,\%. В 1912 и 1913\,гг. -- годы подъёма деловой активности -- \(P'\), в свою очередь, был ниже \(P\) на 4 и 6\,\% соответственно, также в 1915 и 1916\,гг. \(P'\) был ниже \(P\) на 5 и 7\,\% соответственно. В 1922\,г. \(P'\) был ниже \(P\) не менее чем на 13\,\%.
\par
Единственными двумя годами, данные по которым не соответствуют нашим ожиданиям, являются 1918 и 1919\,гг. На указанные годы приходился подъём деловой активности, однако в них \(P'\)
не только не меньше, но даже выше \(P\) на 6 и 7\,\% соответственно. Впрочем, такие результаты могли быть вызваны снижением средней продолжительности стандартной рабочей недели и снижением производительности труда, что фактически привело к снижению производительности условной единицы труда относительно её обычного уровня.
\section{Аппроксимирует ли процесс распределения наблюдаемые законы производства?}
Мы предприняли попытку проверить изложенную выше теорию на предмет того, подчиняются ли в какой-либо мере процессы распределения обнаруженным нами, по нашему мнению, законам производства. На основе методологии, которая ранее была описана (в параграфах 6 и 7), получены следующие относительные конечные физические производительности труда по каждому году исследуемого периода в ценах 1899\,г.:
\par

\begin{table}[!h]
\centering
\footnotesize{
\begin{tabular}{r@{~\ldots~}rr@{~\ldots~}lr@{~\ldots~}l}
1899 & 100 & 1907 & 110 & 1915 & 123 \\
1900 & 96 & 1908 & 104 & 1916 & 123 \\
1901 & 102 & 1909 & 110 & 1917 & 116 \\
1902 & 103 & 1910 & 110 & 1918 & 111 \\
1903 & 101 & 1911 & 105 & 1919 & 113 \\
1904 & 105 & 1912 & 116 & 1920 & 119 \\
1905 & 114 & 1913 & 119 & 1921 & 121 \\
1906 & 115 & 1914 & 113 & 1922 & 149
\end{tabular}
}
\end{table}


\par
Полученные относительные физические производительности затем были умножены на относительную меновую стоимость составной единицы продукции обрабатывающей промышленности, и тем самым в дополнение к ранее полученному относительному физическому продукту был получен относительный \emph{продукт в стоимостном выражении} в расчёте на работника по каждому году исследуемого периода. После чего становится возможным сопоставить изменение такого продукта в стоимостном выражении, созданного конечным трудом, с относительным изменением реальной заработной платы работников в течение исследуемого периода с тем, чтобы определить степень согласованности между данными показателями.
\par
Тем не менее прежде чем приступить к такому сопоставлению, следует описать, как был получен ряд коэффициентов сопоставления единицы выбранных групп \emph{продукции} обрабатывающей промышленности с единицей продукции обрабатывающей промышленности в целом. Такой ряд был получен путём умножения индекса физического объёма производства на отношение уровня цен продукции обрабатывающей промышленности к относительному общему уровню цен.

\[\parbox[c]{3.5cm}{Индекс физического объёма производства} \times \frac{\parbox[t]{3.5cm}{Уровень цен продукции обрабатывающей промышленности}}{\parbox[t]{3.5cm}{Общий уровень цен}}\]
Ряд такого отношения цен продукции обрабатывающей промышленности к общему уровню цен был рассчитан на основе статистических данных по оптовым ценам, полученным Федеральным бюро статистики труда, и представлен в таблице, следующей по тексту далее.
\par
Данный ряд показывает, что покупательная способность исчисленной в ценах 1899\,г. единицы продукции обрабатывающей промышленности убывала в течение десяти последующих лет и при этом достигла своего наименьшего значения равного 85 в 1910\,г. Меновая стоимость такой единицы в течение последующих лет была несколько выше и даже немного возросла в 1922\,г. до значения, которое тем не менее на 10\,\% было ниже уровня 1899\,г. Что, в свою очередь, снизило относительное значение совокупного
продукта в стоимостном выражении с 240 до 217; последнее, наряду со значением 1920\,г., было наибольшим значением в пределах исследуемого периода.
\par
\begin{table}
\centering
\footnotesize{
\caption{Относительный продукт обрабатывающей промышленности в стоимостном выражении и совокупный продукт обрабатывающей промышленности в стоимостном выражении, 1899--1922\,гг. (1899\,г. = 100)}%
\label{tab9}%
\begin{tabular}{p{0.05\textwidth}|p{0.175\textwidth}|p{0.175\textwidth}|p{0.175\textwidth}|p{0.175\textwidth}}
\hline
Год & Индекс цен по всем выбранным группам продукции обрабатывающей промышленности & Индекс общего уровня цен продукции обрабатывающей промышленности% (<L3>)
& Коэффициент сопоставления единицы выбранных групп продукции обрабатывающей промышленности с единицей продукции обрабатывающей промышленности & Совокупный продукт в стоимостном выражении (физический продукт умножить на столбец 3)
%<1>
\\
& \centering 1 & \centering 2 & \centering 3$^{\ast}$ & \\
\hline
1899 & \hfill 100 \hspace*{6mm} & \hfill 100 \hspace*{6mm} & \hfill 100 \hspace*{6mm} & \hfill 100 \hspace*{6mm} \\
1900 & \hfill 105 \hspace*{6mm} & \hfill 108 \hspace*{6mm} & \hfill 98 \hspace*{6mm} & \hfill 99 \hspace*{6mm} \\
1901 & \hfill 101 \hspace*{6mm} & \hfill 106 \hspace*{6mm} & \hfill 96 \hspace*{6mm} & \hfill 107 \hspace*{6mm} \\
1902 & \hfill 103 \hspace*{6mm} & \hfill 113 \hspace*{6mm} & \hfill 91 \hspace*{6mm} & \hfill 111 \hspace*{6mm} \\
1903 & \hfill 104 \hspace*{6mm} & \hfill 114 \hspace*{6mm} & \hfill 91 \hspace*{6mm} & \hfill 113 \hspace*{6mm} \\
1904 & \hfill 103 \hspace*{6mm} & \hfill 114 \hspace*{6mm} & \hfill 90 \hspace*{6mm} & \hfill 109 \hspace*{6mm} \\
1905 & \hfill 106 \hspace*{6mm} & \hfill 115 \hspace*{6mm} & \hfill 92 \hspace*{6mm} & \hfill 132 \hspace*{6mm} \\
1906 & \hfill 112 \hspace*{6mm} & \hfill 118 \hspace*{6mm} & \hfill 95 \hspace*{6mm} & \hfill 144 \hspace*{6mm} \\
1907 & \hfill 119 \hspace*{6mm} & \hfill 125 \hspace*{6mm} & \hfill 95 \hspace*{6mm} & \hfill 144 \hspace*{6mm} \\
1908 & \hfill 110 \hspace*{6mm} & \hfill 120 \hspace*{6mm} & \hfill 91 \hspace*{6mm} & \hfill 115 \hspace*{6mm} \\
1909 & \hfill 112 \hspace*{6mm} & \hfill 129 \hspace*{6mm} & \hfill 87 \hspace*{6mm} & \hfill 134 \hspace*{6mm} \\
1910 & \hfill 115 \hspace*{6mm} & \hfill 135 \hspace*{6mm} & \hfill 85 \hspace*{6mm} & \hfill 136 \hspace*{6mm} \\
1911 & \hfill 111 \hspace*{6mm} & \hfill 124 \hspace*{6mm} & \hfill 90 \hspace*{6mm} & \hfill 137 \hspace*{6mm} \\
1912 & \hfill 116 \hspace*{6mm} & \hfill 132 \hspace*{6mm} & \hfill 88 \hspace*{6mm} & \hfill 156 \hspace*{6mm} \\
1913 & \hfill 117 \hspace*{6mm} & \hfill 134 \hspace*{6mm} & \hfill 88 \hspace*{6mm} & \hfill 162 \hspace*{6mm} \\
1914 & \hfill 113 \hspace*{6mm} & \hfill 131 \hspace*{6mm} & \hfill 86 \hspace*{6mm} & \hfill 146 \hspace*{6mm} \\
1915 & \hfill 119 \hspace*{6mm} & \hfill 135 \hspace*{6mm} & \hfill 88 \hspace*{6mm} & \hfill 167 \hspace*{6mm} \\
1916 & \hfill 156 \hspace*{6mm} & \hfill 169 \hspace*{6mm} & \hfill 92 \hspace*{6mm} & \hfill 207 \hspace*{6mm} \\
1917 & \hfill 210 \hspace*{6mm} & \hfill 237 \hspace*{6mm} & \hfill 89 \hspace*{6mm} & \hfill 201 \hspace*{6mm} \\
1918 & \hfill 226 \hspace*{6mm} & \hfill 259 \hspace*{6mm} & \hfill 87 \hspace*{6mm} & \hfill 194 \hspace*{6mm} \\
1919 & \hfill 242 \hspace*{6mm} & \hfill 276 \hspace*{6mm} & \hfill 89 \hspace*{6mm} & \hfill 191 \hspace*{6mm} \\
1920 & \hfill 284 \hspace*{6mm} & \hfill 302 \hspace*{6mm} & \hfill 94 \hspace*{6mm} & \hfill 217 \hspace*{6mm} \\
1921 & \hfill 186 \hspace*{6mm} & \hfill 196 \hspace*{6mm} & \hfill 95 \hspace*{6mm} & \hfill 170 \hspace*{6mm} \\
1922 & \hfill 179 \hspace*{6mm} & \hfill 199 \hspace*{6mm} & \hfill 90 \hspace*{6mm} & \hfill 217 \hspace*{6mm} \\
\hline
\end{tabular}
}
\par
\footnotesize{{}$^{\ast}$ Значения колонки (3) равны частным от деления значений колонки (1) на (2).}
\end{table}

Относительные физические производительности конечных условных единиц труда за последующие годы затем были умножены на относительный коэффициент сопоставления единицы физического производства соответствующего года, и тем самым был получен динамический ряд относительной \emph{стоимостной} производительности конечных условных единиц труда. Данный ряд представлен в таблице X, в рамках которой средний уровень 1899--1908\,гг. был принят за 100\footnote{Статистические данные по ценам были почерпнуты из Статистического бюллетеня 390 Федерального бюро статистики труда САСШ. В целях определения уровня цен продукции всей обрабатывающей промышленности были использованы следующие группы: (1) пищевые продукты, (2) ткани и одежда, (3) химико-фармацевтические препараты, (4) металлы и металлические изделия, (5) строительные материалы, (6) предметы домашнего обихода, (7) прочая продукция обрабатывающей промышленности, включая, например, кожаные изделия, целлюлозно-бумажные изделия, мыло и табачные изделия.}.
\par
Полученный ряд затем был сопоставлен с индексом реальной заработной платы в обрабатывающей промышленности, построенным одним из авторов данной работы\footnote{Пол Х. Дуглас, <<Изменение реальной заработной платы в течение последних лет и его экономическая значимость>>. Приложение, \emph{American Economic Review}, март 1926\,г., с. 33.}. Дабы напрасно не допускать, что значения двух рядов идеально совпадали в 1899\,г., средний уровень
1899--1908\,гг. был принят в качестве базового. Сравнительная таблица приведена на странице 164.
\par
\begin{table}
\centering
\footnotesize{
\caption{Относительная стоимостная производительность в расчёте на условную единицу труда, 1899--1922\,гг.}%
\label{tab10}%
\begin{tabular}{p{0.15\textwidth}|p{0.25\textwidth}||p{0.15\textwidth}|p{0.25\textwidth}}
\hline
Год & Относительная стоимостная производительность в расчёте на условную единицу труда & Год & Относительная стоимостная производительность в расчёте на условную единицу труда \\
\hline
1899 & \hfill 101 \hspace*{5mm} & 1911 & \hfill 96 \hspace*{5mm} \\
1900 & \hfill 95 \hspace*{5mm} & 1912 & \hfill 103 \hspace*{5mm} \\
1901 & \hfill 99 \hspace*{5mm} & 1913 & \hfill 106 \hspace*{5mm} \\
1902 & \hfill 95 \hspace*{5mm} & 1914 & \hfill 98 \hspace*{5mm} \\
1903 & \hfill 93 \hspace*{5mm} & 1915 & \hfill 110 \hspace*{5mm} \\
1904 & \hfill 96 \hspace*{5mm} & 1916 & \hfill 115 \hspace*{5mm} \\
1905 & \hfill 106 \hspace*{5mm} & 1917 & \hfill 104 \hspace*{5mm} \\
1906 & \hfill 111 \hspace*{5mm} & 1918 & \hfill 98 \hspace*{5mm} \\
1907 & \hfill 105 \hspace*{5mm} & 1919 & \hfill 102 \hspace*{5mm} \\
1908 & \hfill 96 \hspace*{5mm} & 1920 & \hfill 114 \hspace*{5mm} \\
1909 & \hfill 97 \hspace*{5mm} & 1921 & \hfill 117 \hspace*{5mm} \\
1910 & \hfill 95 \hspace*{5mm} & 1922 & \hfill 136 \hspace*{5mm} \\
\hline
\end{tabular}
}
\end{table}

\par
Коэффициент корреляции между обозначенными рядом и индексом составляет 0,69 со средней ошибкой в размере ±0,072, а коэффициент корреляции между их семилетними скользящими средними составляет 0,89 со средней ошибкой в размере ±0,03. Впрочем, краткосрочные колебания обозначенных ряда и индекса происходят практически независимо, поскольку коэффициент корреляции между отклонениями данных ряда и индекса от собственных долгосрочных трендов составляет всего лишь 0,12.
\par
Так или иначе установленная мера согласованности позволяет в достаточной степени подтвердить разрабатываемый закон производства, а также показать, что на достаточно продолжительном временном отрезке процессы распределения в значительной мере находятся в зависимости от процессов производства.
\par
Работы Национального бюро экономических исследований по динамике доли промышленного производства, приходящейся на труд, в течение десятилетия 1909--1918\,гг. позволяют провести очередное примечательное сопоставление. В соответствии с данными работами, заработная плата и жалованье в среднем составляли 74\,\% от совокупной добавленной стоимости в обрабатывающей промышленности в течение указанного периода\footnote{Национальное бюро экономических исследований, \emph{<<Доход в САСШ>>}, Том 2, с. 98. Полученные доли по годам были следующими:
\begin{tabular}{lrlrlrlrlr}
\centering
1909 & 72,2 & 1911 & 76,4 & 1913 & 74,5 & 1915 & 75,4 & 1917 & 71,0\\
1910 & 71,6 & 1912 & 74,5 & 1914 & 77,8 & 1916 & 68,7 & 1918 & 78,1
\end{tabular}}.
На основе выведенной выше формулы мы определили, что если мы относим 75\,\% продукта на труд, мы получаем близкое совпадение с фактическим обычным ходом производственного процесса.
\par
Таким образом, очевидно существует явная тенденция к тому, что распределение подчиняется законам вменённой производительности. Дабы никого не подталкивать к сколь бы то ни было поспешному умозаключению о том, что сказанное придаёт нравственное оправдание существующему социально-экономическому порядку, следует отметить, что
даже если бы существовало точное статистическое соответствие, это не внесло бы ясности, например, в вопрос того, следует ли капиталу находиться в частной собственности в той мере, в какой это присуще нашему обществу. Вместе с тем из того, что капитал может быть <<производительным>>, не следует то, что таковым всегда является капиталист. Капитал был бы по-прежнему <<производительным>>, даже если бы изменилась его форма собственности. Из этого также не следует, что цели, на которые капиталисты направляют получаемый доход, всегда являются оптимальными с точки зрения общественной пользы. Таким образом, разработанная нами теория производства вполне может быть увязана сторонниками социализма, коммунизма или индивидуализма с их собственными социальными воззрениями.
\par
\begin{table}
\centering
\footnotesize{
\caption{Относительное изменение вменённого продукта в стоимостном выражении в расчёте на работника и реальная заработная плата в обрабатывающей промышленности (1899--1922\,гг.) (1899--1908\,гг. = 100)}%
\label{tab11}%
\begin{tabular}{p{0.075\textwidth}|p{0.14\textwidth}|p{0.14\textwidth}|p{0.14\textwidth}|p{0.3\textwidth}}
\hline
& \centering (1) & \centering (2) & \centering (3) & \\
& Продукт условной единицы труда в стоимостном выражении (среднее за 1899--1908\,гг. = 100) & Реальная заработная плата (среднее за 1899--1908\,гг. = 100) & Процентное отклонение (2) от (1) \par \(\frac{(2)-(1)}{(2)}\) & Экономические хроники (кратко) \\
\hline
1899 & \hfill 101 \hspace*{6mm} & \hfill 99 \hspace*{6mm} & \hfill \(-\)2 \hspace*{6mm} & \\
1900 & \hfill 95 \hspace*{6mm} & \hfill 98 \hspace*{6mm} & \hfill 3 \hspace*{6mm} & Краткосрочная рецессия \\
1901 & \hfill 99 \hspace*{6mm} & \hfill 101 \hspace*{6mm} & \hfill 2 \hspace*{6mm} & \\
1902 & \hfill 95 \hspace*{6mm} & \hfill 102 \hspace*{6mm} & \hfill 7 \hspace*{6mm} & \\
1903 & \hfill 93 \hspace*{6mm} & \hfill 100 \hspace*{6mm} & \hfill 8 \hspace*{6mm} & \\
1904 & \hfill 96 \hspace*{6mm} & \hfill 99 \hspace*{6mm} & \hfill 3 \hspace*{6mm} & Умеренный спад деловой активности \\
1905 & \hfill 106 \hspace*{6mm} & \hfill 103 \hspace*{6mm} & \hfill \(-\)3 \hspace*{6mm} & \\
1906 & \hfill 111 \hspace*{6mm} & \hfill 101 \hspace*{6mm} & \hfill \(-\)9 \hspace*{6mm} & \\
1907 & \hfill 105 \hspace*{6mm} & \hfill 99 \hspace*{6mm} & \hfill \(-\)6 \hspace*{6mm} & \\
1908 & \hfill 96 \hspace*{6mm} & \hfill 94 \hspace*{6mm} & \hfill \(-\)2 \hspace*{6mm} & Спад деловой активности \\
1909 & \hfill 97 \hspace*{6mm} & \hfill 102 \hspace*{6mm} & \hfill 5 \hspace*{6mm} & \\
1910 & \hfill 95 \hspace*{6mm} & \hfill 104 \hspace*{6mm} & \hfill 9 \hspace*{6mm} & \\
1911 & \hfill 96 \hspace*{6mm} & \hfill 97 \hspace*{6mm} & \hfill 1 \hspace*{6mm} & Умеренный спад деловой активности \\
1912 & \hfill 103 \hspace*{6mm} & \hfill 99 \hspace*{6mm} & \hfill \(-\)4 \hspace*{6mm} & \\
1913 & \hfill 106 \hspace*{6mm} & \hfill 100 \hspace*{6mm} & \hfill \(-\)6 \hspace*{6mm} & \\
1914 & \hfill 98 \hspace*{6mm} & \hfill 99 \hspace*{6mm} & \hfill 1 \hspace*{6mm} & Спад деловой активности \\
1915 & \hfill 110 \hspace*{6mm} & \hfill 99 \hspace*{6mm} & \hfill \(-\)10 \hspace*{6mm} & \\
1916 & \hfill 115 \hspace*{6mm} & \hfill 104 \hspace*{6mm} & \hfill \(-\)10 \hspace*{6mm} & \\
1917 & \hfill 104 \hspace*{6mm} & \hfill 103 \hspace*{6mm} & \hfill \(-\)1 \hspace*{6mm} & Военные действия \\
1918 & \hfill 98 \hspace*{6mm} & \hfill 107 \hspace*{6mm} & \hfill 9 \hspace*{6mm} & Военные действия \\
1919 & \hfill 102 \hspace*{6mm} & \hfill 111 \hspace*{6mm} & \hfill 9 \hspace*{6mm} & \\
1920 & \hfill 114 \hspace*{6mm} & \hfill 114 \hspace*{6mm} & \hfill 0 \hspace*{6mm} & \\
1921 & \hfill 117 \hspace*{6mm} & \hfill 115 \hspace*{6mm} & \hfill \(-\)2 \hspace*{6mm} & Спад деловой активности \\
1922 & \hfill 136 \hspace*{6mm} & \hfill 119 \hspace*{6mm} & \hfill \(-\)12 \hspace*{6mm} & \\
\hline
\end{tabular}

\par

\begin{enumerate}[{(1)}]
\item Сумма отклонений без учёта знака \(= 125\text{\,\%}\);
\item Среднее отклонение \(= \frac{125\text{\,\%}}{24} = 5,\!2\text{\,\%}\);
\item Сумма отклонений с учётом знака \(= -68\text{\,\%}+57\text{\,\%} = -11\text{\,\%}\);
\item Среднее отклонение с учётом знака \(= \frac{-11\text{\,\%}}{24} = -0,\!5\text{\,\%}\).
\end{enumerate}
}
\end{table}
\section{Программа дальнейших исследований}
В заключение следует прояснить, что мы не утверждаем, что мы в самом деле вывели закон производства,
напротив -- мы всего лишь утверждаем, что мы аппроксимировали его и предложили возможный подход к решению задачи. Дальнейшему продвижению в этом вопросе будет способствовать разработка более точных рядов данных, применение различных математических методов и анализ иных наборов данных.
\par
Стало быть, мы можем надеяться, что появятся: (1) усовершенствованный индекс предложения труда, который будет более точно аппроксимировать относительное фактическое количество часов, отработанных не только производственными, но также и административными работниками; (2) улучшенный индекс роста объёмов капитала; (3) усовершенствованный индекс производства, который будет основан на результатах замечательной работы д-ра Томаса; (4) более точный индекс относительной меновой стоимости единицы продукции обрабатывающей промышленности.
\par
При анализе таких данных, нам следует (1) быть готовыми вывести формулы, которые не обязательно будут основываться на постоянных относительных <<вкладах>> каждого из факторов в совокупном продукте, но которые будут делать поправку на колебания от года к году и (2) будут исключать влияние фактора времени на исследуемый процесс настолько, насколько это возможно.
\par
Мы разработали нашу теорию на основании исследования изменений труда, капитала, производства, стоимости и заработной платы в отраслях обрабатывающей промышленности данной страны в целом. Возможно также применить тот же, либо же усовершенствованный метод анализа к прочим отраслям промышленности САСШ, как то: транспорт, добывающая промышленность, коммунальные услуги и т. д. -- и к подобным данным других стран. По завершении такого комплексного исследования в наших руках окажется ценнейший статистический материал по углам наклона кривых вменённой производительности по целому ряду отраслей промышленности, на основе которого мы сможем построить комбинированные кривые для стран в целом, а уже отсюда -- построить интересные международные сопоставления.
\par
Наконец, нам следует в конечном счёте ожидать включения в наши уравнения природных ресурсов в качестве третьего фактора производства и понять, как это преобразовывает наши умозаключения и насколько это раскрывает действие закона ренты.
\par
Для решения всех этих задач потребуется много времени, однако мы утверждаем, что их решение неизбежно, если необходимо обнаружить и количественно оценить строгие зависимости, которые, по всей вероятности, скрываются за экономическими явлениями.
\end{document}