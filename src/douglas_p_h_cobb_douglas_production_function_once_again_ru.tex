\documentclass{article}
\usepackage[utf8]{inputenc}
\usepackage[english,main=russian]{babel}
\usepackage[T2A]{fontenc}
%\usepackage{indentfirst}
\usepackage{amsmath,amsfonts,amssymb}
\usepackage{enumerate}
\usepackage{nccfoots}
\usepackage{array}
\usepackage{booktabs}
\usepackage{multirow}
%\usepackage[symbol*]{footmisc}
\newcolumntype{C}[1]{>{\centering\arraybackslash}p{#1}}


\title{Ещё раз о производственной функции Кобба--Дугласа: история, апробация и некоторые новые эмпирические данные}

\author{Пол\,Х.~Дуглас\\
\emph{Вашингтон, округ Колумбия}\thanks{Исследования производственной функции проводились большим количеством исследователей, которые заслуживают быть указанными в качестве соавторов. Среди них особого упоминания заслуживает Грейс Ганн, т.\,к. она на протяжении 40 лет участвовала не менее чем в трёх отдельных этапах исследования. Мэри Хук содействовала на поздних этапах исследования, а также Стенли Хоровиц и Кристофер Джен оказывали поддержку с финальной подготовкой данной статьи. Я обязан им всем и многим другим за их помощь. Среди других, кто оказывал поддержку в ходе работы, следует назвать Марджори Хэндсейкер, Мартина Бронфенбреннера и Патришу Дейли.}
}
\makeatletter
\def\thanks#1{\protected@xdef\@thanks{\@thanks
 \protect\footnotetext{#1}}}
\makeatother

\newenvironment{poliabstract}[1]
  {\renewcommand{\abstractname}{#1}\begin{abstract}}
  {\end{abstract}}

\begin{document}


\selectlanguage{russian}
\renewcommand{\abstractname}{От редактора}
\maketitle

%[Page 903]
\selectlanguage{russian}
\begin{poliabstract}{От редактора} 
Мы рады были получить к публикации и представить теперь читателям \emph{Журнала} данный обзор ранних эмпирических исследований функции Кобба--Дугласа, который был составлен её соавтором и экономистом"=исследователем, новатором, Полом\,Х.\,Дугласом.
%{1: [Journal of Political Economy, 1976, Т. 84. № 5]/® 1976 Чикагский университет, Все права защищены.}
\end{poliabstract}

\selectlanguage{russian}
\begin{poliabstract}{Аннотация}
У исследования производственной функции долгая история. С момента публикации первой работы в 1928\,г. многие исследования имели свойство подтверждать гипотезу о том, что производственные процессы хорошо описываются линейной однородной функцией с единичной эластичностью замещения факторов производства. В данной работе представлены результаты новых исследований, охватывающих наблюдения по отраслям австралийской обрабатывающей промышленности за 7 лет из периода 1950--1960-х\,гг. Во всех семи случаях результаты практически точно согласуются с постоянной отдачей от масштаба, а показатель труда находится вблизи значения 0,60. Согласованность соответствующих полученных показателей и долей произведённого продукта, приходящихся на труд, подтверждает справедливость принципа конкурентного распределения доходов.
\end{poliabstract}

Данная работа представляет собой попытку продолжить и расширить ранние исследования производственной функции, которые впервые начались без малого полвека назад. Исходным пунктом таких исследований стал 1927\,г., когда я рассчитал ежегодные значения индекса общей численности производственных работников \((L)\), занятых в обрабатывающей промышленности США, за период с 1899 по 1922\,гг., рассчитал подобным образом индекс основного капитала \((C)\), представил графически данные индексы в логарифмическом масштабе и затем сопоставил их с индексом физического объёма~%[Page 904]
производства \((P)\) обрабатывающей промышленности\footnote{Данный индекс был разработан Э. Э. Дэем.}. Я обнаружил, что <<кривая>> индекса производства в среднем расположена на расстоянии приблизительно в четверть относительного расстояния между <<кривой>> индекса труда, который показал наименьший рост за указанный период (до 162), и <<кривой>> индекса капитала, который показал наибольший рост за указанный период (до 431). Значения 1899\,г. были приняты за 100. В то время я временно читал лекции в Амхерстском колледже и обратился к помощи моего друга и коллеги, математика Чарльза\,У.\,Кобба. По предложению последнего была принята формула \(P=bL^kC^{1-k}\) -- выражение, которое ранее также использовали Уикстид и Викселль. Данное выражение в соответствии с тождеством Эйлера описывает однородную функцию первой степени. Определив по методу наименьших квадратов, что значение \(k\) равно 0,75, мы обнаружили, что расчётные значения \(P\) достаточно точно аппроксимируют фактические значения в течение рассмотренного 23-летнего периода при том, что наблюдаемые расхождения объясняются в основном колебаниями деловой активности \cite[сс.\,139--65]{Douglas:1}. В рамках одной из работ Национального бюро экономических исследований было обнаружено, что средняя доля промышленного производства, приходившаяся на труд, в течение десятилетия 1909--1918\,гг., составила 74,1\,\%\footnote{См. \cite[с.\,98]{Mitchell:1}.}, т. е. указанное значение практически точно соответствовало значению показателя производственной функции. Таким образом, я убедился в том, что теория предельной производительности заработной платы получила существенное подтверждение.

Последующие исследования Кобба по Массачусетсу, 1890--1928\,гг., \cite{Cobb:1} и Директора по Новому Южному Уэльсу, 1901--1927\,гг., (не опубликовано) показали идентичное значение для Массачусетса \((k = 0{,}743)\) и близкое значение для Нового Южного Уэльса \((k = 0{,}65)\).

Мы провели ещё одно исследование временных рядов. Совместно с г\mbox{-}жой Марджори Хэндсейкер в 1936\,г. мы обработали ряды \(P\), \(L\) и \(C\) за 1907--1929\,гг. по австралийскому штату Виктория и нашли значение \(k\), равное 0,71. Соотношение труда к производству \(W/P\) за указанный период составило 0,61 \cite{Douglas:6}.

\section*{Исследования 1937--1947\,гг.}

В 1937\,г. с выходом статьи Дэвида Дюранда \cite{Durand:1} и со вступлением Грейс Ганн в коллектив исследовательская работа претерпела несколько изменений. Во-первых, по предложению Дюранда формулу \(P=bL^kC^{1-k}\) заменили формулой \(P=bL^kC^j\), которая предполагает независимость показателя степени \(C\) вместо его комплементарности по отношению к показателю степени \(L\) в составе линейного однородного уравнения. При независимом параметре \(j\) более не требуется, чтобы производственная функция была однородной функцией первой степени, но тем не менее она может принимать такую форму при соответствующем наборе рассматриваемых реальных данных. В случае, когда \(k\:+\:j = 1{,}0\), экономическая система характеризуется постоянной отдачей от масштаба. В случае, когда \(k\:+\:j > 1{,}0\), одновременный прирост как \(L\), так и \(C\) на 1\,\% будет сопровождаться приростом производства более чем на 1\,\%, и система в %[Page 905]
целом будет характеризоваться возрастающей отдачей. В случае, когда \(k+j < 1{,}0\), система будет характеризоваться убывающей отдачей. Мы пересмотрели исследования временных рядов исходя из формулы Дюранда и обнаружили, что -- весьма примечательно -- значение суммы \(k + j\) оказалось достаточно близко к единице в соответствии с нашей первоначальной гипотезой. Значение \(k\) сократилось почти до двух третей, а именно до 0,65, а значение \(j\) возросло приблизительно до одной трети. В то же самое время Национальное бюро экономических исследований пересматривали результаты своих предыдущих исследований распределения промышленного производства в обрабатывающей промышленности и независимо пришли к актуальной оценке соотношения труда к производству на уровне примерно 65\,\%. Таким образом, два указанных независимых исследования производства и распределения оказались в точном соответствии друг с другом.

Вторым важным внесённым изменением стал переход от прежних исследований временных рядов к структурным исследованиям наблюдений по отдельным отраслям. Использование данных, представленных в виде временных рядов, при построении производственных функций сопряжено со множеством технических трудностей, возникающих в основном в силу необходимости расчёта сопоставимых значений индексов капитала и физического объёма производства; необходимые корректировки занимали много сил и времени и тем самым существенно ограничивали количество доступных для оценки числовых параметров функции годовых наблюдений. Также оставалась возможность того, что истинные значения показателей степеней изменялись динамически.

Всё перечисленное отмечалось критиками метода производственной функции, в числе которых были Хорст Мендерсхаузен и его научный руководитель, Рагнар Фриш. Они утверждали, что столь малое количество проведённых наблюдений свидетельствует о том, что какая-либо математическая зависимость является чистой случайностью и не отражает причинно-следственных связей. Они искренне полагали, что такой метод следует отбросить и, по словам Мендерсхаузена, что все прежние работы следует разорвать и отправить в мусорную корзину.
%Frisch's Econometric Laboratory and the Rise of Trygve Haavelmo's Probability Approach
%Olav Bjerkholt
%University of Oslo/Statistics Norway
%Econometric Theory
%Vol. 21, No. 3 (Jun., 2005), pp. 491-533 (43 pages)
%Published by: Cambridge University Press
%511
%1938 Econometrica article ``On the Significance of Professor Douglas' Production Function'' by Mendershausen

%Jesus Felipe
%John S.L. McCombie
%The Aggregate Production Function and the Measurement of Technical Change

Таким же было общее мнение старшего поколения американских экономистов, и нигде оно не было выражено столь явно, как среди моих коллег в Чикагском университете. Должен признать, что был обескуражен такого рода критикой и думал прекратить работы, но мне что-то подсказывало, что я должен продолжать. И потому вместе с мисс Ганн мы вопреки ожиданиям решили расширить поле наших исследований и принять в качестве объекта исследования наблюдения по отдельным отраслям в рамках отдельной экономической системы за отдельные годы \cite{Bronfenbrenner:1,Douglas:4,Douglas:5,Douglas:3}. Такой подход устранил проблему динамических изменений производственной функции, однако требовал принять допущение о существовании идентичных функций по всем отраслям.

Мы обнаружили большой объём статистического материала. Бюро переписи населения США рассчитали данные по \(P\), \(L\) и \(C\) за 1889, 1899\,гг. и за каждый 5-й год из последующих. Однако в конечном итоге в 1931\,г. официальная консультативная комиссия высокопоставленных экономистов и статистиков вынесла данным рядам приговор, мотивируя это тем, что таковые не стоят дальнейшей поддержки. Тем не менее, мы решили использовать имеющийся статистический материал и привлечь к работе некоторое количество помощников. Мы также обнаружили изобилие статистического материала по \(P\), \(L\) и \(C\) в ежегодных промышленных переписях по отдельным австралийским штатам в частности и по Содружеству Австралии в целом. Австралийские переписи были %[Page 906]
\begin{table}[!t]
\centering
\footnotesize{
\caption{Производственная функция на основе структурных исследований по США, 1904, 1909, 1914, 1919\,гг.}%
\label{tab1}%
\begin{tabular}{p{0.15\textwidth}
p{0.11\textwidth}
p{0.11\textwidth}
p{0.11\textwidth}
p{0.11\textwidth}
p{0.11\textwidth}
p{0\textwidth}}
\toprule
\toprule
\centering Год & \centering \(k\) & \centering \(\text{СО}_k\) & \centering \(j\) & \centering \(\text{СО}_j\) & \centering \(k + j\) & \\
\midrule
1904 & \hfil 0,65 & \hfil 0,02 & \hfil 0,31 & \hfil 0,02 & \hfil 0,96 & \\
1909 & \hfil 0,63 & \hfil 0,02 & \hfil 0,34 & \hfil 0,02 & \hfil 0,97 & \\
1914 & \hfil 0,61 & \hfil 0,03 & \hfil 0,37 & \hfil 0,02 & \hfil 0,98 & \\
1919 & \hfil 0,76 & \hfil 0,02 & \hfil 0,25 & \hfil 0,02 & \hfil 1,01 & \\
Среднее & \hfil 0,66 & \hfil 0,02 & \hfil 0,32 & \hfil 0,02 & \hfil 0,98 & \\
\bottomrule

\end{tabular}
}
\end{table}
учреждены крупным австралийским статистиком Дж.\,Х. Ниббсом, чьё имя заслуживает стоять в одном ряду с именами Кэррола\,Д.\,Райта и Р.\,Х.\,Коутса в числе виднейших специалистов по прикладной статистике столетия.

Мы успели завершить всего четыре структурных исследования по США (за 1904, 1909, 1914 и 1919\,гг.), охвативших 1\,490 наблюдений, когда наша работа была прервана войной. Ганн затем поступила на работу в правительственную статистическую службу; я был зачислен на службу в морскую пехоту. Публиковались статьи, представлявшие результаты исследований, однако в атмосфере вызванного войной напряжения они не привлекли большого внимания. Результаты, тем не менее, были достаточно примечательными (см. табли. \ref{tab1}).

Прежде всего необходимо отметить, что сумма независимо рассчитанных показателей степеней оказалась достаточно близка к единице, составляя в среднем 0,98, что всего на 2\,\% меньше требования условия постоянной отдачи от масштаба. Таким образом была подкреплена справедливость использования прежней функции с параметрами \(k\) и \(1-k\) в качестве первого приближения.

Относительно небольшие величины стандартных ошибок оценок дают дальнейшее подтверждение указанному уравнению. Значения стандартных ошибок \(k\) и \(j\) оцениваются всего в 0,02, что составляет всего одну тридцать третью и одну шестнадцатую от показателей степеней \(L\) и \(C\) соответственно. Таким образом этот результат даёт основание усомниться в справедливости одного из пунктов критики Фриша и Мендерсхаузена.

Работа была прервана на несколько лет войной и её последствиями. Тем не менее, в 1947\,г. мы вновь объединили усилия с мисс Ганн. Когда меня избрали президентом Американской экономической ассоциации, я решил посвятить свой доклад производственной функции, в котором бы освещались аспекты теории и эмпирического анализа проблемы. Ганн получила отпуск в своём учреждении. Таким образом, через 20 лет после исходной работы \cite{Douglas:1}, мы продолжили нашу программу исследований. Мы приступили к ещё нескольким структурным исследованиям по США и главным образом по Австралии. В общей сложности 695 наблюдений за 1889 и 1899\,гг. добавились к прежним исследованиям по США, составляя вместе с проведённым до 1942\,г. анализом совокупность, основанную на 2\,185 отраслевых наблюдениях. Значения \(k\) и \(j\) за 1889 и 1899\,гг. составили 0,51 и 0,62, и 0,43 и 0,33 соответственно. Таким образом суммы показателей степеней составили 0,94 и 0,95. Поскольку стандартные ошибки \(k\) составили 0,03 и 0,02 и были такими же для \(j\), представленные результаты убедительно демонстрируют по меньшей мере приближение к выполнению требования условия постоянной отдачи, %[Page 907]
причём наблюдается продолжительная тенденция к тому, что значение суммы \(k + j\) составляет чуть меньше единицы в течение трети века.

Структурные исследования по Содружеству Австралии были выполнены за 1913, 1923, 1927, 1935 и 1937 фин.\,гг., при этом мы также исследовали данные по штату Виктория за 1911, 1924 и 1928 фин.\,гг., и по штату Новый Южный Уэльс за 1934 фин.\,г. (см. табл. \ref{tab2}).

\begin{table}[!t]
\centering
\footnotesize{
\caption{Производственная функция Австралии за некоторые финансовые годы}%
\label{tab2}%
\begin{tabular}{p{0.25\textwidth}p{0.12\textwidth}p{0.1\textwidth}p{0.05\textwidth}p{0.1\textwidth}p{0.05\textwidth}p{0.05\textwidth}p{0pt}}
\toprule
\toprule
\centering Структурные исследования и финансовый год & \centering Количество наблюдений \((N)\) & \centering Значения \(k\) & \centering \(\text{СО}_k\) & \centering Значения \(j\) & \centering \(\text{СО}_j\) & \centering \(k + j\) & \\
\midrule
Австралия: & & & & & &  & \\
\hspace{3mm} 1913 & \hfil 85 & \hfil 0,52 & 0,05 & \hfil 0,47 & 0,05 & 0,99 & \\
\hspace{3mm} 1923 & \hfil 87 & \hfil 0,53 & 0,05 & \hfil 0,49 & 0,05 & 1,02 & \\
\hspace{3mm} 1927 & \hfil 85 & \hfil 0,59 & 0,05 & \hfil 0,34 & 0,04 & 0,93 & \\
\hspace{3mm} 1935 & \hfil 138 \phantom{\,} & \hfil 0,64 & 0,04 & \hfil 0,36 & 0,04 & 1,00 & \\
\hspace{3mm} 1937 & \hfil 87 & \hfil 0,49 & 0,04 & \hfil 0,49 & 0,04 & 0,98 & \\
Виктория: & & & & & &  & \\
\hspace{3mm} 1911 & \hfil 34 & \hfil 0,74 & 0,08 & \hfil 0,25 & 0,11 & 0,99 & \\
\hspace{3mm} 1924 & \hfil 38 & \hfil 0,62 & 0,08 & \hfil 0,31 & 0,10 & 0,93 & \\
\hspace{3mm} 1928 & \hfil 35 & \hfil 0,59 & 0,07 & \hfil 0,27 & 0,09 & 0,86 & \\
Новый Южный Уэльс: & & & & & &  & \\
\hspace{3mm} 1934 & \hfil 125 \phantom{\,} & \hfil 0,64 & 0,04 & \hfil 0,34 & 0,03 & 0,99 & \\
Среднее по всем исследованиям по Содружеству Австралии и по штатам & \hfil 714 \phantom{\,} & \hfil 0,60 & 0,06 & \hfil 0,37 & 0,06 & 0,97 & \\
Среднее по группе исследований по Содружеству Австралии & \hfil 482 \phantom{\,} & \hfil 0,55 & 0,04 & \hfil 0,43 & 0,04 & 0,98 & \\
Среднее по группе исследований по штатам & \hfil 232 \phantom{\,} & \hfil 0,65 & 0,07 & \hfil 0,29 & 0,08 & 0,94 & \\

\bottomrule

\end{tabular}
}
\end{table}

Анализ указанных данных даёт следующие результаты:
\begin{enumerate}[{1.}]
\item Сумма показателей степеней для труда и капитала достаточно близко приближается к единице, что соответствует требованию условия постоянной отдачи от масштаба, однако почти всегда остаётся чуть ниже единицы.
\item Значения стандартных ошибок \(k\) были относительно небольшими: в диапазоне от одной десятой до одной пятнадцатой значения \(k\). Значения стандартных ошибок \(j\) по Содружеству и по штату Новый Южный Уэльс также были относительно небольшими, хотя такие значения по штату Виктория как по \(k\), так и по \(j\) были значительно б\'{о}льшими.
\item Среднее значение показателя степени труда \((k)\) по всем девяти австралийским исследованиям составило 0,60; а такое значение показателя степени капитала \((j)\) составило 0,37. Значения \(k\) по группе исследований по Содружеству Австралии оказались немного ниже (в среднем -- 0,55), а значения \(j\) -- немного выше (в среднем -- 0,43). Средние значения по группе исследований по австралийским штатам показали обратную картину. Значения \(k\) были выше (в среднем -- 0,65), а значения \(j\) -- ниже (в среднем -- 0,29).
\end{enumerate}

Джордж Бринигар и Кит Кэмпбелл в дополнительном исследовании по штату Квинсленд за 1937--1938\,гг. оценили значения \(k\) и \(j\) на уровне 0,58 и 0,45 соответственно.

Если мы возьмём в качестве наиболее вероятных <<нормальных>> значений \(k\) и \(j\) по Австралии 0,60 %[Page 908]
и 0,37 соответственно, то коэффициенты эластичности кривых предельной производительности труда и капитала составят приблизительно 2,7 и 1,7 соответственно.

Вместе с тем мы не закончили исследования на данных по Австралии. Далее мы включили в нашу работу исследование данных по Канаде, Новой Зеландии и Южно-Африканскому Союзу. Мы провели четыре структурных исследования по Канаде, охватывающих 1923, 1927, 1935 и 1937\,гг., и включающих в общей сложности 659 отраслевых наблюдений \cite{Douglas:2}. В дополнение к перечисленным, Дж.\,У.\,Г. Браун [Browne] провёл два структурных исследования по Южно-Африканскому Союзу за 1937--1938\,гг., включающих в общей сложности 102 отраслевых наблюдения \cite{Browne:1}; Дж.\,В.\,Уильямс охватил в своём структурном исследовании 61 отрасль обрабатывающей промышленности Новой Зеландии за 1938--1939\,гг. \cite{Douglas:7}; Макс Браун [Brown] в своей неопубликованной диссертации на соискание степени доктора философии в Кембриджском университете изучил обрабатывающую промышленность Новой Зеландии за 18 лет в интервалах с 1915 по 1935\,гг. (интервал военных лет, 1917--1918\,гг., не рассматривался); Дж.\,В. Уильямс также изучил 18-летний период с 1923 по 1940\,гг.; К. С. Ломакс изучил британскую экономику за 1924 и 1930\,гг. \cite{Lomax:1}; и К.\,Э.\,В. Лезер изучил британскую угольную промышленность за 1943--1953\,гг. \cite{Leser:1}. Результаты структурных исследований по Южно-Африканскому Союзу, Канаде, Новой Зеландии и Великобритании представлены в табл. 3.

\begin{table}[!t]
\centering
\footnotesize{
\caption{Производственная функция для других стран Британского Содружества}%
\label{tab3}%
\begin{tabular}{p{0.2\textwidth}p{0.14\textwidth}p{0.08\textwidth}p{0.06\textwidth}p{0.08\textwidth}p{0.06\textwidth}p{0.08\textwidth}p{0pt}}
\toprule
\toprule
\centering Страна и Год межотраслевого или структурного исследования & \centering Количество отраслевых наблюдений \((N)\) & \centering Значение \(k\) & \centering \(\text{СО}_k\) & \centering Значение \(j\) & \centering \(\text{СО}_j\) & \centering Значение \(k + j\) & \\%
\midrule
Южно-Африканский Союз: & & & & & & &\\
\hspace{3mm} 1937--1938~гг.* & \hfil \phantom{\,} 17 & \hfil 0,66 & \hfil 0,08 & \hfil 0,32 & \hfil 0,08 & \hfil 0,98 &\\
\hspace{3mm} 1937--1938~гг.$\dagger$ & \hfil \phantom{\,} 85 & \hfil 0,65 & \hfil \dots & \hfil 0,37 & \hfil 0,08 & \hfil 1,02&\\
Канада: & & & & & & &\\
\hspace{3mm} 1923 & \hfil 167 & \hfil 0,48 & \hfil 0,04 & \hfil 0,48 & \hfil 0,04 & \hfil 0,96&\\
\hspace{3mm} 1927 & \hfil 163 & \hfil 0,46 & \hfil 0,04 & \hfil 0,52 & \hfil 0,04 & \hfil 0,98&\\
\hspace{3mm} 1935 & \hfil 165 & \hfil 0,50 & \hfil 0,04 & \hfil 0,52 & \hfil 0,04 & \hfil 1,02&\\
\hspace{3mm} 1937 & \hfil 164 & \hfil 0,43 & \hfil 0,04 & \hfil 0,58 & \hfil 0,04 & \hfil 1,01&\\
Новая Зеландия: & & & & & & &\\
\hspace{3mm} 1938--1939~гг.$\ddagger$ & \hfil \phantom{\,} 61 & \hfil 0,46 & \hfil \dots & \hfil 0,51 & \hfil \dots & \hfil 0,97&\\
Великобритания: & & & & & & &\\
\hspace{3mm} 1924$\mathsection$ & \hfil \dots & \hfil 0,72 & \hfil \dots & \hfil 0,18 & \hfil \dots & \hfil 0,90&\\
\hspace{3mm} 1930$\mathsection$ & \hfil \dots & \hfil 0,73 & \hfil \dots & \hfil 0,13 & \hfil \dots & \hfil 0,86&\\
\hspace{3mm} 1943--1953~гг.$\|$ & \hfil \phantom{M} 88$\#$ & \hfil 0,42 & \hfil 0,13 & \hfil 0,60 & \hfil 0,08 & \hfil 1,02&\\
\hspace{3mm} 1943--1953~гг.$\|$ & \hfil \phantom{M} 99$\#$ & \hfil 0,51 & \hfil 0,12 & \hfil 0,49 & \hfil 0,08 & \hfil 1,00&\\

\bottomrule
\end{tabular}
\raggedright{
\par
$\ast$ Исследование Дж.\,Брауна.
\par
$\dagger$ Исследование Дж.\,Брауна при разбивке данных по труду чернокожих и белых работников.
\par
$\ddagger$ Исследование Уильямса.
\par
$\mathsection$ Исследование Ломакса.
\par
$\|$ Исследование Лезера.
\begin{flushleft}
$\#$ Данное исследование совмещает временные ряды и региональные структурные наблюдения по британской угольной промышленности.
\end{flushleft}
}
}
\end{table}

Значение \(k\) для Южно-Африканского Союза при рассмотрении данных, охватывающих 17 отраслей, по труду чёрнокожих и белых работников в среднем составило 0,66, а при разбивке работников на две указанные группы и при более детальной разбивке отраслей -- 0,65. Значения \(j\) составили 0,32 и 0,37 соответственно, т. е. оказались практически идентичными. Более того, значения \(k\) и \(j\) оказались, по существу, такими же, как и те, что мы получили по исследованиям обрабатывающей промышленности США, и не сильно отличались от средних значений по всем исследованиям по Австралии. Тем не менее, значения \(k\) и \(j\) по Канаде были стабильно более низкими и более высокими соответственно, чем в аналогичных исследованиях по США и Австралии. Значения \(k\) лежат в пределах от 0,43 до 0,50, составляя в среднем 0,47, а значения \(j\) изменяются в пределах 0,48 до 0,58, составляя в среднем 0,52.

Значения \(j\) оказались больше соответствующих значений \(k\) также в структурных отраслевых наблюдениях по Новой Зеландии: \(k = 0{,}46\) и \(j = 0{,}51\). Такой же результат наблюдался при использовании формулы \(P=bL^kC^j\) для исследования временных рядов. Тем не менее, при использовании формулы \(P=bL^kC^{1-k}\) значения \(k\) оказались более высокими: 0,51 -- за период 1915--1935\,гг., и 0,54 -- за период 1923--1940\,гг.

По рассмотрении полученной группы неоднородных исследований в совокупности, мы вновь обнаружили, что сумма показателей степеней \(k + j\) приближается к единице, но оказывается чуть ниже этого значения. Исследования Ломакса в наименьшей степени согласовывались с условием наличия постоянной отдачи от масштаба, однако сложно было оценить достоверность этих результатов, поскольку в этих исследованиях не сообщались значения соответствующих стандартных ошибок.

Значения стандартных ошибок \(k\) по всем четырём структурным исследованиям Канады составляли 0,04, или от одной одиннадцатой до одной двенадцатой соответствующего значения \(k\). Значения стандартных ошибок \(j\) также составляли 0,04 по всем четырём исследованиям, или от одной двенадцатой до одной тринадцатой соответствующего значения \(j\). Значения стандартных ошибок по двум исследованиям по Южно-Африканскому Союзу оказались немного выше, составляя для всех трёх проведённых оценок 0,08, что было равно одной восьмой соответствующего значения \(k\) и от одной четвёртой до одной пятой соответствующих значений \(j\). %[Page 909]

%Мультиколлинеарность
Рассматриваемый далее критерий достоверности гипотетической зависимости заключается в рассмотрении невязок между фактическими значениями производства и расчётными значениями производственной функции\footnote{Анализ разностей фактических и теоретических значений, выраженных в стандартных ошибках оценок, впервые был введён в исследовании Бронфенбреннера--Дугласа.}. Мы предположили, что положительные невязки будут приходиться на те отрасли, для которых характерны (\emph{а}) монополия или высокая степень монополизации или (\emph{б}) растущий спрос, при котором кривая спроса сдвигается параллельно вправо. В таких отраслях цены будут более высокими, чем можно было бы наблюдать в долгосрочной перспективе при совершенной конкуренции. Поскольку производство исчислено в стоимостном выражении, это приводит к положительным невязкам. 19 из 49 отраслевых наблюдений, данные по которым показали положительные невязки в размере более 2 СО, характеризовались преимущественно несовершенной конкуренцией, как например: резьба по дереву, аффинаж золота и серебра, гранильное производство, издание музыкальных произведений, производство глюкозы, крахмала, льняной олифы, патентованных лекарственных средств, жести, латуни и свинца. По меньшей мере шесть отраслевых наблюдений характеризовались преимущественно растущим спросом, а именно: ликёры и кондитерские сиропы (1909, 1914, 1919\,гг.), маргарин (1914\,г.), парфюмерия (1919\,г.) и стиральные машины (1919\,г.). Также по меньшей мере 12 отраслевых наблюдений характеризовались как несовершенной конкуренцией, так и растущим спросом: самолёты (1914\,г.), жевательная резинка (1919\,г.), сигары и сигареты (1919\,г.), перьевые ручки (1914\,г.), фотографические принадлежности и оборудование (1904, 1909 и 1914\,гг.), контрольно-кассовые машины (1889\,г.), выплавка и рафинирование меди (1899, 1904\,гг.), пишущие машины и сопутствующие принадлежности (1889\,г.). Перечисленные выше отраслевые наблюдения в совокупности представляют %[Page 910]
37 из 49 случаев выявленных значительных положительных невязок. Вследствие чего мы пришли к выводу о том, что подавляющее большинство выявленных значительных положительных невязок по всей генеральной совокупности наблюдений по США приходились в точности на те отрасли, круг которых мог быть определён исходя из априорных предпосылок. Такой вывод убедительно говорит в пользу достаточно хорошей объяснительной силы производственной функции в отношении нормальных конкурентных отношений.

В силу ограниченных временных рамок мы не проводили столь же подробное исследование 38 значительных отрицательных невязок. Мы предположили, что такие невязки будут приходиться на те отрасли, для которых характерны избыточное предложение труда или капитала, падающий спрос, или преимущественное использование низкоквалифицированного труда. Пятнадцать случаев, данные наблюдений по которым показали отрицательные невязки, пришлись на отрасли с избыточным предложением труда: 10 таких случаев наблюдались в группе отраслей, производящих лён, пеньку, льняную ткань, джут и паклю. Ещё три случая пришлись на хлопковую промышленность, в которой, как правило, используется низкоквалифицированный труд. Два случая: гамаки (1889\,г.) и рыболовецкие сети и невода (1914\,г.) -- пришлись на смежные отрасли. Кроме того, ещё три случая: древесный уголь, отходы и консервирование устриц -- пришлись на весьма неоднородные отрасли. Перечисленные выше отраслевые наблюдения составляли в совокупности по меньшей мере 21 случай из общего количества выявленных случаев значительных отрицательных невязок, которые объяснялись избыточным предложением труда или использованием низкоквалифицированного труда. Ещё несколько таких случаев объяснялись сжимающимся спросом: в отраслях, производящих точильные камни, жернова, крюки и огоны. К сожалению, у нас не было возможности провести равноценный по подробности анализ такого класса случаев в силу нехватки времени, обусловленной событиями, о которых я расскажу далее по тексту.

Впрочем, было накоплено достаточно материалов, свидетельствующих о том, что большая часть значительных отклонений фактических от теоретических значений \(P\) объяснялись динамическими отклонениями от <<нормального состояния>>, которые, к добру или к худу, служили причиной существенных отклонений фактической <<производительности>> от формулы, характеризующей <<нормальные>> условия. Таким образом, аргументация в пользу применения рассматриваемой функции становится более убедительной. Более того, поскольку воздействие факторов несовершенной конкуренции, растущего спроса, избыточного предложения труда и сжимающегося спроса обусловили столь большое количество значительных ошибок, представляется весьма вероятным, что многие меньшие отклонения были обусловлены теми же факторами.

Диаграммы распределения фактических значений производства вдоль линии предполагаемой теоретической зависимости были подготовлены и представлены по всем 15 исследованиям США и британских доминионов, но в силу ограничений по объёму и затратам всего три из них могли быть опубликованы в докладе. За одним или двумя исключениями материалы в общей сложности 3\,558 наблюдений служат убедительным подтверждением как формулы, так и функции.

Я представил все перечисленные результаты в своём президентском докладе перед Американской экономической ассоциацией в Чикаго в 1947\,г. По завершении своего доклада, я предложил оппонентам проверить использованные статистические материалы и выводы работы и выразил готовность предоставить необходимые данные в их распоряжение. Насколько мне известно, никто этим предложением так и не воспользовался. Разумеется, ранее высказанное возражение о том, что выводы работы основываются на недостаточном количестве рассмотренных случаев, более не может быть выдвинуто.
%[Page 911]

\section*{После Доклада 1947\,г.}

Моё выступление с президентским докладом в Чикаго совпало по времени и городу проведения со съездом Национального комитета Демократической партии, в повестке которого значились вопросы отбора кандидатов для избрания на посты сенатора и губернатора. Как только я закончил наряд и уже собирался уходить с супругой на конференцию, раздался телефонный звонок. На проводе был мой друг, полковник Джек Арви: он сказал, что я только что был выдвинут кандидатом на пост сенатора (от штата Иллинойс -- \emph{Прим. перев.}), и спросил моего согласия. Получив одобрительный жест от супруги, я поблагодарил его и комитет и пообещал, что сделаю всё от меня зависящее. Уже по пути в конференц-зал я заметил, что это может быть концом моих научных исследований, и поймал себя на том, что повторяю слова персонажа Отелло: <<Ныне ж навсегда прощай, спокойная жизнь% Вильям Шекспир. Отелло, венецианский мавр (Пер. Б. Н. Лейтина)
>>\Footnote{а}{По \emph{Вильям Шекспир, <<Отелло, венецианский мавр>>, Пер. Б.\,Н. Лейтина, 1968}.}. Она не была спокойной ранее, ещё менее таковой она оказалась.

Здесь не место рассказу о дальнейшем ходе событий, кроме того, как признать, что в течение 18 лет на посту в Сенате США % January 3, 1949--January 3, 1967
и двух досадных лет на посту председателя президентской комиссии% National Commission on Urban Problems, the Douglas Commission, 1967--1968
, у меня не было возможности посвятить время работе над производственной функцией. Затем я выбыл из состава комиссии ещё на год по причине перенесённых мной <<азиатского гриппа>>, пневмонии, инфаркта и, в довершение ко всему, инсульта средней степени тяжести. По выздоровлении, я вновь вернулся к изучаемому вопросу и обнаружил, что поток критических статей прежних оппонентов практически сошёл на нет, что функция Кобба--Дугласа нашла широкое применение и что многие молодые учёные, в авангарде которых стоят мой бывший студент, Пол Самуэльсон, его коллега Солоу и Марк Нерлов, сын моего друга и бывшего коллеги, Самуэля Нерлова, прорываются в новые и технически более сложные области исследований. Первым делом я предложил молодым учёным воспользоваться обширным статистическим материалом по Канаде, Австралии и Новой Зеландии в целях проведения эмпирических исследований. Также совместно с сенатором Проксмайром мы обратились к американским статистическим органам и Конгрессу США с предложением пересмотреть принятое ими ранее по авторитетному совету упомянутой выше группы экономистов и статистиков решение об отмене сбора статистических данных по капиталу, начиная с 1920--1921\,гг., в сторону возобновления такой работы. Затем, ввиду отсутствия видимых признаков того, что такое предложение нашло ожидаемый отклик, вместе с мисс Ганн мы решили провести такие исследования самостоятельно. Таким образом почти полвека после исходной работы Кобба и Дугласа и без малого четверть века после моего президентского доклада мы устроили четвёртый бросок в область исследования производственной функции.

Мы выбрали данные последних лет по Австралии: за 1956, 1957, 1964, 1965, 1966, 1967 и 1968 фин.\,гг.\footnote{Рей Б. Маду любезно предоставил нам данные по США за 1967\,г., которые он использовал в работе над своей готовящейся диссертацией на соискание степени доктора философии (<<Производство, эффективность и масштаб в США. Обрабатывающая промышленность [1967], меж-внутри-отраслевой анализ>>, Калифорнийский университет в Беркли). По итогам расчётов межотраслевой производственной функции на основе этих данных по 50 отраслям я получил значения \(k = 0{,}60\) и \(j = 0{,}40\) при \(R^2=0{,}95\).} Выборка охватывает примерно 160 отраслей за каждый из рассматриваемых годов или 1\,123 наблюдения в общей сложности. Последние вместе с прежними исследованиями составляют совокупность из 2\,496 наблюдений в рамках структурного анализа по странам Британского Содружества, а вместе с 2\,185 структурными %[Page 912]
наблюдениями по США составляют совокупность в общей сложности из 4\,681 наблюдения. С учётом сказанного выше утверждение о том, что наши выводы основываются на недостаточном количестве рассмотренных случаев, едва ли может быть признано состоятельным.

Значения \(k\) и \(j\), а также соответствующих стандартных ошибок, полученных на основе данных последних лет, представлены в табл. 4. Приведённые результаты подтверждают полученные ранее выводы. Значения сумм \(k + j\) стремятся к единице, что свидетельствует в пользу того, что экономическая система характеризуется приблизительно постоянной отдачей от масштаба. Как и в ранее полученных результатах, величины наблюдаемых незначительных отклонений от единицы имеют отрицательные значения. Вместе с тем такие отклонения не превышают стандартных ошибок расчётных показателей.

\begin{table}[!t]
\centering
\footnotesize{
\caption{Производственная функция австралийской обрабатывающей промышленности, 1956, 1957, 1964--1968\,гг.}%
\label{tab4}%
\begin{tabular}{lcccccc}
 \toprule
 \toprule
Финансовый год & \(N\) & \(k\) & \(\text{СО}_k\) & \(j\) & \(\text{СО}_j\) & \(k + j\)\\
\midrule
1956 & 159 & 0,615 & 0,03 & 0,365 & 0,02 & 0,980\\
1957 & 159 & 0,610 & 0,03 & 0,381 & 0,02 & 0,991\\
1964 & 163 & 0,595 & 0,03 & 0,396 & 0,03 & 0,991\\
1965 & 161 & 0,576 & 0,03 & 0,414 & 0,03 & 0,990\\
1966 & 161 & 0,562 & 0,03 & 0,434 & 0,03 & 0,996\\
1967 & 160 & 0,575 & 0,03 & 0,425 & 0,03 & 1,000\\
1968 & 160 & 0,536 & 0,03 & 0,456 & 0,03 & 0,992\\
\bottomrule
\end{tabular}
}
\end{table}

Перейдём теперь к рассмотрению невязок, полученных на основе данных последних лет\footnote{При распределении ошибок регрессии согласно нормальному закону, абсолютная величина каждой из них не превышала бы значения 1 СО приблизительно в 68\,\% случаев. В целом, ошибки в приведённой здесь работе распределены таким образом, что абсолютная величина каждой из них в 70--75\,\% случаев находится в пределах значения 1\,СО.}. Состав отраслей, для которых расчётное значение производства превышало фактический объём производства на величину более 2 СО, в 7 рассмотренных годах оставался практически неизменным. Фактические объёмы производства оружия и боеприпасов оказались меньше соответствующих расчётных значений по всем 7 рассмотренным годам, то же самое наблюдалось в производстве взрывчатых материалов в 5 годах из 7. Фактические объёмы превысили расчётные значения в отраслях, производящих лекарственные средства, табачные изделия, моющие средства, свечи и граммофоны, во всех 7 рассмотренных годах. С учётом того, что последние активно рекламируют свою продукцию, вполне возможно, что невключение стоимости гудвилла в состав основных фондов стало одной из причин заниженных оценок в данных отраслях. Данные отрасли, по всей вероятности, также характеризовались несовершенной конкуренцией.

Зависимость \(W/P\) и \(k\) представляет существенный интерес с точки зрения теории распределения. В табл. 5 представлены значения данных показателей по 7 годам из периода 1950--1960-х\,гг. Очевидно, что между \(k\) и \(W/P\) наблюдается тесная корреляция. В 1956 и 1957\,гг. \(W/P\) было всего на 1--3 процентных пункта ниже \(k\), а в 1964--1968\,гг. величина такой разницы лежала в пределах от 2 до 7 процентных пунктов.

\begin{table}[!t]
\centering
\footnotesize{
\caption{Зависимость между \(\frac{W}{P}\) и \(k\) по обрабатывающей промышленности Австралии, 1956,\,1957, 1964--1968\,гг.}
\label{tab5}
\begin{tabular}{ l C{2cm}C{2cm} }
 \toprule
 \toprule
\multirow{2}{*}{Финансовый год}
 & \multicolumn{2}{c}{\parbox[c]{4cm}{Производственная функция по всем отраслям обрабатывающей промышленности}} \\
 \cmidrule(lr){2-3}
 & \(k\) & \(\frac{W}{P}\) \\
 \midrule
1956 & 0,615 & 0,602\\
1957 & 0,610 & 0,581\\
1964 & 0,595 & 0,527\\
1965 & 0,576 & 0,530\\
1966 & 0,562 & 0,528\\
1967 & 0,575 & 0,517\\
1968 & 0,536 & 0,514\\
 \bottomrule
\end{tabular}
}
\end{table}

Даже с учётом наблюдаемых малых расхождений следует подчеркнуть, что соответствующие значения \(k\) и \(W/P\) являются практически равными друг другу. Данный факт является центральным: он служит дальнейшим теоретическим подкреплением производственной функции и демонстрирует строгое соответствие ожидаемой зависимости распределения %[Page 913]
произведённого продукта от предельной производительности труда работников в преимущественно конкурентном обществе. Таким образом, теоретические представления производительности и распределения взаимно дополняют друг друга.

%сти значительного повышения заработной платы (оно якобы ведет к сокращению прибыли и инвестиций, что конечном счете снижает производство и занятость, т. е. использование положений так называемой "конкурентной модели распределения" в синтезе с кейнсианской теорией регулирования спроса и предложения); 6) наличия так называемого функционального и
%
%Значительная масса независимых работ склоняется к подтверждению исходной формулы Кобба-Дугласа. Но более важно (!) то, что приблизительное совпадение рассчитанных коэффициентов с действительно получаемой долей также усиливает конкуренционную теорию распределения и опровергает марксизм". "Очевидно больше не может быть приемлем грубый марксистский лозунг: "Капитал -- это мертвый труд, который, как вампир, оживает лишь тогда, когда всасывает живой труд и живет тем полнее, чем больше живого труда он поглощает". Вместо этого лозунга новый: капитал сам производителен, капитал - не эксплуататор"
%Сергей Владимирович Казанцев. Макромоделирование расширенного воспроизводства. сс. 88--89
%Обычно теория "вменения

Если бы существовала возможность учесть платежи всем сторонним работникам, то степень соответствия \(W/P\) и \(k\) (либо \(k/[k + j]\)) была бы ещё более высокой. Между тем, как отмечалось, \(W\) не учитывает ни торговых посредников, в т.\,ч. коммивояжёров; ни ломовых извозчиков, работающих на производственные предприятия; ни работников, отпускающих готовую продукцию в розницу со складов предприятий. Трудовая деятельность таких людей составляет неотъемлемую часть производственного процесса. Однако же они получают заработную плату и жалованье на этапе, предшествующем распределению процентов и прибыли. Следовательно, в целях получения более достоверных величин долей произведённого продукта, приходящихся на труд, \(W\) должно учитывать их деятельность. Насколько увеличится \(W/P\) при учёте платежей сторонним работникам, разумеется, неизвестно. Впрочем, по моему мнению, введения таких дополнительных платежей в \(W\) будет недостаточно для того, чтобы избавиться от разницы между \(k\) и \(W/P\).

\section*{Будущие исследования}

Результаты приведённого здесь исследования дают дальнейшее подтверждение в пользу корректности производственной функции в качестве модели, описывающей промышленное производство и характеризующей распределение произведённого продукта; второй аспект хотя и является самостоятельным предметом исследования, однако он находится в тесной связи с первым.

Можно также возразить, что различные отрасли описываются различными производственными функциями, однако вместе с тем степень подтверждения логарифмического выражения производственной функции с постоянной отдачей от масштаба, которая была обнаружена для межотраслевых исследований, в самом деле является весьма внушительной. Структурные наблюдения по отдельным отраслям легли в основу целого ряда исследований. В силу своего построения в них удаётся избежать возможных трудностей, связанных с использованием данных различных отраслей для определения общей для всех функции, тем не менее, их выводы согласовываются с приведёнными здесь результатами.

%[Page 914]

Наиболее интересной из таких работ является работа Бенджамина Клотца, в которой использованы данные по 17 агрегированным группам отраслей обрабатывающей промышленности США за 1957\,г. и 1963\,г. \cite{Klotz:1}. Несмотря на то, что полученные им показатели значительно различаются, и то, что большинство отраслей демонстрируют убывающую отдачу от масштаба, Клотц отмечает, что ни по одной из отраслей данные решительно не противоречат гипотезе о наличии постоянной отдачи от масштаба. Ровно к таким же выводам мы приходили на основе других массивов данных, охватывающих наблюдения в течение почти 50 лет.

Подобное исследование на основе данных по отдельным предприятиям Норвегии за 1963\,г. во многом вторит работе Клотца \cite{Griliches:1}. Грилихес и Рингстад оценивают значения показателей труда и капитала на уровне 0,865 и 0,199 соответственно для всей совокупности рассмотренных в их работе наблюдений. Тем самым, они выявляют совершенно незначительно возрастающую отдачу от масштаба. Авторы приходят к выводу о том, <<что улучшить точность оценок регрессии путём использования форм, отличных от простейшей формы Кобба--Дугласа, представляется чрезвычайно трудной задачей>>.

%Page 62
%1. Overview: total manufacturing
%The results of estimating Cobb-Douglas and related functional forms by the method of least squares from the pooled total manufacturing sample are presented in table 4.1. They indicate
%<...>
%Page 63
%that it is very hard to improve upon the simple Cobb-Douglas form. Attempts to estimate the elasticity of substitution from by approximation
%
%the main results of our study are presented in this chapter. in discussing them we shall concentrate on the pooled total manufacturing regression results and on the "average" results for the individual industry estimates. we shall not consider the results for individual industries in any great detail; that will be done in appendix a. we will allude in places to various additional exploratory studies
%attempts to estimate the elasticity
%V = value added; L = total man-hours; SK = capital services
%Derived from the estimated

Значительный корпус независимых работ имел свойство подтверждать исходную формулу Кобба--Дугласа, но -- что более важно -- приблизительное совпадение соответствующих полученных показателей с фактическими долями национального продукта, приходящегося на оплату труда, также подкрепляет принцип конкурентного распределения доходов и опровергает марксистский. На многие из исходных возражений были даны ответы. Некоторые остались неотвеченными.

Тем не менее очевидно, что более не может быть приемлем грубый марксистский лозунг: <<Капитал -- это мёртвый труд, который, как вампир, оживает лишь тогда, когда всасывает живой труд и живёт тем полнее, чем больше живого труда он поглощает%К. Маркс, Капитал, Т. 1. Пер. И. И. Степанова-Скворцова, 1952. с. 238
>>\Footnote{б}{К. Маркс, Капитал, Т. 1. Пер. И. И. Степанова-Скворцова, 1952. с. 238.}. Напротив, капитал как таковой имеет свойство быть производительным, а не эксплуатирующим. Капитал будет создавать свою часть промышленного продукта как в коммунистическом, так и в обществе капитализма <<всеобщего благосостояния>>. Моральный вопрос о том, кому должен принадлежать капитал и как такая собственность должна быть институционализирована, по-прежнему остается открытым и подлежит решению по существу. Как бы то ни было, некоторые реляции можно и должно отбросить, а именно: моральные, экономические и технологические. При этом истинными провозвестниками нового порядка являются не Маркс или Ленин, но Роберт Оуэн и британские фабианцы.

Ожидалось, что данное дополнительное исследование по 7 указанным годам станет лишь прологом исследования за без малого 60-тилетний период (1912--1970\,гг.). Ожидаемым результатом такого исследования было получение итоговых значений \(k\) и \(j\) по Австралии, а также соотношений между ними и более точными значениями \(W/P\). Откуда также можно было бы в больших подробностях проследить, какие изменения, если таковые имели место, были обусловлены экономическим циклом, а также установить, происходили ли долгосрочные изменения \(k\), \(j\) и \(W/P\) и менялся ли наблюдаемый характер невязок. Но время берёт своё, и ответ на вопрос о том, буду ли я в состоянии завершить намеченное исследование, далеко не так очевиден. Поэтому я по-прежнему призываю к помощи молодого поколения экономистов и статистиков. Четверть века назад такая работа несла определённые научные риски для автора. Оппоненты производственной функции были уважаемыми, влиятельными и непреклонными. Таких опасностей сегодня не существует. Настоящее время благоприятствует такого рода исследованиям. Посему выражаю надежду на возможное проведение таких исследований. Невзирая на все сложности, я намерен продолжить и завершить, если это будет в моих силах, долгосрочное исследование функции по Австралии за 1912--1970\,гг.

%[Page 915]

\begin{thebibliography}{00}
%%
\bibitem{Bronfenbrenner:1}
Bronfenbrenner, M., and Douglas, P. H. ``Cross Section Studies in the Cobb-Douglas Function 1909''. \emph{J.P.E.} 47 (December 1939): 761--85.
%%
\bibitem{Browne:1}
Browne, G. W. G. \foreignlanguage{english}{``The Production Function for South African Manufacturing Industry''. \emph{South African J. Econ.} 11 (1943): 259.}
%%
\bibitem{Cobb:1}
Cobb, C. W. \foreignlanguage{english}{``Production in Massachusetts Manufacturing, 1890--1928.'' \emph{J.P.E.} 38, no.~6 (December 1930): 705--7.}
%%
\bibitem{Douglas:1}
Cobb, C. W., and Douglas, P. H. \foreignlanguage{english}{``A Theory of Production.'' \emph{A.E.R.} 8, no.~1, suppl. (March 1928): 139--65.}
%%
\bibitem{Douglas:2}
Daly, P., and Douglas, P. H. \foreignlanguage{english}{``The Production Function for Canadian Manufacturing Industry.'' \emph{J. American Statis. Assoc.} 38 (1943): 78--86.}
%%
\bibitem{Douglas:3}
Daly, P.; Olson, E.; and Douglas, P. H. \foreignlanguage{english}{``The Production Function for Manufacturing in the United States in 1904.'' \emph{J.P.E.} 51, no.~1 (February 1943): 61--65.}
%%
\bibitem{Durand:1}
Durand, D. \foreignlanguage{english}{``Some Thoughts on Marginal Productivity with Special Reference to Professor Douglas.'' \emph{J.P.E.} 45, no.~6 (December 1937): 740--58.}
%%
\bibitem{Griliches:1}
Griliches, Z., and Ringstad, V. \foreignlanguage{english}{\emph{Economies of Scale and the Form of the Production Function}. Amsterdam: North-Holland, 1971.}
%%
\bibitem{Douglas:4}
Gunn, G. T., and Douglas, P. H. \foreignlanguage{english}{``The Production Function for American Manufacturing in 1919.'' \emph{A.E.R.} 31 (March 1941): 67--80.}
%%
\bibitem{Douglas:5}
Gunn, G. T., and Douglas, P. H. \foreignlanguage{english}{``The Production Function for American Manufacturing for 1914.'' \emph{J.P.E.} 50, no.~4 (August 1942); 595--602.}
%%
\bibitem{Douglas:6}
Handsaker, M. L., and Douglas, P. H. \foreignlanguage{english}{``The Theory of Marginal Productivity Tested by Data for Manufacturing in Victoria.'' \emph{Q.J.E.} 52 (November 1937 and March 1938); 1--36; 215--54.}
%%
\bibitem{Klotz:1}
Klotz, B. P. \foreignlanguage{english}{``Productivity Analysis in Manufacturing Plants.'' Bureau of Labor Statistics Staff Paper no.~3, U.S. Dept. Labor, 1970.}
%%
\bibitem{Leser:1}
Leser, C. E. V. \foreignlanguage{english}{``Production Functions and British Coal Mining.'' \emph{Econometrica} 23 (October 1955): 442--46.}
%%
\bibitem{Lomax:1}
Lomax, K. S. \foreignlanguage{english}{``Production Functions for Manufacturing Industry in the United Kingdom.'' \emph{A.E.R.} 40 (June 1950): 397--99.}
%%
\bibitem{Mitchell:1}
Mitchell, W. C. (ed.); King, W. I.; Macauley, F. R.; and Knauth, O. W. \foreignlanguage{english}{\emph{Income in the United States: Its Amount and Distribution. II, Detailed Report}. New York: Nat. Bur. Econ. Res., 1922.}
%%
\bibitem{Douglas:7}
Williams, J. W., and Douglas, P. H. \foreignlanguage{english}{``Production Functions.'' Econ. Rec. 21 (1945): 55--63.}

\end{thebibliography}
\end{document}