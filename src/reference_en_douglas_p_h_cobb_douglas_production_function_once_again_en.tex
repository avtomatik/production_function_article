\documentclass{article}
\usepackage[utf8]{inputenc}
\usepackage[english,main=russian]{babel}
\usepackage[T2A]{fontenc}
%\usepackage{indentfirst}
\usepackage{amsmath,amsfonts,amssymb}
\usepackage{enumerate}
%\usepackage[symbol*]{footmisc}

\title{The Cobb-Douglas Production Function Once Again: Its History, Its Testing, and Some New Empirical Values}

\author{
	Paul H. Douglas\\
	\emph{Washington, D.C.}\thanks{Work on the production function was carried on by a large number of persons who deserve to be credited as coauthors. Foremost among these is Grace Gunn who participated, over a period of 40 years, in no less than three separate phases of the work. Mary Hook helped in the later stages and Stanley Horowitz and Christopher Jehn in the final preparation of this article. I am indebted to all of them and many others for their help. Others who helped in the production were Marjorie Handsaker, Martin Bronfenbrenner, and Patricia Daly.}
}
\makeatletter
\def\thanks#1{\protected@xdef\@thanks{\@thanks
        \protect\footnotetext{#1}}}
\makeatother
\begin{document}
\selectlanguage{english}
\renewcommand{\abstractname}{Editors' Note.}
\maketitle

%[Page 903]
\begin{abstract}
-- We are pleased to have been offered, and to offer the readers of the \emph{Journal}, this account of the early development of empirical work on the Cobb-Douglas function by its coinventor and pioneer economic investigator, Paul H. Douglas.
%{1: [Journal of Political Economy, 1976, vol. 84. no. 5]/® 1976 by The University of Chicago, All rights reserved.}

Research into the production function has a long history. Since the first work, in 1928, many studies have tended to support the hypothesis that production processes are well described by a linear homogeneous function with an elasticity of substitution of one between factors. New results are presented here, using 7 years of observations on Australian manufacturing industries during the 1950s and 1960s. In all seven cases, constant returns to scale are very closely approximated, and the coefficient for labor hovers near .60. The appropriate coincidence of the estimated coefficients with the shares received strengthens the competitive theory of distribution.
\end{abstract}


This paper is an effort to continue and extend earlier studies of the production function that were first begun nearly a half century ago. For it was in 1927 that I computed the index numbers of the total number of manual workers \((L)\) employed in American manufacturing by years from 1899 to 1922, did the same for fixed capital \((C)\), expressed these in logarithmic terms on a chart, and then added the index for physical %[Page 904]
production \((P)\) in manufacturing.\footnote{This index had been developed by E. E. Day.} I found the curve for product to lie, in general, approximately one-quarter of the distance between the curve for labor, which had increased the least (to 162), and the curve of capital, which had increased the most (to 431). The year 1899 was taken as 100. I was then temporarily lecturing at Amherst College, and consulted with my friend and colleague, Charles W. Cobb, a mathematician. At the latter's suggestion, the formula \(P=bL^kC^{1-k}\) was adopted, a form that had also been used by Wicksteed and Wicksell. This, following Euler, was a simple homogeneous function of the first degree. After finding the value of \(k\) by the method of least squares to be .75, we found that the estimated values of \(P\) closely approximated the actual values for the 23-year period --- with such differences as occurred being due primarily to the business cycle \cite[pp. 139--65]{Douglas:1}. The National Bureau of Economic Research had found that the average share of products going to labor during the decade 1909--18 was 74.1 percent,\footnote{See Mitchell et al. 1922, p. 98.} or an almost precise agreement with the values of the production function. I felt, therefore, that the marginal productivity theory of wages had received a substantial degree of confirmation.

Later studies by Cobb for Massachusetts, 1890--1928 \cite{Cobb:1}, and by Director for New South Wales, 1901--27 (unpublished), gave values which were identical in the case of Massachusetts \((k = .743)\) and similar in the case of New South Wales \((k = .65)\).

We made one more study of a time series. In 1936, in cooperation with Mrs. Marjorie Handsaker, series of \(P\), \(L\), and \(C\) were worked out for the years 1907--29 for the state of Victoria in Australia, and the value of \(k\) was found to be .71. Labor's share of the product \(W/P\) was computed as .61 for this period \cite{Douglas:6}.

\section*{Research between 1937 and 1947}

With the David Durand article (1937) and the coming of Grace Gunn to the staff in the same year, several changes were made. First, following the suggestion by Durand, the formula \(P=bL^kC^{1-k}\) was changed to \(P=bL^kC^j\), thus making the exponent of \(C\) independently determined instead of treating it as the residual in a homogeneous linear equation. With \(j\) independently determined, the production function was no longer constrained to be homogeneous of degree 1, but could instead take such form as the actual figures might dictate. If \(k + j = 1.0\), the economic system was subject to constant returns to scale. If \(k + j\) was greater than 1.0, then a 1 percent increase in both \(L\) and \(C\) would be accompanied by an increase of more than 1 percent in product, and the system as a %[Page 905]
whole would operate under increasing returns. If \(k + j\) was less than 1.0, then the system was characterized by diminishing returns. We corrected the time studies according to the Durand formula and found, interestingly enough, that the sum of \(k + j\) was very close to the previous assumption of unity. The values of \(k\) were reduced to close to two-thirds, that is, .65, and \(j\) to approximately one-third. At the same time, the National Bureau of Economic Research was revising its previous studies on the distribution of product in manufacturing and had independently arrived at approximately .65 percent as labor's actual share. The two independent studies of production and of distribution were, therefore, once again in unison.

The second important change introduced was to substitute cross-section studies of separate industry observations for the previous time series. The use of time-series data in computing the production functions carried with it a host of technical problems, mainly resulting from the necessity of calculating comparable index numbers for capital and for physical product; the necessary adjustments were complicated and time consuming and, thus, severely limited the number of yearly observations available for estimating the function. There was also the possibility that the true values of the exponents changed over time.

All this was noted by such critics of the production function analysis as Horst Mendershausen and his mentor, Ragnar Frisch. They urged that so few observations were involved that any mathematical relationship was purely accidental and not causal. They believed sincerely that the analysis should be abandoned and, in the words of Mendershausen, that all past work should be torn up and consigned to the wastepaper basket.

This was also the general sentiment among senior American economists, and nowhere was it held more strongly than among my colleagues at the University of Chicago. I must admit that I was discouraged by this criticism and thought of giving up the effort, but there was something which told me that I should hold on. Miss Gunn and I, therefore, determined instead to broaden the scope of our studies and to take as our observations individual industries within a given economy in a given year (Bronfenbrenner and Douglas 1939; Gunn and Douglas 1941, 1942; Daly, Olson, and Douglas 1943). This methodology eliminated the problem of changes in the production function over time, but it had to assume identical functions across industries.

We found a wealth of material. The American census had computed data on \(P\), \(L\), and \(C\) for 1889, 1899, and every 5 years thereafter. Finally in 1931 the series was put to death by an official advisory commission of eminent economists and statisticians. Not improvement but decapitation was their motto. We determined, however, to utilize the existing material and worked with several collaborators. We also found abundant statistics on \(P\), \(L\), and \(C\) in the annual censuses of manufacturing for the various Australian states and for the commonwealth as a whole. These were %[Page 906]
\begin{table}[!t]
\centering
\footnotesize{
\caption{Production Function Based on American Cross-Section Studies, 1904, 1909, 1914, 1919}%
\label{tab1}%
\begin{tabular}{l|c|c|c|c|c}
\hline
\hline
Year & \(k\) & SE of \(k\) & \(j\) & SE of \(j\) & \(k + j\) \\
\hline
1904 & .65 & .02 & .31 & .02 & .96 \\
1909 & .63 & .02 & .34 & .02 & .97 \\
1914 & .61 & .03 & .37 & .02 & .98 \\
1919 & .76 & .02 & .25 & .02 & 1.01 \\
Average & .66 & .02 & .32 & .02 & .98 \\
\hline

\end{tabular}
}
\end{table}
started by the great Australian statistician G. H. Knibbs, who deserves to rank with Carroll D. Wright and R. H. Coats in the front rank of the applied statisticians of the century.

We had time to finish only four cross-section American studies (1904, 1909, 1914, and 1919) involving 1,490 observations when the war intervened. Gunn then entered the governmental statistical service; I enlisted in the marines. Articles were published giving the results, but in the excitement caused by the war they did not attract much attention. The results, however, were notable (table 1).

In the first place it should be noted that the sum of the independently computed exponents was very close to unity, averaging .98 or only 2 percent less than true constant returns. The appropriateness of the previous \(k\) and \(1-k\) function was, therefore, reinforced as a first approximation.

The relative smallness of the standard errors of estimate gave a further corroboration to the equation. The standard errors of both \(k\) and \(j\) were found to be only .02. This was only one thirty-third the exponent of \(L\) and one-sixteenth the exponent of \(C\). Another of the criticisms of Frisch and Mendershausen, therefore, was still further weakened.

Work was stopped for some years because of the war and its aftermath. In 1947, however, Gunn and I joined forces again. I had been elected president of the American Economic Association and determined to write my address on the production function covering both the theory and an empirical analysis of the problem. Gunn obtained a leave from the government. And so, 20 years after the initial Cobb-Douglas effort, we continued on our course. More cross-section studies, both for the United States and especially for Australia, were launched. The years 1889 and 1899 with 695 observations in all were added in the American study, making with the pre-1942 analysis a total based on 2,185 industry observations. The values of \(k\) for 1889 and 1899 were .51 and .62 and for \(j\) .43 and .33. The sum of the exponents was, therefore, .94 and .95. Since the probable errors for \(k\) were .03 and .02, and for \(j\) were the same, the results strongly indicated at least an approach to true constant returns %[Page 907]
\begin{table}[!t]
\centering
\footnotesize{
\caption{Production Function for Australia, Selected Fiscal Years}%
\label{tab2}%
\begin{tabular}{p{0.25\textwidth}p{0.12\textwidth}p{0.07\textwidth}p{0.07\textwidth}p{0.07\textwidth}p{0.07\textwidth}p{0.07\textwidth}}
\hline
Cross-Section Studies and Fiscal Year & Observations \((N)\) & Values of~\(k\) & SE of \(k\) & Values of~\(j\) & SE of \(j\) & \(k + j\)\\
\hline
Australia: &  &  &  &  &  &  \\
\hspace{3mm} 1913 & 85 & .52 & .05 & .47 & .05 & .99\\
\hspace{3mm} 1923 & 87 & .53 & .05 & .49 & .05 & 1.02\\
\hspace{3mm} 1927 & 85 & .59 & .05 & .34 & .04 & .93\\
\hspace{3mm} 1935 & 138 & .64 & .04 & .36 & .04 & 1.00\\
\hspace{3mm} 1937 & 87 & .49 & .04 & .49 & .04 & .98\\
Victoria: &  &  &  &  &  & \\
\hspace{3mm} 1911 & 34 & .74 & .08 & .25 & .11 & .99\\
\hspace{3mm} 1924 & 38 & .62 & .08 & .31 & .10 & .93\\
\hspace{3mm} 1928 & 35 & .59 & .07 & .27 & .09 & .86\\
New South Wales: &  &  &  &  &  & \\
1934 & 125 & .64 & .04 & .34 & .03 & .99\\
Average of all commonwealth and state studies & 714 & .6 & .06 & .37 & .06 & .97\\
Average of commonwealth studies only & 482 & .55 & .04 & .43 & .04 & .98\\
Average of state studies only & 232 & .65 & .07 & .29 & .08 & .94\\

\hline

\end{tabular}
}
\end{table}
but with a continuing tendency during the third of a century for the sum of \(k + j\) to be slightly less than unity.

Australian Commonwealth cross-section studies were completed for fiscal years 1913, 1923, 1927, 1935, and 1937, while we also covered the state of Victoria for fiscal years 1911, 1924, and 1928, and New South Wales for fiscal 1934 (table 2).

An analysis of these figures indicates the following:
\begin{enumerate}[{1.}]
\item The sum of the exponents for labor and capital closely approached unity, or true constant returns, but were nearly always slightly below unity.
\item The standard errors of \(k\) were comparatively slight, ranging from one-tenth to one-fifteenth the value of \(k\). The standard errors of \(j\) for the commonwealth and New South Wales were also relatively small, while those for Victoria for both \(k\) and \(j\) were much larger.
\item The average exponent of labor, or \(k\), for all nine Australian studies was .60; and the exponent for capital, or \(j\), was .37. The \(k\)'s for the commonwealth studies only were somewhat lower (average of .55) and the \(j\)'s somewhat higher (average of .43). The averages in the four state studies were in reverse. That of the \(k\)'s was higher (average of .65) and of the \(j\)'s lower (average of .29).
\end{enumerate}

A supplementary study by George Brinigar and Keith Campbell for Queensland in 1937--38 found \(k\) to have a value of .58 and \(j\) one of .45.

If we take .60 as the most probably ``normal'' value of \(k\) in Australia, %[Page 908]
and .37 for \(j\), the elasticity of the marginal productivity curve for labor was somewhere around 2.7 and for capital about 1.7.

But we did not stop with Australia. We pushed on to include Canada, New Zealand, and South Africa. We carried out four cross-section studies for Canada covering 1923, 1927, 1935, and 1937 which included a total of 659 industry observations (Daly and Douglas 1943). In addition, G. W. G. Browne made two cross-section studies for South Africa in 1937--38, including in all 102 industry observations (Browne 1943); J. W. Williams covered 61 industries in his cross-section study of New Zealand manufacturing in 1938--39 (Williams and Douglas 1945); Max Brown, in his unpublished doctoral dissertation at Cambridge University, covered New Zealand manufacturing during the 18 years from 1915--16 to 1935--36 (the war year 1917--18 was omitted); J. W. Williams also covered 18 years from 1923 to 1940; K. S. Lomax studied the British economy in 1924 and 1930 (Lomax 1950); and C. E. V. Leser studied British coal mining between 1943 and 1953 (Leser 1955). The results for the South African, Canadian, New Zealand, and U.K. cross-section studies are shown in table 3.

In South Africa the value of \(k\) when blacks and whites were combined in 17 industries was .66, when separated the value of \(k\) was .65. The \(j\)'s were, respectively, .32 and .37, or almost identical values. They were, moreover, very similar to the \(k\)'s and \(j\)'s which had been discovered for American manufacturing and were not greatly different from the averages for all the Australian studies. The values of \(k\) were, however, consistently lower and the \(j\)'s higher in Canada than in the United States and Australia. The \(k\)'s ranged from .43 to .50 averaging .47, while the \(j\)'s ran from .48 to .58 averaging .52.

The New Zealand cross section of industry observations also indicated a \(j\) which was higher than its \(k\) (i.e., \(k\) = .46; \(j = .51\)). This was also true of the \(P=bL^kC^j\) formula when used for the time series. When the \(P=bL^kC^{1-k}\) formula was used, however, the \(k\) was higher, being .51 for 1915--35 and .54 for 1923--40.

Surveying this mixed group as a whole, we were reassured to find that the sum of the exponents of \(k + j\) approximated unity, but was very slightly under 1,0. The studies by Lomax were least consistent with a finding of constant returns to scale, but they are difficult to evaluate since no standard errors were reported.

The standard errors of \(k\) in the four cross-section Canadian studies were all .04, or from one-eleventh to one-twelfth the values of \(k\). The standard errors of \(j\) were also uniformly .04, or from one-twelfth to one-thirteenth the values of \(j\). The standard errors in the case of the two South African studies were somewhat higher, amounting in all three cases to .08. This was one-eighth the value of \(k\) and one-fourth to one-fifth the value of \(j\). %[Page 909]

\begin{table}[!t]
\centering
\footnotesize{
\caption{Production Function for Other British Commonwealth Countries}%
\label{tab3}%
\begin{tabular}{p{0.25\textwidth}p{0.12\textwidth}p{0.07\textwidth}p{0.07\textwidth}p{0.07\textwidth}p{0.07\textwidth}p{0.08\textwidth}}
\hline
\hline
Country and Year of Interindustry or Cross-Section Studies & Industry Observations \((N)\) & Value of \(k\) & SE of \(k\) & Value of \(j\) & SE of \(j\) & Value of \(k + j\)\\
\hline
South Africa: &  &  &  &  &  & \\
\hspace{3mm} 1937--38* & 17 & .66 & .08 & .32 & .08 & .98\\
\hspace{3mm} 1937--38$\dagger$ & 85 & .65 & \dots & .37 & .08 & 1.02\\
Canada: &  &  &  &  &  & \\
\hspace{3mm} 1923 & 167 & .48 & .04 & .48 & .04 & .96\\
\hspace{3mm} 1927 & 163 & .46 & .04 & .52 & .04 & .98\\
\hspace{3mm} 1935 & 165 & .50 & .04 & .52 & .04 & 1.02\\
\hspace{3mm} 1937 & 164 & .43 & .04 & .58 & .04 & 1.01\\
New Zealand: &  &  &  &  &  & \\
\hspace{3mm} 1938--39$\ddagger$ & 61 & .46 & \dots & .51 & \dots & .97\\
United Kingdom: &  &  &  &  &  & \\
\hspace{3mm} 1924$\mathsection$ & \dots & .72 & \dots & .18 & \dots & .90\\
\hspace{3mm} 1930$\mathsection$ & \dots & .73 & \dots & .13 & \dots & .86\\
\hspace{3mm} 1943--53$\|$ & 88$\#$ & .42 & .13 & .60 & .08 & 1.02\\
\hspace{3mm} 1943--53$\|$ & 99$\#$ & .51 & .12 & .49 & .08 & 1.00\\

%1937--38*	17	.66	.08	.32	.08	.98
%1937--38†	85	.65	\dots	.37	.08	1,02
%Canada:						
%1923	167	.48	.04	.48	.04	.96
%1927	163	.46	.04	.52	.04	.98
%1935	165	.50	.04	.52	.04	1,02
%1937	164	.43	.04	.58	.04	1,01
%New Zealand:						
%1938--39‡	61	.46	\dots	.51	\dots	.97
%United Kingdom:						
%1924§	\dots	.72	\dots	.18	\dots	.90
%1930§	\dots	.73	\dots	.13	\dots	.86
%1943--53ǁ	88#	.42	.13	.60	.08	1,02
%1943--53ǁ	99#	.51	.12	.49	.08	1,00
%* Study by Browne.
%
%† Study by Browne with whites and blacks separated.
%
%‡ Study by Williams.
%
%§ Study by Lomax.
%
%ǁ Study by Leser.
%
%# This study combined time series and regional cross-section observations from British coal mining.
%\DefineFNsymbolsTM{myfnsymbols}{% def. from footmisc.sty "bringhurst" symbols
%  \textasteriskcentered *
%  \textdagger    \dagger
%  \textdaggerdbl \ddagger
%  \textsection   \mathsection
%  \textbardbl    \|%
%  \textparagraph \#
%}%
%\setfnsymbol{myfnsymbols}
%
%\footnote{Study by Browne.}\footnote{Study by Browne with whites and blacks separated.}\footnote{Study by Williams.}
%\footnote{Study by Lomax.}\footnote{Study by Leser.}\footnote{This study combined time series and regional cross-section observations from British coal mining.}

\hline
\end{tabular}
\raggedright{
\par
$\ast$ Study by Browne.
\par
$\dagger$ Study by Browne with whites and blacks separated.
\par
$\ddagger$ Study by Williams.
\par
$\mathsection$ Study by Lomax.
\par
$\|$ Study by Leser.
\begin{flushleft}
$\#$ This study combined time series and regional cross-section observations from British coal mining.
\end{flushleft}
}
}
\end{table}

Another test of the consistency of the hypothesized relationship calls for an examination of the residuals from the estimated production function.\footnote{The Bronfenbrenner-Douglas study first introduced an analysis of the deviations of the actual from the theoretical as measured by the standard errors of estimate.} We would expect positive residuals in industries characterized by (\emph{a}) monopoly or highly imperfect competition or (\emph{b}) expanding demand where the demand curve as a whole was shifting to the right. Such industries would have higher prices than would be observed in the long run under perfect competition. Since output was measured in value terms, this will lead to positive residuals. Of 49 industries with positive residuals of more than 2 SE, we found 19 industries to be in the first group including wood engraving, gold and silver reducing, lapidary work, music publishing, glucose, starch, linseed oil, patent medicines, tin plate, brass, and lead. At least six were in the group characterized by expanding demand, namely, cordials and flavoring syrups (1909, 1914, 1919), oleomargarine (1914), perfumery (1919), and washing machines (1919). There were also at least 12 industries which were characterized both by imperfect competition and by expanding demand. These were airplanes (1914), chewing gum (1919), cigars and cigarettes (1919), fountain pens (1914), photographic supplies and equipment (1904, 1909, and 1914), cash registers (1889), smelting and refining copper (1899, 1904), typewriters and supplies (1889). All these taken together accounted %[Page 910]
for 37 of the 49 high positive residuals. We concluded, therefore, that a very large majority of the major positive residuals, so far as the United States was concerned, were from precisely those industries that one would expect on a priori grounds. This strengthened the case for the production function as a description of normal competitive relationships.

Because of limited time, we did not make as detailed an inquiry into the 38 major negative residuals. We would expect them to be from industries characterized by overexpansion of the stock of labor or capital, by declining demand, or by the use of predominantly low-skilled labor. Fifteen were found to be industries where the supply of labor was overabundant. Among these industries were 10 cases in the flax, hemp, linen, jute, and oakum group. Three more were connected with cotton, which tends to use low-skilled labor. Two, hammocks (1889) and nets and seines (1914), were in allied industries while three more, charcoal, waste, and the canning of oysters, were highly disagreeable. This made a total of at least 21 of the major negative residuals which were due to an oversupply of labor or the use of low-skilled labor. Several more were due to contracting demand. Among these were grindstones, millstones, and hooks and eyes. We regret that it was not possible to make as full an analysis of this class of cases because of lack of time caused by developments which I shall later narrate.

But enough evidence was accumulated to indicate that most of the major deviations of the actual from the theoretical values of \(P\) were explainable by dynamic departures from ``normal'' which, whether for good or evil, caused the ``productivity'' to depart markedly from the formula as a description of ``normal'' conditions. The case for the use of the function was, therefore, further strengthened. Moreover, since the influences of imperfect competition, expanding demand, oversupplies of labor, and contracting demand caused so many of the large errors, it seemed probable that many of the lesser deviations were also caused by those same factors.

Charts showing the distributions of the actual products from about the line of the theoretical relationship were prepared and shown for all 15 American and dominion studies, but because of restrictions of space and expense only three could be published in the address. With one or two exceptions the records of the total of 3,558 observations strongly corroborated both the formula and function.

I presented all of these findings in my presidential address to the American Economics Association in Chicago in 1947. At the conclusion of my address, I invited critics to check my statistics and conclusions and offered to put the necessary data at their disposal. To the best of my knowledge this has not been taken advantage of. Certainly, the previous objection that the conclusions were based on an insufficient number of cases could no longer be advanced.  
%[Page 911]

\section*{The Years after the 1947 Address}

On the very night of my presidential address in Chicago, the Democratic State Committee was meeting in that city to select candidates for U.S. senator and for governor. Just as I had finished dressing and was about to go down with my wife to the meeting, the telephone rang. It was my friend, Colonel Jack Arvey, who told me that I had just been nominated for senator and asked if I would accept. Getting a hurried nod from my wife, I thanked him and the committee and pledged that I would do my best. Then, as we walked down to the meeting hall, I remarked that it might mean the end of my scientific studies, and I found myself repeating a line from Othello, ``O farewell forever now, the tranquil life.'' It had not been tranquil, but it was to prove even less so.

This is not the place to describe what followed except to say that during my 18 years in the Senate, and for two frustrating years as chairman of a presidential commission, I was unable to work on the production function. Then I was put out of commission for another year by the Asiatic flu, pneumonia, a heart attack, and finally, a moderately severe stroke. As I recovered, I picked up the subject once more and discovered that the old opponents were now relatively silent, that the Cobb-Douglas function was being widely used, and that a host of younger scholars led by my former student, Paul Samuelson, his colleague Solow, and Marc Nerlove, the son of my friend and former colleague, Samuel Nerlove, were all pushing forward into new and more sophisticated fields. I first urged younger scholars to carry out empirical studies for which there was a plethora of evidence in Canada, Australia, and New Zealand. Together with Senator Proxmire, I urged the American statistical authorities and the Congress to resume the collection of capital statistics which the economists and statisticians had persuaded them to drop in 1920--21. Seeing no visible signs that this advice was being followed, Gunn and I then resolved to do it ourselves. So nearly a half century after the original Cobb-Douglas study, and almost a quarter of a century after my presidential address, we launched a fourth sortie into the production function.

We chose seven more recent years in Australia. These were fiscal years 1956, 1957, 1964, 1965, 1966, 1967, and 1968.\footnote{We also obtained U.S. data for 1967 from Rey B. Madoo, who had employed them in his forthcoming thesis (``Production, Efficiency, and Scale in U.S. Manufacturing [1967], an lnter-Intra-lndustry Analysis,'' University of California, Berkeley). Using these data to estimate an interindustry production function for 50 industries, I found \(k = .60\) and \(j = .40\), with an \(R^2\) of .95.} About 160 industries were covered each year or 1,123 in all. These, when combined with the earlier studies, made a total of 2,496 for the cross-section analysis of the British Commonwealth and, with the 2,185 American cross-section %[Page 912]
\begin{table}[!t]
\centering
\footnotesize{
\caption{Production Function for Australian Manufacturing, 1956, 1957, 1964--68}%
\label{tab4}%
\begin{tabular}{lcccccc}
\hline
\hline
Fiscal Year & \(N\) & \(k\) & SE of \(k\) & \(j\) & SE of \(j\) & \(k + j\)\\
\hline
1956 & 159 & .615 & .03 & .365 & .02 & .980\\
1957 & 159 & .610 & .03 & .381 & .02 & .991\\
1964 & 163 & .595 & .03 & .396 & .03 & .991\\
1965 & 161 & .576 & .03 & .414 & .03 & .990\\
1966 & 161 & .562 & .03 & .434 & .03 & .996\\
1967 & 160 & .575 & .03 & .425 & .03 & 1.000\\
1968 & 160 & .536 & .03 & .456 & .03 & .992\\
\hline
\end{tabular}
}
\end{table}
observations, a grand total of 4,681. The contention that our conclusions were based on an insufficient number of cases would hardly seem to be tenable after all this.

The values of \(k\) and \(j\) and their probable errors of estimate for the more recent years are shown in table 4. The results are a corroboration of what had been found before. The values of \(k + j\) tend roughly to equal unity, indicating a tendency toward true constant returns. But once again such slight deviations from unity as exist are on the minus side. These deviations, however, are smaller than the standard errors of the estimated coefficients.

Let us turn now to an examination of the residuals in these more recent years.\footnote{If the residuals from the regressions were distributed according to the normal distribution, we should have observed roughly 68 percent of them which were equal or smaller in size than 1 SE. In general, the residuals from the work reported here were distributed such that 70--75 percent of them were within 1 SE of the estimate.} Those industries whose estimated product was more than 2 SE away from actual product remained fairly constant over the 7 years. Actual product fell short of estimated for arms and ammunitions producers in all 7 years and for explosives manufacturers in 5 of the 7 years. Actual product exceeded estimates for pharmaceuticals, tobacco, soap, candles, and gramophones in all 7 years. Since all of these are industries that advertise heavily, it is likely that the omission of goodwill from the capital stock has contributed to the underestimates in these cases. They also were probably characterized by imperfect competition.

The relationship of \(W/P\) to \(k\) is very important from the standpoint of distribution theory. Table 5 sets forth these values for 7 years in the 1950s and 1960s. The relationship between \(k\) and \(W/P\) is quite strong. For the 2 years of the fifties \(W/P\) was only from 1 to 3 percentage points below \(k\), while in the sixties the gap ranged from 2 to 7 percentage points.

We should not let these minor differences obscure the fact that a substantial degree of equality between \(k\) and \(W/P\) has been attained. That is the central fact, and it both gives further corroboration to the production function and tends to show that the distribution of the %[Page 913]
\begin{table}[!t]
\centering
\footnotesize{
\caption{Relationship between \(\frac{W}{P}\) and \(k\) for Australian Manufacturing, 1956,~1957, 1964--68}%
\label{tab5}%
\begin{tabular}{p{0.45\textwidth}p{0.15\textwidth}p{0.15\textwidth}}
\hline
Fiscal Year & \multicolumn{2}{c}{Production Function for Total Production}\\
& \(k\) & \(\frac{W}{P}\)\\
\hline
1956 & .615 & .602\\
1957 & .61 & .581\\
1964 & .595 & .527\\
1965 & .576 & .53\\
1966 & .562 & .528\\
1967 & .575 & .517\\
1968 & .536 & .514\\
\hline
\end{tabular}
}
\end{table}
product closely conforms to what, in a largely competitive society, we would expect the marginal productivity of labor to produce. Both productivity and distribution therefore, mutually reinforce each other.

There would be a still closer degree of agreement between \(W/P\) and \(k\) (or \(k/[k + j]\)) if it had been possible to include the payments to all outside workers. Instead, as has been pointed out, \(W\) does not include outside salesmen, stationary or traveling. Nor does it cover carters in the employ of the manufacturing establishments, nor storemen who sell products at retail from the factories. These men are necessary parts of the production process. But they receive wages and salaries which have to be paid before interest and profits can be distributed. They, therefore, should be added to \(W\) to give a truer picture of the share received by labor. By how much their inclusion would raise \(W/P\) is of course unknown. But, in my opinion, it would not be by enough to eliminate the difference between \(k\) and \(W/P\).

\section*{What of the Future?}

The results of this study lend further corroboration to the accuracy of the production function as a description of manufacturing production and as a determinant of the distribution of the product --- which is a separate but allied subject.

One might still object that different industries have different production functions, but the degree of confirmation received by the logarithmic formulation with constant returns to scale in interindustry studies is very striking indeed. A number of studies have been performed that use cross-sectional observations on individual industries. They avoid the possible problem of using diverse industries to estimate a single function, but their results are also in accord with those reported here. 
 
%[Page 914]

The most ambitious of these is by Benjamin Klotz, who used data for 17 four-digit manufacturing industries in the United States for 1957 and 1963 \cite{Klotz:1}. While his estimated coefficients vary considerably, and while most industries show decreasing returns to scale, Klotz notes that no industry conclusively rejects the hypothesis of constant returns to scale. These are the same conclusions that we have been reaching with other bodies of data for almost 50 years.

A similar study using 1963 data from individual establishments in Norway yields results very much like Klotz's \cite{Griliches:1}. Griliches and Ringstad estimate a coefficient for labor of .865 and one for capital of .199, when all their observations are pooled. Thus, they find very slightly increasing returns to scale. They conclude ``that it is very hard to improve upon the simple Cobb-Douglas form.''

A considerable body of independent work tends to corroborate the original Cobb-Douglas formula, but, more important, the approximate coincidence of the estimated coefficients with the actual shares received also strengthens the competitive theory of distribution and disproves the Marxian. Many of the original objections have been answered. Some remain.

Certainly the crude Marxian slogan, ``Capital is dead labor which vampire-like lives by sucking living labor and lives the more, the more labor it sucks,'' can no longer be accepted. Capital itself is instead productive, not exploitative. It will add to production in a communistic as well as a welfare capitalistic society. The ethical question of who should own the capital and the form which such ownership should take is still open and should be decided on its merits. But some of the denunciatory appeals can and should be omitted. The issues are ethical, economic, and engineering, and the proper prophets of a new order are not Marx or Lenin, but Robert Owen and the British Fabians.

It was my hope that the present study of 7 additional years would be only the prelude to a study of nearly 60 years (1912--70). From this I hoped to get final values for Australia of both \(k\) and \(j\) and their relationship to a true \(W/P\). I could also then see more fully what changes, if any, were caused by the business cycle and whether there were long-term changes in \(k\), \(j\), and \(W/P\), as well as alteration in the pattern of residuals. But the ravages of time raise serious doubts whether I can accomplish this. I, therefore, still appeal to the younger generation of economists and statisticians for help. A quarter of a century ago there would have been distinct professional risks in such a venture. The opponents of the production function were eminent, powerful, and determined. No such danger exists today. The times favor such studies. I hope they may be made. Despite all the difficulties, I intend to push on and complete, if I can, a long-time study of the function in Australia for the years 1912--70. 

%[Page 915]

\begin{thebibliography}{00}
%%
\bibitem{Bronfenbrenner:1}
Bronfenbrenner, M., and Douglas, P. H. \foreignlanguage{english}{``Cross Section Studies in the Cobb-Douglas Function 1909.'' \emph{J.P.E.} 47 (December 1939): 761--85.}
%%
\bibitem{Browne:1}
Browne, G. W. G. \foreignlanguage{english}{``The Production Function for South African Manufacturing Industry.'' \emph{South African J. Econ.} 11 (1943): 259.}
%%
\bibitem{Cobb:1}
Cobb, C. W. \foreignlanguage{english}{``Production in Massachusetts Manufacturing, 1890--1928.'' \emph{J.P.E.} 38, no.~6 (December 1930): 705--7.}
%%
\bibitem{Douglas:1}
Cobb, C. W., and Douglas, P. H. \foreignlanguage{english}{``A Theory of Production.'' \emph{A.E.R.} 8, no.~1, suppl. (March 1928): 139--65.}
%%
\bibitem{Douglas:2}
Daly, P., and Douglas, P. H. \foreignlanguage{english}{``The Production Function for Canadian Manufacturing Industry.'' \emph{J. American Statis. Assoc.} 38 (1943): 78--86.}
%%
\bibitem{Douglas:3}
Daly, P.; Olson, E.; and Douglas, P. H. \foreignlanguage{english}{``The Production Function for Manufacturing in the United States in 1904.'' \emph{J.P.E.} 51, no.~1 (February 1943): 61--65.}
%%
\bibitem{Durand:1}
Durand, D. \foreignlanguage{english}{``Some Thoughts on Marginal Productivity with Special Reference to Professor Douglas.'' \emph{J.P.E.} 45, no.~6 (December 1937): 740--58.}
%%
\bibitem{Griliches:1}
Griliches, Z., and Ringstad, V. \foreignlanguage{english}{\emph{Economies of Scale and the Form of the Production Function}. Amsterdam: North-Holland, 1971.}
%%
\bibitem{Douglas:4}
Gunn, G. T., and Douglas, P. H. \foreignlanguage{english}{``The Production Function for American Manufacturing in 1919.'' \emph{A.E.R.} 31 (March 1941): 67--80.}
%%
\bibitem{Douglas:5}
Gunn, G. T., and Douglas, P. H. \foreignlanguage{english}{``The Production Function for American Manufacturing for 1914.'' \emph{J.P.E.} 50, no.~4 (August 1942); 595--602.}
%%
\bibitem{Douglas:6}
Handsaker, M. L., and Douglas, P. H. \foreignlanguage{english}{``The Theory of Marginal Productivity Tested by Data for Manufacturing in Victoria.'' \emph{Q.J.E.} 52 (November 1937 and March 1938); 1--36; 215--54.}
%%
\bibitem{Klotz:1}
Klotz, B. P. \foreignlanguage{english}{``Productivity Analysis in Manufacturing Plants.'' Bureau of Labor Statistics Staff Paper no.~3, U.S. Dept. Labor, 1970.}
%%
\bibitem{Leser:1}
Leser, C. E. V. \foreignlanguage{english}{``Production Functions and British Coal Mining.'' \emph{Econometrica} 23 (October 1955): 442--46.}
%%
\bibitem{Lomax:1}
Lomax, K. S. \foreignlanguage{english}{``Production Functions for Manufacturing Industry in the United Kingdom.'' \emph{A.E.R.} 40 (June 1950): 397--99.}
%%
\bibitem{Mitchell:1}
Mitchell, W. C. (ed.); King, W. I.; Macauley, F. R.; and Knauth, O. W. \foreignlanguage{english}{\emph{Income in the United States: Its Amount and Distribution. II, Detailed Report}. New York: Nat. Bur. Econ. Res., 1922.}
%%
\bibitem{Douglas:7}
Williams, J. W., and Douglas, P. H. \foreignlanguage{english}{``Production Functions.'' Econ. Rec. 21 (1945): 55--63.}

\end{thebibliography}
\end{document}